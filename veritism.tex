%!TEX program = xelatex
\documentclass[12pt,numbers=noenddot]{scrartcl}
% \usepackage[ngerman]{babel}
\usepackage[a4paper,lmargin={3cm},rmargin={3.5cm}, tmargin={2.5cm},bmargin = {2.5cm}]{geometry}
\usepackage{amsmath}
\usepackage{setspace}
\onehalfspacing
% Entfernt Blocksatz!
% \usepackage[document]{ragged2e}

\usepackage[authordate, ibidtracker=context]{biblatex-chicago}
\addbibresource{veritism.bib}
\usepackage{url}
\urlstyle{same}

% \usepackage[disable]{todonotes} % notes don't show
\usepackage[draft]{todonotes}   % notes show

% \usepackage[square]{natbib}
% \usepackage{jurabib}
% \usepackage {algorithm2e}

% Package, das die Benutzung von Old Standard erlaubt
\usepackage{fontspec}

% \setmainfont{OldStandard-Regular.otf}[
% Path = /usr/local/texlive/texmf-local/opentype/,
% BoldFont = OldStandard-Bold.otf,
% ItalicFont = OldStandard-Italic.otf]


% \setmainfont{Old Standard TT}
\setmainfont{Baskerville}
\newfontfamily\osfamily{Old Standard TT}
% \newfontfamily\osfamily{Helvetica}


\setkomafont{subject}{\small}
\addtokomafont{title}{\osfamily}

\addtokomafont{publishers}{\normalfont\small}
\addtokomafont{date}{\normalfont\small}
\addtokomafont{author}{\normalfont\small}
\addtokomafont{descriptionlabel}{\normalfont\bfseries}

\addtokomafont{disposition}{\osfamily}

% \renewcommand{\thesection}{\Roman{section}} 
% \renewcommand{\thesubsection}{\thesection.\Roman{subsection}}


\subject{}
\title {    \begin{center}
        \includegraphics[scale=0.5]{philfak.jpg}
    \end{center}
     Veritism and Epistemic Normativity }
\author{C. Friedrich}
\date{\today}

\publishers{ University of Cologne \\
    Department of Philosophy \\
    Bachelor Thesis \\

}

\makeatletter
\renewbibmacro*{cite:ibid}{%
  \iftoggle{cms@noibid}%
    {\blx@ibidreset\usebibmacro{cite}}%
    {\printtext[bibhyperref]{\bibstring[\mkibid]{ibidem}}}}
\makeatother

\begin{document}
\begin{titlepage}
\maketitle

\abstract{\textbf{Abstract:} Can epistemic normativity be plausibly explained by appealing to truth or true belief as the single fundamental epistemic value? To answer this question, I first sketch an overview of the normative surrounding and prerequisites before giving an account of what it means, exactly, to take a veritistic position, how it differs from other approaches to explain epistemic normativity, and how plausibly this account can explain guiding intuitions about value in epistemology. Veritism is used as a basis for a an account of epistemic normativity and I show how such an account could be developed. I present salient objections against and problems for a veritistic picture and try to defend the position against them, concluding that as a theory of value, veritism can plausibly be defended, however, as a bases of consequentialist accounts of justification veritism quickly runs into trouble. [TO DO: Use better words. Use the best words!]}

\thispagestyle{empty}
\end{titlepage}

\listoftodos

\tableofcontents
\newpage

\section{Introduction}

\begin{quote}
    It is not that hard to be skeptical about value. You can't see values. You can't touch them. And many think we can't even define them. Partly as a result, our disagreements about value have an interminable and intractable feel. Philosophically speaking, skepticism about value seems an easy sell.

    At least this is the case when we are talking about the type of value that most philosophers worry about—moral value. But we have other values besides moral values. And one of the most basic of these other values is the value of truth. \autocite[225]{Lynch2009}
\end{quote}


In a certain sense, and one that, for example, \textcite{Berker2013-BERETA-2} holds, epistemology is an inherently normative endeavor. To Berker the most fundamental question of epistemology is, despite the etymological origins of its name, not concerned with what makes a belief count as knowledge but is instead purely normative: “What should I believe“?

In pondering this question the similarities of epistemology to the theory of normativity and especially ethics are strikingly obvious, if only at a first superficial glance. It makes a lot of sense then, if one is to undergo the project of viewing epistemology through the lens of normativity, to at first take a look at ethical theories of value and normativity. While the properties of epistemic and ethical objects of interest may stop sooner than expected, the groundwork in ethical theory shouldn't be completely disregarded, either.


\begin{quote}
I don't know how to prove that the acquisition, retention, and use of true beliefs about matters that are of interest and/or importance is the most basic and most central goal of cognition. I don't know anything that is more obvious from which it could be derived. But I suggest that anyone can see its obviousness by reflecting on what would happen to human life if we were either without beliefs at all or if our beliefs were all or mostly false. Without beliefs we would be thrown back on instinct as our only guide to behavior. And as far as thought, understanding, linguistic communication, theorizing, science, art, religion-all the aspects of life that require higher-level cognitive processes-are concerned, we would be bereft of them altogether. And if we had beliefs but ones that were mostly false, we would constantly be led astray in our practical endeavors and would be unlikely to survive for long. \textcite[30]{Alston2005-ALSBJD}
\end{quote}

It's really unübersichlicht out there, norms and values seem to be thrown around. I wanted to put things a bit in perspective and say where this debate takes place in epistemology, what the relevant questions are, and if there 
\begin{itemize}
    \item Where are we in epistemology? Why is a normative approach important / interesting?
    \item explain very clearly and in formidable prose what it is this paper is trying to do, how it will be done, and what it accomplished (obiovusly, write it at the end).
\end{itemize}

\begin{quote}
 I suggest that the primary function of cognition in human life is to acquire true rather than false beliefs about matters that are of interest or importance to us. \textcite[29]{Alston2005-ALSBJD}
\end{quote}

\begin{quote}
What makes us cognitive beings at all is our capacity for belief, and the goal of our distinctively cognitive endeavors is truth: we want our beliefs to correctly and accurately depict the world. \textcite[7]{Bonjour1985}
\end{quote}

\todo[inline]{More quotes can be found in \textcite{Goldman2002-GOLTUO-2}.}
\footnote{As may be obvious, I presuppose a realist notion of truth. As perhaps less obvious, throughout this paper I assume that every proposition is either true or false and not neither or both. I won't discuss complications that arise for the veritist account if one supposes, say, dialetheism. Let's further suppose that eternalism about propositions is true, that is, one and the same proposition doesn't change its truth value, ever, in order not to run into complications when faced with a challenge from relativism brought forth by \textcite{Brogaard2008-BROTTA-3}.}

In Chapter 2, I outline the preliminaries for this paper. I introduce some basic concepts of the normative landscape and talk about what they have to do with the position of interest, veritism. I explain what a proper account of epistemic normativity amounts to, and how a consequentialist account of epistemic normativity grounded in veritism might look like.

In Chapter 3, I detail what veritism actually entails and how a form of veritism can be made plausible with regard to intuitions about epistemic value. As this might be quite important from the outset, I discuss this in detail to give a solid foundation for the normative epistemology. I present a salient objection to the veritist account based just on its axiology, look at defenses in the literature, and expand upon them to propose a veritist response to this challenge.

In Chapter 4, finally, I discuss the deontic or prescriptive epistemic norms in relation to veritism. I first take into consideration whether there it is even possible for deontic norms of belief to be true, as has been often and famously challenged. I then discuss how epistemic norms might look like and consider problems for and objections to them.

In conclusion,

\section{Values, Norms and Truth}\label{sec:values}

Why is it ever that anyone \emph{should} believe something, or is \emph{required} to believe, or is \emph{justified} in believing, or is \emph{rational} to believe? What is that yields some kind of normative requirement to the believer? A most natural answer, one that immediately springs to mind, is it's connection to the truth. You should believe that proposition because it is most likely to be true! (You are justified in believing that proposition because it is the only one that your evidence supports. It would be completely irrational to believe that other proposition, you don't have any reason to believe it.) What else than true belief is the epistemic goal of anyone in the business of believing anything, that is, a cognitive agent? It is this very intuitive thought that motivates postulating truth as an–or the–epistemic value.

When talking about values, the deontic or about normativity in general, it is helpful to look for established theories as a guideline. Naturally, these theories have been developed in the realm of practical philosophy, so that's where I'll turn for some clarificatory groundwork.

A good starting point to look for some guidance when talking about values are the fields of value theory or, more specifically, axiology.

\subsection{ Taxonomy of the Normative [Exposition]}

To clarify the notions, I very shortly introduce the normative terminology and how it is used it in this paper.

Normativity is roughly divided into axiology, the study of values, and deontology, the study of rules and norms. Axiology is concerned with what is good or bad and what is better or worse. Closely related is the term \emph{correctness}, which is sometimes also established as a deontic term \textcite[268f.]{Wedgwood2002-WEDTAO}. Axiological claims are therefore \emph{evaluative}. Deontology is concerned with what is required, forbidden, permissible, what one one ought to do and so on. A closely related but distinct group of terms consists of \emph{responsibility, duty, blame, praiseworthiness} and so. On some theories there is a slight, on others a significant difference between these groups of concepts. Deontic claims are prescriptive or \emph{regulative}. There can be evaluative norms, specifying what is of value, as well as deontic norms, specifying what is to be done or believed. Sometimes deontic norms of the form 'you ought to $\Phi$' are meant to be interpreted evaluatively, but I will mostly leave this complication out of the paper.

Any account of normativity aspires to explain the axiological and deontological questions. The tension between different types those theories are of some import for the present paper, and I circle back to this topic frequently.

What \emph{is} a value, then? \textcite[79]{scanlon1998} puts it thusly: “ ... a notion of how it would be best for the world to go, or of what would be best for particular people”. In addition, it is what makes evaluative sentences like “pleasure is good” come out (perhaps) true. There is something that corresponds to these sentences, for instance a property of pleasure, that gives “pleasure is good“ a status as a true sentence. Since axiology is defined above in terms of claims of the form “pleasure is good” this approach seems to beg the question. It is such a simple and perhaps primitive notion. What's left is the appeal to intuition–certain things are intuitively good, or are better than others. That's what \emph{value} refers to.

- value come in degrees
- Intrinsic Value vs Extrinsic value

So far I have not really been quite firm on the terminology when talking about values and aims or goals. Let us specify that a bit, following \textcite[344f.]{Berker2013-BERETA-2}. What is valuable in itself are not beliefs per se, but states of affairs or situations in which someone has a true belief. But of course, the value of that situation only obtains in virtue of the true belief someone has, so it makes still sense to speak of true belief as valuable. Compare that to a goal: a goal in this sense is a situation of ideal or maximized value, something similar to the ideally rational reasoner who comes up frequently in the formal epistemology literature or the situation science is supposed to reach granted indefinite time for research. So the content of a goal describes that situation towards which doing something valuable or having a true belief promotes. Not much hangs on this distinction, but I think it important to be clear about these notions. I talk more about the veritist values and content of a goal in REF and REF.
\begin{itemize}
    \item{ Instrumental Value vs. Final value }
    see \textcite{Sosa2007-SOSAVE-2} for fundamental vs derivative value. There is a difference between fundamental and final values. Something has fundamental value if its value is not derived from something else. Final value on the other hand describes something that has value \emph{simpliciter} without regard to some domain. For the present purposes, I just the qualifier \emph{epistemic} when talking about values most of the time, so that distinction can hopefully be disregarded.
    \item{ Good vs. right }
    \item{ Why talk about value? To get an account of normativity, of course! }
\end{itemize}

\todo[inline]{Shouldn't here be a desciprtion of different norms? maybe already distinguish between mistake and blameworthiness}

\subsection{ Teleology / Consequentialism [Exposition]} \label{subsec: teleology}
A theory of Normativity with a central value / central values as its fundamental principle.

The guiding motivation behind painstakingly creating a theory of value is not to stop there an be done with it, of course, but instead developing it into a full-fledged theory of normativity. Normativity includes the deontic as well, that is, the concepts of \emph{right}, \emph{reason}, \emph{rational}, \emph{ought} and so forth \autocite[21]{sep-value-theory}. Values and deontic norms are two central concepts in normative theory. Which of these two is the basic or fundamental one through which the other concept can be explained or derived is an ongoing debate and roughly divides normative theorists into two camps: consequentialists (or teleologists) and deontologists. 

Consequentialists hold that all regulative norms usually phrased with \emph{oughts} and the like obtain in virtue of being directed at a value or goal. In slogan form, they put the good prior to the right. The basic idea is to first ask “what is the best (i.e. most valuable) action / belief / situation?” and then ask “what do I do to achieve it?” to arrive at what one \emph{ought} to do. „However much consequentialists differ about what the Good consists in, they all agree that the morally right choices are those that increase (either directly or indirectly) the Good.“ \autocite{sep-ethics-deontological}

Deontologists hold that all evaluative norms obtain in virtue of being directed by a regulative norm. In slogan form, they put the right prior to the good. They start out with asking “what should I do?” and then determine the most valuable action / belief / situation on this grounds. 

So the direction of explanation is reversed in these types theories, but instances of both types aspire to account for the \emph{good} as well as the \emph{right}, in other words, aspire to present a complete account of normativity \parencite[341]{Berker2013-BERETA-2}. This distinction is, as in most fields of philosophy, not as clear-cut as it might seem though, and there are theories that neither fall clearly on one side or the other of the divide and may incorporate features of the opposed theory.

This is a very rough sketch of teleological / deontological accounts of normativity that leaves most everything to be desired, to be sure, but for my purposes it's enough give a quick overview what this distinction is all about.

\subsection{Epistemic Normativity} \label{subsec: epistemic}

Veritism is a theory about value, so in that sense it determines the questions regarding axiology in a normative framework. To be compatible with veritism, a deontic theory has to accept that the only fundamental epistemic value is truth or true belief. Veritism combined with such a deontic theory amounts to a teleologist or consequentialist account of epistemic normativity in which deontic concepts are explained by appeal to veritism.

The discussion of the normative has been quite general so far, in order to be compatible with ethics as well as epistemology. Veritism is an epistemological position, so what does that mean for normative theories about it? Most of what has been said translates directly into the \emph{epistemic domain}, meaning anything that pertains to concepts like belief, truth, knowledge and justification \autocite{David2001-DAVTAT-7}. Even \textcite[7]{grundmann2008}, who holds that conceptual analysis in epistemology precedes the normative inquiry, emphasizes the normative character of justification, one of \emph{the} central concepts in epistemology \autocite[226]{grundmann2008}. Still more prevalent is the connection of epistemological concepts to the normative in the \emph{deontic conception of epistemic justification}, presented by \textcite{Steup1988-STETDC} and argued against by \textcite{Alston1988-ALSTDC}. The deontic conception states that epistemic justification can be “cashed out“ out in terms of permissions, obligations and so forth. This goes one step beyond the claim of Grundmann, who merely notes the evaluative normative character of epistemic justification.

So the interesting question is whether veritism can account for and explain the normative character of epistemology. What are those normative concepts, and how can it be done? I see two basic different approaches:

\begin{enumerate}
    \item Determine the exact content of the goal stipulated by veritism. What exactly is the goal of epistemology? By finding out the most plausible version an exact description of the goal, one thereby acquires a putative description of the ideal epistemic state of a believer. Taking this description as a regulative ideal, one can try and work out the most plausible epistemic regulative norms, that, if followed, brings one closer to being in the ideal epistemic state\footnote{That this is not as easy as it might sound is \textcite{Gibbons2013-GIBTNO} major point. There is a huge gap in explanation between the objective goal of truth and regulative norms pertaining to the individual subject. These even come in conflict, in what he calls \emph{the puzzle}. However, \textcite{Goldman2002-GOLTUO-2} does not think this problem grave at all. [Go on A BIT and clarify own position]}.
    If and how these regulative norms so construed pertain to the concept of justification or knowledge is a different question, and, depending on one's assumptions, can follow directly or not at all.
    \item Stipulate that the normative conception of epistemology just is a feature of what the central epistemic concept, depending on your theory of choice \emph{justification, reason, evidence, rationality, warrant} or \emph{intellectual virtue} \autocite[153]{David2001-DAVTAT-7}, amount to. On this approach, what it means to be justified (or rational, reasonable..., from now on justified for short) is in some way or other derived from veritism. The value of justification is derived from the value of true belief, but whether “justified belief has anything to do with fulfilling one's epistemic obligations” is left undecided \autocite[66]{Steup1988-STETDC}.

    It is important to note that this is \emph{not} equivalent to a deontic conception of justification. Whereas the deontic conception holds that justification can only be \emph{analyzed} in deontic terms, this account does make this claim. It would, however, not really be a proper concept of epistemic deontic normativity if there wouldn't be some claims made about the connection of justification and deontic norms.
\end{enumerate}

\subsection{How Can Veritism Ground Epistemic Normativity?}\label{sec:groundingnormativity}
\todo[inline]{perhaps this section is BS or at least unnecessary}
It is perhaps not obvious why it is that, given veritism, regulative norms of belief have any normative force. How is the normative force grounded in veritism? In virtue of what are these norms authoritative? It may be that true beliefs are good, or valuable, but why should that require me, as a believer, to follow norms given provided by veritism? This problem is especially salient when additionally stating that beliefs have a constitutive aim of being true, as \textcite{Cote-BouchardForthcoming-CTBCTA} points out. His challenge goes something like this: Suppose that there is a true evaluative norm of belief, of the form
\begin{description}
    \item[TN] A belief is correct if and only if it is true. \autocite{Wedgwood2002-WEDTAO}
\end{description}
For epistemic normativity to be binding, it needs to be the case the there is necessarily a good reason to follow epistemic norms \textcite[13]{Cote-BouchardForthcoming-CTBCTA}. For TN to deliver such a verdict, we need to make the additional assumption that
\begin{description}
    \item[RTN] There is necessarily a good reason to believe correctly.
\end{description}
However, this does not follow from TN. And is has to follow from TN, since TN is supposed to–on its own–explain the force of epistemic normativity. So to say that there is a true norm of belief of this kind does not imply anything about there necessarily providing me with good reasons to believe according to that norm. For consider the analogy: “writing the same number twice in one of the columns of a Sudoku grid is incorrect relative to the constitutive norms or rules of Sudoku. Yet there might be no good reason for me to avoid that incorrect Sudoku move.” \autocite[12]{Cote-BouchardForthcoming-CTBCTA}. So although one may grant TN, RTN still does not have to obtain.

One objection that springs to mind is that it is (i) our role \emph{qua believer} that brings us to satisfy the norms and (ii) occupying the role of a believer is inescapable for any human being.



However, what has to be addressed here is that the role of a believer is aimed at a good thing, at something valuable. That is what veritism states, after all. The \emph{combination} of these claims makes all the difference: any epistemologically relevant subject is in the position that (i) she is in the role of a believer (ii) this is necessarily and not only contingently so and (iii) only true belief is correct belief. What Cote-Bouchard has shown is that neither of these conditions alone is sufficient to justify RTN. But as a believer TN applies to me, hence I have good reasons to believe correctly \emph{qua believer}. Since I'm necessarily in the role of a believer, and believing correctly is a good thing (as stated in veritism), I necessarily have a good reason to believe correctly. If that reason actually applies though or may be challenged by other principles 

If this objection works, it lifts the normativity of belief \emph{out} of the merely epistemic domain. So I do not merely have an epistemic reason, but a good reason, \emph{simpliciter}
\footnote{This does not amount to all-things-considered reason, however, as there might be other, heftier reasons that outweigh my good reason derived from truth. This, though, is another contentious topic, and some hold the position that only epistemic reasons can give me a reason to belief anything \autocite{Kelly2003-KELERA}}.
But not much hangs on this claim for the purposes of this paper, since for normativity of belief to be relevant, it is enough to just apply to the epistemic domain. This question becomes relevant again as soon as one tries to compare epistemic (or intellectual) reasons with practical (or other minds of)reasons to determine an all-things-considered reason. That is not the target of this paper, though. If this norm is in fact  only hypothetical on my role as a believer–so be it.

\section{Veritism}
\subsection{Varieties of Veritism}\label{sec:varieties}

\textcite[54]{Goldman2002-GOLTUO-2} defines veritism as  “... the unity if epistemic virtues in which the cardinal value, or underlying motif, is something like true, or accurate, belief”. Borrowing the terminology from Goldman, \textcite[360]{Berker2013-BERETA-2} proposes to define veritism as the position that „ ... our only epistemic goals are (i) the accumulation of true beliefs and (ii) the avoidance of false beliefs”. \textcite{Zagzebski2004-ZAGEVM-2} has a different name for a very similar position, what she calls epistemic value monism: “Any epistemic value other than the truth of a belief derives from the good of truth“. 

The unifying theme of these quotes is apparent: What’s central to all three accounts clearly is the notion of \emph{true belief}\footnote{It might be argued that \emph{truth} is the actual only epistemic value, assigning it to true belief would be “undue reification” \autocite{Pritchard2014}. Then, however, one has to tell a story as to why other truth-apt doxastic attitudes do not have epistemic value. This could be done by pointing out that belief is the only relevant epistemic doxastic attitude. And this would be not much different than to be clear about it and state true belief as of fundamental epistemic valuable.}. True, or accurate, belief has to be a value, goal, function or motif with a distinct role in the theory. This role is determined as cardinal, single, primary and non-derivative. In other words, veritism says that true belief has to be regarded as the only \emph{final} value. No other epistemic concept like knowledge or justification has final epistemic value. It is important to note, however, that this does \emph{not} entail that other epistemic concepts have \emph{no} epistemic value, they just don’t have \emph{final} epistemic value.

So it is perfectly compatible with veritistic accounts to say that epistemic justification is valuable and that it has a value distinct from the value of true belief. Justification does not, however, have final or intrinsic epistemic value.

\subsubsection{The Value of Believing}
That is the gist of the position: veritism claims that true belief is the only epistemic value or goal. In some sense it is an ontological claim about the domain of epistemology: Veritist describe the property of value to true beliefs, or the existence of of a unifying goal for all believers. It is somewhat removed from psychological considerations, but that was to be expected, given this inquiry into the normative.

\begin{description}
    \item[Simple Veritism:] True belief is the only state of affairs that has final epistemic value.
\end{description}

\todo[inline]{of course, this is still underspecified. What exactly is of value? Having as many true beliefs as possible? How are true and false aggregated to accommodate something like a total value?}
\todo[inline]{add a comparative notion to it.}

Can this account explain intuition about scenarios concerning epistemic value?  A clear-cut case: two propositions $P$ and $Q$ where $P$ is true and $Q$ is false–it couldn't be more obvious: believing $P$ is more valuable than believing $Q$. This very natural thought directly leads to a comparative notion which holds that true belief is more valuable or better than false belief. Simple enough.

Now what if we want to evaluate the doxastic system of a person, or her intellectual attainment, as \textcite[58]{Goldman2002-GOLTUO-2} puts it? This seems just as straightforward. The more true beliefs a person has, the more valuable the position that she occupies. So no matter the epistemic situation she am in, by acquiring more true beliefs she can better her epistemic standing. So far, so good. Goldman\textcite[59]{Goldman2002-GOLTUO-2}

All is not well, though. Consider these two:
\begin{description}
    \item \emph{Gullible Gilbert} the believes everything anyone tells tells him, anything he thinks about, in general any proposition he encounters.
    \item \emph{Precise Priscilla} the is more careful in forming her beliefs. She carefully weighs her evidence and, as a result, has a lot fewer beliefs than Gilbert, but her beliefs are largely true.
\end{description}

Intuitively
\footnote{I make a lot of assumptions about intuitions here. Granted, this is not ideal, but I tried to only incorporate cases where the intuitions seem uncontroversial, or if the intuitions seemed problematic, noted that they might be so and tried not to put too much argumentative weight on them. As these are empirical claims, of course all of my assumptions about intuitions are open to challenges from experimental philosophy.},
it is obvious that Priscilla occupies the more valuable epistemic state. Yet on the account just sketched, Gilbert amasses lots and lots of beliefs, among these many true beliefs. Hence his epistemic situation is immensely valuable. Priscilla on the other hand has fewer true beliefs to show for, her epistemic situation is therefore less valuable. This is a terrible result for the simple account.

Gilbert’s situation is an instance of what \textcite[360]{Berker2013-BERETA-2} calls \emph{epistemic recklessness}. 

To avoid this result, it seems only natural to expand the epistemic goal to consist of to goals, really: ”...the twin goals of acquiring true beliefs and avoiding false ones“ \textcite[339]{Berker2013-BERETA-2}. This dualistic rendition of epistemic value has already been proposed by \textcite[17]{James1896-JAMTWT-19}, who coined the phrase: ”Believe truth! Shun error!“.

\begin{description}
    \item[Dualistic Veritism:] True belief and the avoidance of error are the only two states of affairs that have final epistemic value.
\end{description}
\label{SECTION SO UND SO}
What does it mean to avoid error? An error in this sense \textcite[362]{Berker2013-BERETA-2} is a false belief. But in everyday language, disbelieves in true propositions are errors as well. I discuss this complication later on \ref{SECTION SO UND SO}. What is avoidance, then? In a narrow reading, only disbelieving or suspension of judgment towards a false proposition count as avoiding errors. Interpreted more loosely, just being ignorant about a given proposition can count as avoidance, too. This relates to the criterion that determines which propositions actually contribute to the epistemic value, I discuss it in \ref{SECTION SO UND SO}.

So to maximize my epistemic value, I aim to have as many true beliefs and as few false beliefs as possible.

This account fares better with respect to Gilbert and Priscilla. As Gilbert does have some true beliefs that count towards a valuable position, Priscilla’s carefulness now brims with epistemic value as she avoids most of the errors that Gilbert makes. So this intuition, at least, can be explained.

But let's revisit one of the key merits of veritism at the core of any deontic framework. One central motivation is to unify all epistemic evaluation. This approach reaches as far back as Socrates, who proclaimed: ”Virtue is one!” and meant it quite literally \textcite{penner1973}, entailing that what Socrates regarded as virtues where really all the same thing. So bravery, wisdom, temperance, justice, piety and even knowledge amount to something equivalent, if not identical.
\textcite{Goldman2002-GOLTUO-2} picks up the ball in his self-appointed task to unify all epistemic virtues\footnote{Now, virtues are not the same thing as values or goals, and I don't want to get into a discussion about virtue epistemology at all here. What Goldman concedes, though, is that his conception of epistemic virtues builds upon or at the very least is compatible with that form of consequentialism value-directed accounts present us with.}. As one might think, he is not so quick to drop value monism to make it two, that is to switch to a dualistic account, which would pretty much mean to give up the idea of a single unifying epistemic virtue.

\textcite[58]{Goldman2002-GOLTUO-2} proposes an a little more complex form of veritism which takes its motivation from the model of partial belief, or degree of belief, or levels of confidence, or credences. Let's first get clear on the different notions of belief he employs:

\begin{quote}
    First, we can use the traditional classification scheme which offers three types of credal attitude toward a proposition: believe it, reject it (disbelieve it), or withhold judgment. I call this the trichotomous approach. Second, we can allow infinitely many degrees or strengths of belief, represented by any point in the unit interval (from zero to one). I call this scheme the degree of belief (DB) scheme. \textcite[88]{Goldman1999-GOLKIA}
\end{quote}

So nothing non-standard here. Note, that we want to account for each of the doxastic attitudes belief, disbelief and suspension of judgment.

In the model of degree of beliefs, it is not mere true belief that has final epistemic value. Instead, the value of a credence in a true proposition derives its value as a function of its degree: maximal value if the degree is maximal, and minimal value if the degree is minimal. This yields a slightly different comparative notion than before: A degree of belief in a true proposition is more valuable than another degree of belief simply if it is higher. Given a true proposition $P$, your credence of $0.9$ is more valuable than mine of $0.4$, simple as that.
Goldman then supposes a workable way to translate degrees of beliefs into full beliefs, suspension of belief, and disbelieve, by some threshold measure that is left unspecified. He can then compare: ”... believing a truth carries more veritistic value than suspension of judgment; and suspension of judgment carries more veritistic value than disbelief“. This leads to a comparative notion:

\begin{quote}
    If a person regularly has a high level of belief in the true propositions she considers or takes an interest in, then she qualifies as “well‐informed.” Someone with intermediate levels of belief on many such questions, amounting to “no opinion,” qualifies as uninformed, or ignorant. And someone who has very low levels of belief for true propositions—or, equivalently, high levels of belief for false propositions—is seriously misinformed. \textcite[58]{Goldman2002-GOLTUO-2}
\end{quote}

In this way, Goldman concludes, the veritistic account can accommodate everything an account with two epistemic goals instead of one can. In particular, he claims that having very low levels of belief for true propositions is \emph{equivalent} to having high level of belief in false propositions, at least insofar as the epistemic value is concerned. On this account, a credence of $0.2$ in a true proposition has the same epistemic value as a credence of $0.8$ in a false proposition, and both counts equally towards a person being seriously misinformed. For Goldman, this notion of comparative value does justice to our intuitions regarding relevant cases. It is very reminiscent of the concept of utility in utilitarianism, which is “sum of the action's positive and negative consequences, that is, the pain and pleasure caused by the action.” \autocite{depaul_value_2001}, in that it subsumes the positive value of true belief and the negative value of false belief under one unifying concept.

\begin{description}
    \item[Accuracy Veritism:] The value of a belief is completely determined by its accuracy.
\end{description}

\subsubsection{The Value of Disbelieving}

However, as I see it, there are problems with this translation of conclusions drawn with the notion of degree of belief to the notion of full belief that Goldman does not seem to take into account. It might be interesting to have a look at the notion of belief and disbelieve that Goldman employs, as it does a lot of work in explaining the value of the doxastic attitudes.

Goldman claims that disbelieving a proposition $P$ is equivalent to believing a proposition non-$P$ \textcite[58]{Goldman2002-GOLTUO-2}, so that I disbelieve a proposition $P$ if and only if I believe a proposition non-$P$. That is, whenever it is true that i disbelieve $P$, it is also true that I believe non-$P$, and whenever it is false that i disbelieve $P$, it is also false that i believe non-$P$, and \emph{vice versa}. I presume this is a consequence from the translation of the notion of degree of beliefs to the notion of full beliefs. Given scaling of degree of beliefs on the unit interval, whenever I believe $P$ to the degree $c$, I also believe non-$P$ to the degree $1-c$. For example: I am pretty unconvinced of $P$, my credence is $0.1$. Hence, my credence in non-$P$ is $0.9$. Translated back to full beliefs, this would then amount to belief and disbelieve, if $c$ is sufficiently low for disbelief. So Goldman's claim about equivalence would follow. What shouldn't be disregarded, though, is that this only applies to agents that obey the axioms of probability theory, in many theories a necessary condition on rationality of degree of beliefs. It is still very much conceptually possible to believe $P$ to the degree $0.8$ \emph{and also} believe non-$P$ to the degree, say $0.7$. I wouldn't be quite rational to do so, of course, but it is certainly possible. Now, Goldman does not state conceptual identity, granted. Equivalence is still a very strong claim, but: my believing $P$ to the degree $c$ \emph{does not entail} my believing non-$P$ to the degree $1-c$ (or vice versa). So, translated back to full beliefs, the equivalence of disbelieving that $P$ and believing that non-$P$ does not follow from the notion of degree of beliefs\footnote{The notion of translating between different models of belief formation is not self-explanatory and seems in need of an independent argument. Let's suppose, for the moment, that such a sufficient argument has been presented.}.

One could object to this that this is not at all a move from degree of beliefs to full beliefs. Instead, is just very naturally follows from thinking about doxastic attitudes. What else should disbelieving $P$ amount to, if not believing that non-$P$? When I disbelieve $P$, I hold that $P$ is false. But believing that $P$ is false \emph{just is} believing that non-$P$ is true. The argument looks something like this:

\begin{enumerate}
\item [P1] If I disbelieve that $P$, I believe that <$P$ is false>. 
\item [P2] <$P$ is false> is equivalent to <non-$P$ is true>.
\item [P3] Belief is closed under single premise logical entailment.
\item [C1] If I disbelieve that $P$, I believe that <non-$P$ is true>. (From P1 to P3)
\item [P4] Believing that <$Q$ is true> \emph{just is} to believe that $Q$.
\item [C] Therefore, If I disbelieve that $P$ then I believe that believe that non-$P$. (From C1 and P4)
\end{enumerate}

A similar argument may be sketched for the opposite direction of entailment.

\todo[inline]{There may be problems or an objection with frege senses and thoughts. Look into it if time}
But I would hold that I presumably can disbelieve $P$ and also not believe non-$P$, that is, disbelieve non-$P$ or suspend judgment towards non-$P$. There is no entailment relation between either of these doxastic attitudes. To take a familiar example: I can disbelieve that Hesperus shines without believing that Phosphorus doesn't shine, even though, as is common knowledge, Hesperus and Phosphorus denote the same object and therefore <'Hesperus shines' is false> and <'Phosphorus does not shine' is true> are equivalent propositions. What goes wrong in the argument? P1 may just follow from your definition of disbelieve. I won't argue against that here, or ever, so this premise seems fine. P2 just is a necessary statement given that every Proposition is either true or false. P4 seems like a truism, although one might argue against it on grounds that a conception of truth or true belief is not necessary for a subject to hold true beliefs, however, let's grant this as well. P3, on the other hand, is the most dubious candidate. While it is a hot topic whether knowledge is closed under entailment\footnote{See, for example, \textcite{Dretske2005-DREIKC}.}, and even so for ration belief\footnote{See \textcite{KyburgJr1970-KYBC-2} for a case against.}, it is a lot less plausible that \emph{mere belief} is closed under (single premise) entailment. For I can, of course, believe that $P$, where it is some necessary truth that If $P$, then $Q$, but still fail to grasp that $Q$ obtains. Without this or a similar premise\footnote{Perhaps the premise that it is closed under \emph{obvious} single premise closure, but then, I think, the same counterexamples work. But isn't closure under logical entailment too strong a requirement? What's really at issue is closure under logical equivalence, and that seems a lot more plausible? Well, even in this case, the presented counterexample about Hesperus and Phosphorus would work.} the argument just isn't sound.

So, I suggest that yes, of course, <$P$ is false> is equivalent to <non-$P$ is true>, however, disbelieving that $P$ is not equivalent to believing that non-$P$.

Goldman can still retreat to the position that the \emph{epistemic value} of disbelieving $P$ is equivalent to the \emph{epistemic value} of believing non-$P$. Let's suppose $P$ is false, and hence non-$P$ true, then on this account, disbelieving $P$ has epistemic value–the corresponding credence is very close to the actual truth value– and believing non-$P$ has value, since one believes a true proposition, namely non-$P$. So what can be accounted for is the dual value of disbelieving a falsehood and believing a truth. This position seems a lot more plausible, then. Perhaps it is what Goldman had in mind, anyway. What does it help veritism with? Consider the case:

\begin{description}
   \item \emph{Stephanie Skeptical} is in a very unfortunate situation. Her questionable colleagues all present her with unconvincing propositions. As a result, she comes to disbelieve most of them. And rightly so, all of them are false.
\end{description}

We intuitively want to say that Stephanie's disbeliefs contribute something epistemically valuable to her situation. Yet, on the simple account of veritism, I could only ascribe value to her disbeliefs if I presume the equivalence stated by Goldman, that her disbelief in the false proposition $P$ entails a believe in the true proposition non-$P$. On the dualistic account, her disbeliefs would count as an avoidance of error only if (a) the equivalence claim above is supposed true, or (b) error is meant to encompass disbelieve in false propositions as well. Option (a) would presuppose to much to be incorporated into the value directly, for my taste, whereas option (b) seems \emph{prima facie} to be a plausible candidate to deal with this problem, however, isn't emphasized that much in the literature.

Goldman's accuracy account is partly designed to handle exactly this type of case. Stephanie's disbeliefs all have epistemic value equivalent to that of true beliefs. Hence, her epistemic situation is valuable, agreeing with intuition.

\subsubsection{The Value of Suspending Judgment}

What about suspension of judgment? Given a true proposition $P$, it is most valuable to belief that $P$. It seems less valuable to suspend judgment towards $P$, and it is even less valuable to disbelieve $P$. Conversely in the case of a false proposition $Q$: disbelieving is most valuable, suspension less so, and believing that $Q$ least valuable. Intuitively then, there is a clear ranking of these types of doxastic attitudes. On the simple account, none of the features of the ranking save true belief can be explained. The dualistic account with the above modification regarding error gets two out of three right, it does not say anything about suspension of judgment though. What should it count as? Is it some form of true-at-least-a-bit belief, or should it be regarded as an error? Neither of these seem plausible at all. So if we are to include suspension of judgment into our notion of doxastic attitudes, both the simple and the dualistic veritistic account can't seem to handle the intuitions regarding its value. The accuracy veritist is able to cope a lot better, as suspension is directly incorporated into the comparative notion that Goldman spelled out.

However, there might still be a problem: \todo{Insert Dominiks Argument here} i suggest that the translation to a model of full belief does not work in all respects: A degree of belief of 0.5 is not the same a suspension of judgment, which does not entail a doxastic attitude towards a proposition \footnote{I owe this point to \textcite{Balg2018} }

So far, it seems that accuracy veritism comes out on top, at least if we only consider cases in which there is a fixed set of propositions against which we evaluate different intellectual attainments. Consider again the case of Gilbert and Priscilla: Gilbert is right quite often, but wrong even more often. Priscilla is right in most of her beliefs, but has fewer of them, so is not right as often as Gilbert is. On Goldman's account then, one accumulates epistemic value by believing truths and disbelieving falsehoods, but also by suspending judgment, regardless of the truth value of the proposition. Isn't it still the case that Gilbert accumulates more value by believing all those true propositions? He accumulates a lot, that much is correct, but reading Goldman charitably we might ascribe to Priscilla that she withholds judgments in many of the cases which Gilbert gets wrong. Thus, she accumulates lots of additional value by being indecisive. I would concede that this point is controversial, for it is not the same thing to withhold judgment and to be ignorant about a proposition. Being ignorant does not promote any epistemic value, and isn't Priscilla ignorant about those proposition, really? Whatever the answer, one might object that there is a slight alteration to the Gilbert and Priscilla case that might pose a new problem. Consider:

\begin{description}
    \item \emph{Hesitant Howard} is too indecisive to belief most anything, he withholds judgment on every single proposition he is not perfectly sure of. And he is almost never sure of anything! As a result, most (if not all) of his doxastic attitudes are suspensions of judgment.
\end{description}

Of course, Howard is not in an epistemically notably valuable position. And his situation certainly is not better than that of Priscilla. But doesn't the accuracy account produce the verdict that Howard's situation is more valuable, given that he accumulates value for each suspension of judgment, while Priscilla doesn't? The response is analogous to the previous case. If Priscilla is ignorant or without doxastic attitude towards most of the propositions that Howard withholds judgment on, the accuracy account might produce the verdict that Howard accumulates more epistemic value than Priscilla does. If we grant, however, that Priscilla herself suspends judgment on these propositions, then her overall positive track-record will give her a slight to large edge over Howard and therefore make the account's verdict agree with intuition.

If we stipulate, though, that Priscilla has doxastic attitudes towards a given limited set of propositions, but which are mostly accurate, whereas Howard's set of proposition he withholds judgment on is significantly larger, the accuracy account would value Howard's situation as the more epistemically valuable. I argue that these cases in which the set of propositions in question differ in such a significant way may yield less clear intuitions about the epistemic value. Intuitively, there might be something of value in the mere range of propositions that Howard is familiar with, and even though he is not knowledgeable about most of them, just having this many doxastic attitudes may count some way towards epistemic value.

I would propose to set aside such edge cases and predominantly compare epistemic situations on opinionated, fixed or comparably-sized sets of propositions–doxastic attitudes are, after all, what we want to compare the value of.

\subsubsection{Summary}

In this section, I present simple and straightforward forms of spelling out what a veritist account of epistemic value might look like, and compare their verdicts against intuitions about epistemic value in ready-made and non-obscure cases. I then evaluate which, if any, of the presented accounts seem plausible enough on this first challenge to then be incorporated into a theory of epistemic normativity. It becomes apparent that getting clear on what it is, exactly, that veritism states, is not as straightforward as one might think. Simple veritism can't accommodate the value of disbelieving a false proposition. Dualistic veritism has problems explaining intuitions about the value of suspending judgment. Accuracy veritism is not able to properly explain intuitions about comparing the value of wildly diverging sets of beliefs, however, that problem might not be of so much import after all.

To make headway, I propose to accept the account of accuracy veritism for the time being as a predominantly plausible one. To avoid complications, I still talk of veritism as employing true belief as the only final epistemic value with the notion of accuracy in mind.


\subsection{Veritism and Other Axiological Accounts}

Veritism proposes that true belief is the only final epistemic value. This would not be saying much if there were not different accounts of value that try to explain epistemic normativity from a different starting point. In this section, I outline some of them and give an overview of the different accounts in use in contemporary normative epistemology.

\subsubsection{ Truth Value Monism }
I stated in section \ref{sec:varieties} that \textcite[191]{Zagzebski2004-ZAGEVM-2} has a different name for a very similar position to veritism: epistemic value monism. I label this position \emph{truth value monism}, however, since epistemic value monism seems to be somewhat of a misnomer for the position that Zagzebski sketches: “Any epistemic value other than the truth of a belief derives from the good of truth.” Although this is a case of epistemic value monism, that is, a position that stipulates a single final epistemic value, other positions may also stipulate a single final epistemic value, most naturally knowledge, and thereby classify as epistemic value monism. So the term is ambiguous, hence truth value monism.
Is truth value monism the same as veritism? In section \ref{sec:varieties} is presented different renditions of what veritism might amount to, one of these simply stating truth as the final epistemic value. The other versions may still classify as truth value monism, depending on how tight you want to draw the concept, but i think the label veritism with the various addenda such as \emph{dualistic} much more appropriate as working name, so that is what I am going to use.

\subsubsection { Virtues of Veritism }
Terminology out of the way, Let's see what reasons there are to endorse veritism as a theory.

First, it is a most natural notion. It is not controversial at all to say that truth is valuable one way or the other, or that having a true belief about some matter is better than having a false belief about that matter. This much is conceded by pluralist notions as well a other epistemic value monism, such as a knowledge-first account, even if in the latter case true belief would be valuable only derivatively. This much is even conceded by deontic conceptions of epistemic normativity, which grant that there is value to true belief, but that true belief derives it value trough norm-compliance. The concept is so natural in fact, that some even argued that belief constitutively aims at the truth \autocite{Shah2003-SHAHTG,Velleman2000-VELOTA}. To say that much is not necessary for a veritist position, however.

Secondly, veritism packs a bunch of explanatory power. To see this, consider a very successful theory in traditional epistemology: process reliabilism. This theory is, as Goldman point out, consequentialist in nature \autocite{Goldman2002-GOLTUO-2} employing veritism as axiological groundwork. I discuss extensively that this is not without problems in section (REF->). But even in formal epistemology, veritism has explanatory power and can help vindicate Bayesianism, or so at least \textcite{Joyce2009-JOYAAC} proposes. Trying to vindicate probabilistic coherence has long mostly been a project taking its argumentative force from pragmatic considerations such as intuitions about pragmatic betting behavior. Setting it on alethic foundations, as Joyce puts it, is to present a purely epistemic argument for probabilistic coherence. He then goes on to assume Accuracy of degrees of belief as the epistemic value, which is quite strikingly a veritist variant.

Thirdly, it may be argued that parsimony is a veritable virtue of explanatory theories \footnote{In fact, \textcite[342]{Ahlstrom-Vij2013} posit some key desiderata for a satisficing epistemic axiology, chiefly among them a requirement for parsimony. They argue for it on the grounds that “one should prefer ontologies with fewer rather than more existential commitments, \emph{ceteris paribus}” \autocite{Ahlstrom-Vij2013}. This means that when comparing two theories with similar attractive features then, other things being equal, one ought to choose the theory with the fewer different entities assumed existing as possible. This intuitive principle in philosophy and the sciences was already proposed by Aristotle, Kant, Newton, Einstein, and Lewis among many \autocite[3]{sep-simplicity} and has its most famous philosophical realization in Occam's Razor, which can be stated as “other things being equal, if $T_1$ is more ontologically parsimonious than $T_2$ then it is rational to prefer $T_1$ to $T_2$” \autocite[7]{sep-simplicity}. Arguments for it range from just stating that parsimony is a primitive value over analyzing the quality of actual empirical theories vis-à-vis parsimony to probabilistic arguments showing that rational agents assign distinctive prior probabilities to simpler laws \autocite[11-26]{sep-simplicity}. For the purposes of this paper I take it to be sufficient to assume this quite plausible principle without going any deeper into this discussion.},
and stating just one single epistemic value to explain all of epistemic normativity is of course quite parsimonious.
Forthly, stating a single epistemic value is one way of avoiding the incommensurability problem \autocite[16]{sep-value-theory}, which I discuss in (->REF).

So it seems like veritism has a lot going for it. But of course, there are alternative epistemic axiologies. Next, I sketch an overview of the relevant competing positions.

\subsubsection{ Veritism vs. Value Pluralism [WiP]}

One obvious alternative is to not stipulate truth as the final epistemic value, but to also allow other plausible concepts to have non-derivative epistemic value. Chiefly among these are justification and knowledge. The advantages of this account are immediate: The value problem does not arise, which purports the lack of explanatory power of veritism when faced with the intuitive claim that knowledge is more epistemically valuable than true belief. I discuss this problem and its import for veritism in section REF. One version of epistemic value pluralism is proposed by \textcite[399]{Matheson2011-MATHTB-2}. He suggests that knowledge, justified belief and true belief are of fundamental epistemic value. He argues for it on grounds of alterations of the value problem for veritism that challenge other pluralist versions, and comes to the conclusion that his proposal is the only viable alternative.
As with all theories that posit the existence of more entities [This is not necessarily true, only for realists about values. Rephrase to be neutral about theories but still make the point], pluralism about values also has to account for the question: why? The mere reason that it explains the value problem in a satisfying way would only be sufficient reason if the value problem would decisively rule out veritism as a viable alternative. Assuming for the moment that this is not so, pluralists have the additional explanatory burden to deliver a story about why it is that there are many values, a burden that veritism escapes. Given that both theories are of equal merit, parsimony clearly decides in favor of veritism.

\subsubsection{ Other-Value-Monisms (Knowledge, ...) [WiP]}
    \todo[inline]{spell out}
    Veritist propose truth as the only epistemic goal. But why is that so? Woulnd't other candidates be a lot more plausible? After all, this is epistemology, and not alethology, or something like that.

\subsubsection{ TVM vs. Non-Value-based accounts of normativity [WiP]}
    \todo[inline]{spell out.}

    Deontological account of justification \todo{MEP this is NOT what this section is about} has the problem of explaining intuitions of the form you should believe it because its most likely to be true!

    Deontological accounts also need to provide an explanation for the problems posed by a false doxastic voluntarism and ought-implies-can. Veritism gets evaluative notions for free! Docastic voluntarism still is a problem for deontic requirements, though. Strategies: Flat-out deny deontic (regulative) requirements, or adopt one of the deontological accounts position and distinguish between different forms of doxastic voluntarism or deny ought-implies-can in one of its different forms. (Perhaps) we'll see later what these strategies exactly amount to .

    Deontological acounts want to account for the idea:

    [T]he theory of ‘ideal utilitarianism’ ... seems to simplify unduly our relations to our fellows. It says, in e ect, that the only morally signi cant relation in which my neighbors stand to me is that of being possible bene ciaries by my actions.  ey do stand in this relation to me, and this relation is morally signi cant. But they may also stand to me in the relation of a promisee to a promiser, of friend to friend, of fellow countryman to fellow countryman, and the like; and each of these relations is the foundation of a prima facie duty. \autocite[19]{Ross1930}

\subsection{The Significance Problem}

\todo[inline]{Perhaps restructue this to encorporate a straight argument and the argue against it}

\begin{enumerate}
    \item[P1] We have clear intuitions about differing epistemic value of different true beliefs.
    \item[P2] Veritism cannot explain differing epistemic value of different true beliefs.
    \item[P3] Any theory about $X$ that cannot explain the content of clear intuitions about $X$ is not a plausible theory.
    \item[C] Hence, veritism is not a plausible theory.
\end{enumerate}

In this section I discuss that there might be a problem posed to value-based accounts of epistemic normativity by the clear intuition that some true beliefs seem intuitively more epistemically valuable than others.

Let's compare these two sentences:
\begin{enumerate}
    \item The universe is expanding at an accelerating rate.
    \item The number of people ever to have visited the David Hume memorial up until now is odd. \footnote{Examples from \textcite{Ahlstrom-Vij2013}}
\end{enumerate}

Intuitively, truly believing (1) seems a lot more epistemically valuable than believing (2). \textcite{sep-knowledge-value} put it like this:

\begin{quote}
    Moreover, some true beliefs are beliefs in trivial matters, and in these cases it isn't at all clear why we should value such beliefs at all. Imagine someone who, for no good reason, concerns herself with measuring each grain of sand on a beach, or someone who, even while being unable to operate a telephone, concerns herself with remembering every entry in a foreign phonebook. Such a person would thereby gain lots of true beliefs but, crucially, one would regard such truth-gaining activity as rather pointless. After all, these true beliefs do not seem to serve any valuable purpose, and so do not appear to have any instrumental value (or, at the very least, what instrumental value these beliefs have is vanishingly small).
\end{quote}

Why is it that some beliefs seem more epistemically valuable than others, and how could veritism account for these intuitions?

For most people, believing a proposition is valuable only if it is of interest, be it for day-to-day life or for other purposes. Hence, concerning oneself for no good reason remembering something trivial like each entry in a phone book seems like a utterly point- and valueless undertaking, even though this add lots of true beliefs to ones overall stock. A distinction should be made here between belief-relevant interests of a person, that a somewhat subjective and objective belief-relevant interest, what i call the significance of a belief.

\subsubsection{Subjective Interests vs. Objective Significance}
A belief could be more or less relevant to one's own purposes. In that sense, my own interest determines what is significant to me. Whether the number of people ever to have visited the David Hume memorial up until now is even or odd may not be of interest to most people, but if I have a bet running on the outcome of (2) I am very interested in the result, as \textcite[333]{Ahlstrom-Vij2013} points out. But in this case, i wouldn't value this belief for its own sake, but to promote some other value besides its truth, valuing the truth of the belief only instrumentally. That is not an epistemic notion, however. And we are looking for some kind of epistemic significance. So this type of significance cannot explain the intuition in purely epistemic terms. A different approach is to grant genuine purely epistemic interests to people, that is, someone might be interested really just in the truth of a proposition. If we were to include this into the notion of epistemic value, the epistemic value of a belief would differ from person to person, that is, the epistemic value is relative to a person's interests. This might not necessarily be a bad thing on the conceptual level. Consider: The ascription of value to a belief itself is subject-dependent since there are no beliefs with no one there holding those beliefs. But wouldn't we want to say that a person believing (2) does so in an epistemically worthless manner, even if there are genuine epistemic interests involved? Or granting some epistemic value, if small and obscure however, wouldn't we want to say that to believe (1) would still be more epistemically  valuable than (2)? If this is true, then this kind of subject-relative notion of epistemically relevant significance can't explain our intuitions about it. But still, sometimes we do evaluate a single true belief relative to the epistemic situation of the believer. For example, there seems to be a difference in value between a mere random true belief and a belief that has been acquired through meticulous research. I grant this quite important point and discuss it in section \ref{sec:valueproblem}. However, this does not pertain to the \emph{significance} of the belief in question.

Perhaps some kind of majority-rule could deliver the desired verdict. It could be the case that what is epistemically significant is what most people deem interesting. I don't think this approach worthwhile to pursue, since it would be very strange building a notion epistemic normativity on a seemingly contingent empirical fact about majorities of people, if that is a fact, anyway.

But perhaps what is really meant here is what \emph{should} be of interest to the subject. So it may be the case that one is very interested in the outcome of the bet on (2), but really, what one epistemically should be interested in, is (1). This does not help the veritist one bit, though, as it is her aim to explain epistemic normativity. Presupposing contingent normative requirements as what actually determines epistemic value, from which epistemic normativity is then argued for, would just be circular reasoning. It seems, then, that subjective interests regarding epistemic value either do not explain our intuitions or do no work for a normative theory.

What about some form of objective interest or significance? It is not quite clear what that would actually mean. We might assume some kind of natural curiosity in human beings that entails some form or other of a desire for relevant truths that could be cashed out in purely epistemic terms \autocite[333]{Ahlstrom-Vij2013}. So somehow the relevance of the truth in question is then something ingrained in human nature, an we as believers can't help to value some truths more than others. This idea is somewhat similar to the argument I gave in favor of grounding epistemic normativity in section \ref{sec:groundingnormativity}. What did the argumentative work there was the conjunct of inescapability of the role as believers and the normative statement that only true belief is correct. Similarly, we assume some kind of inescapability in our role as believers, and an additional natural interest in relevant truths. Unlike before, though, this move is not available at this point: there is no normative statement of the form “only relevant beliefs are correct” or the like, at least not in our veritist theory so far. To assume something like that has two consequences: on the one hand, it would mean to give up explaining where the significance comes from and instead just stipulate significance or interest on the grounds of accordance with intuition. On the other hand, this seems like a departure from value monism: Apart from truth, there is an additional fundamental value, significance. Both options don't seem perfectly viable for the veritist.

\textcite[61]{Goldman2002-GOLTUO-2} opts for just stipulating significance in the form of interest: „Let us just say that the core epistemic value is a high degree of truth-possession on topics of interest“. He agrees that this make the core value compound or complex in some manner, but then states that this does not challenge the thematic unity of his virtue theory. The latter part is sound, given that Goldman is in search of some „weak thematic unity“ of values that underpins his virtue-based account of epistemic normativity\footnote{It may seem surprising that I talk of virtue epistemology in this context since Goldman is a prominent externalist and process reliabilist. The theory he proposes is a complex hybrid which I don't discuss here.}. But, alas, for the committed veritist this is not a serious option, in my opinion. The only ontological commitment that the veritist is prepared to make in the realm of epistemic values is true, or accurate, belief, and not some additional qualification like not properly motivated interest.

\textcite[332]{Ahlstrom-Vij2013} first argue that intuitions about significance are trivially compatible with veritism, we just need to emphasize the point that not all true beliefs but only true beliefs are epistemically valuable, incorporating the possibility that some true beliefs are of no value at all. This commits to holding that believing (2) is not epistemically valuable, whereas believing (1) is, and thus accounts for the difference in intuition. This leaves open the question of what it is that distinguishes worthless types of belief from valuable ones. To address this, \textcite[334]{Ahlstrom-Vij2013} suggest that “significance measures the degree of epistemic value as a function of the extent to which the relevant true beliefs speak to inquiries that we deem worthwhile, either on practical grounds or on account of intellectual curiosity.” They don't stipulate an additional epistemic value, but instead give a belief's significance the status of a property of belief, that somehow connects true beliefs with their assigned epistemic value. They make it explicit that this is not an existential commitment as this is not a property in the ontological sense but instead the measure by which we, as human beings, value true beliefs. So it is in virtue of significance that values come in degrees. This would be quite convincing, in my eyes, were it not the practical grounds that they grant to significance. As outlined above, this does not seem to add to epistemic value at all. Additionally, it still does not \emph{explain} significance other than through being a primitive property of human nature or of ideal inquiry or give any kind of criterion as to how this might be determined. Another issue is the compatibility with degree of beliefs. Granted, this idea is not designed for and probably has to be augmented to fit the model of degree of beliefs. As in this model, accuracy as a measure of how much the degree of beliefs agrees with the actual truth value already provides numbers for the value, it is unclear as to how the significance play into this.

\subsubsection{Proposal for a Veritistic Response}
Both accounts are somewhat useful. However, below I argue that considerations of practical value do not play a role in epistemic evaluation. Furthermore, maybe there is a account of truth-directed value that takes into account the significance of belief as an objective notion. I address how this might avoid diluting veritism to a pluralistic position.

What is the exact distinction between values in general and merely epistemic values? Epistemic values pertain to epistemic concepts like knowledge, justification and truth. But is that a division of values that is actually consistent with actual language or just some conventional means to divide and conquer the problem? Speaking from an evolutionary standpoint, it seems likely that we as humans are fitted with a natural goal of believing what maximizes our chances of survival. In most (if not all) cases, this goal coincides with believing what is true. But conceptually, these two are not the same. There may be cases where it would be best from a survival perspective, to believe that there is a predator in the surroundings, even when there is not, and so to believe falsely. So to purport the actual, final value of beliefs as being true seems, if nor fat-fetched, at least somewhat \emph{artificial}. It is a very useful distinction, however, as the epistemologist can free her reasoning from any practical considerations that would otherwise be still on the table. However, if epistemic value \emph{is} artificial it might also not be what guides our intuitions about thought experiments. These intuitions may not share this artificial division of goals and instead subsume epistemic evaluation under some, probably more practical, goals. [I propose that these or some other not merely epistemic intuitions are what guides our evaluation in some cases. Sentence (2) maybe considered of no or low value mainly because it is of no general practical value.] So in some cases, our intuitions about epistemic value could actually be intuitions about practical value. This has the consequence of accepting that there are cases in which it seems obvious which of the beliefs is of more epistemic value, but the theory giving contrary results. One might object that these alleged practical considerations are actually epistemic considerations in which truly believing something now leads to more true beliefs in the future. This is a complicated story, which I address in section \ref{sec:justification}.

But from a different perspective, there might actually be an objective difference in significance in the purely epistemic sense. So it is not that I would give up on \emph{all} intuitions about different epistemic values of beliefs. And this may actually be the case for lots of cases, not just some. This notion of significance starts with observing what we actually mean when we compare values of beliefs. A straightforward approach would count each true belief. The higher the counter, the higher the value! This is pretty naive and of course fails to do justice to the intuitions on (1) and (2). But it fails even earlier. Considering the meaning of (2), what is expressed is actually a number of propositions. That there exists a memorial, that memorial is dedicated to David Hume, that people visited it, that the number of people that visited it is odd. Now believing all that amounts to four beliefs, so it would be four times as valuable. Now, I stated earlier, that believing $P$ is not necessarily equivalent to believing some proposition equivalent to $P$, but the \emph{value} is. So just in grasping what the actual contents of the belief are, the belief's value is raised. This might point to the idea that the value of a belief is reduced to the accumulated value of the atomic propositions that it contains. Hence, is it necessary to provide a complete logical analysis to ascertain the value of a belief. That does not seem like a desirable position to take. However, I merely want to offer a view of the direction where this could lead, and what its advantages would be\footnote{The only papers so far that I found proposing a similar approach is \textcite{Treanor2014-TRETTA} and \textcite{Pritchard2014} picking up Treanor's idea. He proposes that believing something like (1) is just believing \emph{more truths} and hence more valuable. He claims that for the objection to rebut teleological accounts of epistemic normativity it would have to present two sentences that contain a very similar amount of truth, and still generate the needed difference in intuition about their epistemic value. See footnote \ref{foot:noblackholes} for a possible counterexample. However, he also states that this method of quantifying the significance of a belief is somewhat elusive. It is easy to see that the vague nature of language makes this approach difficult.}.
Compare the sentences: “All ravens are black“ and „this raven is black“. Which is more valuable to belief truly? Apparently, the general statement is. Since it says a lot more about [things, the world] believing it  is in value equivalent to a lot more true beliefs than the singular sentence is\footnote{\label{foot:noblackholes}This of curse leaves much to be desired. First, it is not uncontroversial what the actual contents of a belief are. Second, consider negative Statement like “there are no unicorns“. What is the actual truthmaker in the world? Then consider: “There are no black holes”. Doesn't the second seem a lot more epistemically valuable? How could that be explained? A proper account would have to deal with these kinds of issues as well, I think.}.
This would also account nicely for why we value (1) more than (2): To grasp what it means that the universe is expanding at an accelerating rate is to have a working knowledge–or at least idea–of what the universe consists of, that things and concepts like stars, planets, velocity and gravity have something to do with it and so forth. All these things get something attributed to it, and that possibly indefinite conjunction of propositions is equivalent to just stating (1).

Or consider this prominent objection: If true belief is the only valuable thing, then can't I just raise the value of my epistemic situation by taking all my true beliefs, and then belief all permutations of conjunctions of them, and thus increase the number of my true beliefs manifold? Well, on this account, since each of these conjunctions derives their value in full only from the value of their true conjuncts, they don't provide any additional value.

What with someone who doesn't understand\footnote{Unterstanding a belief in this sense is just grasping what the contents of a belief are, not in any more rich or complex sense.} the meaning of a sentence? Say I just heard (1), and now truly believe it, but do not understand what that actually means, entails, or coheres or doesn't cohere with. Would we still be inclined to say that this person is in an epistemically valuable position? I don't think so. 
One could object to this that the significance of a belief is now again relative to a subject's own peculiarities, this time the epistemic value is dependent not on the subject's personal interests but on the subject's level of understanding of the target belief. This looks like it would lead to a subjective conception of epistemic value, and as argued above I don't think that desirable. All is not lost, however. Consider the description of accuracy veritism in section \ref{sec:varieties}. Here, the amount of value a subject accumulates per belief depends on the accuracy of that belief. But is the concept epistemic value thus subject-relative? It is not, I propose, if we take what it means to completely grasp a belief to be constituted in full by what the content of the belief \emph{actually} is fully realized by. So a claim about the significance of a belief would be a claim about the the richness of its content, about how much it has to say. The actual degree to which this significance is appreciated by the subject then depends on its level of understanding of the content of the belief. The understanding of the belief is then a means to more true belief. This explains why understanding is not valuable for its own sake in this sense, instead understanding is only instrumentally valuable to arrive at more true beliefs. Given a full understanding of the contents of a true belief, the subject fully realizes the beliefs value. In other words, the significance of a true belief is dependent on the beliefs content and influences its \emph{potential} epistemic value. The degree to which a subject understands the belief influences the \emph{actual} epistemic value of a belief. So this does not pose a threat to value monism by incorporating understanding as a fundamental epistemic value. It is, in fact, still compatible, since understanding in this sense only has instrumental epistemic value.

I argue that in problematic cases concerning the intuition about values of beliefs with different significances, the person stating her intuition has enough grasp of the given propositions to notice a profound difference in significance to come the conclusion that one belief is more epistemically valuable than the other. So we don't need to assume perfect understanding to ascribe the ability to make judgments about it but just enough understanding to generate a clear sense for the comparative difference.

Of course, this approach in the current form is naive at best. How that should exactly work is beyond unclear at this point. But I think this approach lacks some flaws of the other accounts presented and is sound enough to explain the problematic intuitions about (1) and (2).

\todo[inline]{perhaps note that dretske transfered shannons theorem on INFORMATION into epistemology and point that out for a possible approach of sketching a more exact notion.}

\subsubsection{Summary}
In this section, i presented the objection from significance to veritism. It states that some true belief are intuitively more epistemically valuable than others, where any truth value monism should come to the verdict that they are equivalent in value. I take a look at some answers to this from the literature and then propose that, (i) some of these intuitions get practical value mixed up with epistemic value and (ii) suggest an account of epistemic significance of a belief that explains supposed purely epistemic intuitions while still doing justice to the requirements of truth value monism.

\section{Veritistic Accounts of Epistemic Norms}

Is there a single deontic norm of belief that is both plausible non-contradictory? With just a single norm, it is difficult to capture the range of intuition about norms of belief that seem to exist. With more than one norm, this can be easier, but then there could be cases where the norms contradict. Which norm overrules the other, then? Can such an ordering be sensibly made at all?

\subsection{Can There Be Norms of Belief at All?}

% NEUER ANSATZ

% NOT deny ought implies can. instead grant alson the no-deontics-in-beliefs argument.
% Read: Peels, Rik: Against doxastic compatiblism, does doxastic responbisbility entail ab..

% -> No deontic norms of belief in strong sense (synchronic)

% -> In a weak sense: CHose best route to belief formation (literatur?) indirect voluntarily control? indirect influence

% -> 

Before I investigate whether there are actual plausible norms that can be explained by appeal to veritism, I have to discuss whether it is actually possible for a deontic belief norm to apply. The reasoning why this might be threatened is straightforward: According to some popular opinion, it is not the case that we can voluntarily chose to believe a given proposition. Believing is not analogous to actions in this respect. Suppose further the also widely appreciated principle that ought implies can, meaning that if I ought to $\Phi$, it has to be the case that I can $\Phi$. Then, if applied to belief in the sense that I can believe something if and only if I have voluntary control over it, it is not the case that I can believe something, and therefore it is not the case that I ought to believe something. So deontic norms of belief simply don't apply. That would be a blow to the deontic normative aspiration of veritism, and moreover, lead to a deontic-free epistemic realm. There would simply be no deontic norms governing reasoning, at least not in this strong sense. So claims like 'You are misreading the evidence. You should  believe otherwise.' are at most evaluative claims–I don't express a requirement for you to change your belief, I merely state that I think your believe epistemically worse than some other\footnote{This seems to be the position of, for example, \textcite[241]{grundmann2008}}. This does not rule out the normative nature of, say, justification, as it might be that it still is an evaluative category. 

\subsubsection{The Alston Challenge}

The argument looks something like this\footnote{Loosely following and simplifying \textcite{Alston1988-ALSTDC}. \textcite{Feldman2000-FELTEO-2} proposes a similar argument.}:

\begin{enumerate}
    \item[P1] I ought to believe $P$ only if I have voluntary control over whether to believe $P$.
    \item[P2] I don't have voluntary control over whether to believe $P$.
    \item[C] Therefore, it is not the case that I ought to believe $P$.
\end{enumerate}

P1 is just ought implies can on a popular reading of ‘can’ applied to the case of belief, and P3 is just the claim that doxastic voluntarism is false. The argument is certainly valid, but how sound is it?

One could easily write a whole dissertation on just this question and still not do justice to range of it, so for the present purposes I will just briefly sketch possible strategies to object to these argument and then develop what could be a response from the veritist. I conclude that there is a sense in which one can reasonably assume deontic norms to apply to belief formation.

P2 is very widely agreed upon. Doesn't it seem obvious that beliefs are not actions, and while I can chose to or refrain from raising my arm, it is much less so with what I believe? \textcite[91]{Alston1989} famously argued\footnote{I adopted this quote from \textcite{Steup2000-STEDVA}.}:

\begin{quote}
    When I see a car coming down the street, I am not capable of believing or disbelieving this at will; ... when I look out my window and see rain falling, water dripping off the leaves of trees ... I form the belief that rain is falling willy-nilly. There is no way I can inhibit this belief.
\end{quote}

While denying P2 is an interesting route to take, it would involve arguing against \textcite{Alston1988-ALSTDC} and stating that there are in fact forms of relevant voluntarily control that we can exercise over our belief formation, or, as \textcite{Steup2000-STEDVA} does, it would lead us deep into a discussion about actions, intentions, determinism, and free will, and I don't really want to open that box in this paper. 

What other options are there? P1 is an instance of the well-established principle ought implies can. It is based on the interpretation of 'can $\phi$' as having voluntary control over $\phi$. This, however, is an additional assumption that should be argued for. It might be the case that this is not as straightforward as it appears. I present this line of response in section \ref{sec:oughtdoesnot}.

Another option is to buy into the argument and try to arrange oneself with the claim that there are no deontic norms for belief formation in this very strong sense. However, there might be other forms of influencing how I form my beliefs, and these may be under voluntary control and hence not defeated by this argument. So deontic norms about beliefs wouldn't actually state which beliefs to hold and which not, but instead concern indirect forms of influencing belief formation over some period of time. This is what Alston calls “indirect voluntary influence“. For example, the “epistemic duty to seek more evidence“ \textcite{Hall1998-HALTED-2} when in lack of decisive evidence is such a proposal. Tell a bit more with \textcite{Peels2014-PEEADC}

perhaps see \textcite{Kruse2001}

\textcite{Feldman2001-FELVBA} denies P2 by denying ought implies can. He states that there are kinds of \emph{ought statements} that do not entail voluntary control. An example of what he calls \emph{contractual obligations} is “You can have an obligation to pay your mortgage even if you don't have the money to do so.“ \autocite[674]{Feldman2000-FELTEO-2}. It seems obvious, though, that there are no contracts involved when I ought to believe something. \emph{Paradigm Obligations} describe what is prescriptive under normal, regular circumstances, typically noted when not in those circumstances. For example, the physician says that “you ought to be walking again now, but you still can't“ \autocite[675]{Feldman2000-FELTEO-2} expresses should normally be the case. However, Feldman notes there are no \emph{real} obligations here at all. Epistemic obligations can't be of this sort, then. 

\subsubsection{Response: Feldman's Role Oughts}

Finally, he describes what he calls \emph{role oughts}. This is the type of ought that, according to \textcite[676]{Feldman2001-FELVBA}, might be at play in epistemic obligations. “Teachers ought to explain things clearly. Parents ought to take care of their kids. Cyclists ought to move in various ways. Incompetent teachers, incapable parents, and untrained cyclists may be unable to do what they ought to do.” These ought are based on good performance relative to a standard imposed by the role involved. This standard for believe is to believe truly. And since anyone is in the role of a believer, that ought applies to any human being. So even when there is absolutely no voluntary control involved, Feldman states that this ought still applies. If a teacher is not able to communicate the teaching material clearly, it is still the case that she ought to do so. Why? Because that's what a good teacher would do. And since she is in the business of teaching, that's the gold standard.

How plausible is this account to allow for deontic norms of belief? A lot hangs on the intuitions about Feldman example cases.
\textcite[9]{Cote-BouchardForthcoming-CTBCTA} presents some counterexamples as a reply to a different question, but the critique applies here as well. First, being in the role of something does not imply normativity. Consider someone in the role of a torturer. \emph{Qua torturer} he ought to make his victims suffer. But it would be rather cynical to ascribe good reasons to do so to the torturer. Is it really the case that a bad teacher ought to do better, even tough she is incompetent and is in no position to better herself? The phrasing I used here is somewhat contentious, for I described the teacher as bad. This is already a normative statement. But it is different in a vital way from presupposing obligations, of course, as it is merely evaluative. A teacher failing to live up to any of the job's standards is a bad teacher, but \emph{ought} she do better? According to Feldman, that it exactly what comes with the role of believer. It is a normative requirement to do it right.

But consider: A very engaged teacher suddenly loses her voice in a constructed and implausible manner, interrupting her ability to communicate clearly. Wouldn't it be rather cynical to say she now fails to fulfill her obligations as a teacher? I don't think it would, since what she ought to do now is find other ways to communicate. One could reject this on the grounds that is still something she \emph{can} do, so it is in compliance with ought implies can. Consider that she suddenly becomes utterly incompetent through some mysterious circumstances and is not able to communicate clearly anymore at all. Is it still that she ought to do so? I think that foremost, she perhaps should not be a teacher anymore, but given that she is, yes, that obligation still applies. This intuition might be difficult to tackle, but I think that Feldman has a good point here. This does not, after all, constitute an all-things-considered ought, and there might be other reasons that outweigh the normativity of being a teacher. So all things considered, the situation may be different, but still, as a teacher, the obligation applies.

\textcite[686]{Peels2014-PEEADC} argues that there is another crucial difference between the cases. The teacher entered \emph{voluntarily} into her role. So the obligations associated imply some kind of voluntary control. This is not the case with a believer, I do not have voluntary control over my beliefs. So this analogy fails \autocite[687]{Peels2014-PEEADC}.

Consider another case: The unfortunate older sister suddenly has to care for her little brother. Say, anyone else who could be said to have this \emph{responsibility} has vanished, which leaves her as the only one to care for her brother. But sadly, she is in no position to fulfill this task, be it financial or personal reasons. She just cannot do it, and can't voluntarily chose to do it, either. She did not enter voluntarily into this situation, and cannot evade the role, either. So the big-sister-role is in these regards similar to the role of the believer. But still, we would say that she has a moral \emph{duty} or \emph{responsibility} to care for her little brother. And with that, she \emph{ought} to care for him, regardless of the circumstances. This maybe shows that there are such role oughts, and these role oughts aren't as distinguished from the role of a believer as argued.

But, granted, the intuition in this case maybe clouded from the emotional force of it. It wouldn't say it is decisive that we still would hold her \emph{responsible} in this extreme case. Maybe she still has the \emph{obligation}, but has a moral \emph{excuse} for it. But to discuss this more complex and differentiating terminology will have to stay out of this paper, I'm afraid.

So it is unclear whether Feldman really succeeds in denying \emph{ought implies can} for the role of believer. I propose, however, that this line of response is not convincing enough to go on and presuppose it for the normative purposes of veritism. Moreover, outright denying ought implies can to apply to any belief is a very bold claim, as it would mean to abandon a very plausible principle and have other consequences for reasoning about normativity in the case of belief, as I discuss in section \ref{sec:normsofbelief}.

\subsubsection{Response: Ought Does Not Imply Voluntary Control}\label{sec:oughtdoesnot}

There is a related but distinct line of response proposed by \textcite{Chuard2009-CHUENW}. They challenge P1 in that ‘I ought to believe $P$ only if I have voluntary control over whether to believe $P$’ may not actually be an accurate interpretation of \emph{ought implies can} (OC). They deny that this interpretation is correct, and claim that although although OC is a valid principle, ‘can’ is not to be interpreted as ‘has voluntary control’
\footnote{\textcite[601]{Chuard2009-CHUENW} emphasize that they do not claim the falsity of the premise, but instead that it is unmotivated and without good reason. For the present purposes, this difference is not crucial, though.}.
Instead, on all interpretations they present, ought implies can still holds while allowing epistemic deontic norms for belief. Hence, this is a form of \emph{doxastic compatibilism} \autocite[682]{Peels2014-PEEADC}.

They do so by first analyzing the normal usage of ‘can’ and then show that, for each of these usages, it simply does not follow that ‘can’ means ‘has voluntary control’. they start with the interpretation of possibility, which includes logical, metaphysical, biological, and so on. So ‘S can $\phi$’ would be interpreted as ‘It is [logically, metaphysically, ...] possible for S to $\phi$’. This, naturally, doesn't really capture the ‘can’ expressed in the principle. If the only requirement were that there is some faint logical possibility to do what you ought to, OC would be completely toothless. It is used with much success in the moral realm, but restricting it in this way would hinder any usefulness.

 \textcite[616]{Chuard2009-CHUENW} claim there are two proposals of an interpretation of ‘can’ that come closest to its meaning in OC. First, there is the principle of alternate possibilities which states that ‘S can $\phi$’ means ‘it is possible for S to $\phi$ and possible not to $\phi$’ \autocite[615]{Chuard2009-CHUENW}, with ‘possible’ read as logically, metaphysically, and so on. The resulting OC-variant would then look like this:

 \begin{description}
    \item[OC$_{AP}$] S ought to $\phi$ only if it is [logically, metaphysically, biologically, psychologically, epistemically ...] possible for S to $\phi$ and possible not to $\phi$.
 \end{description}

But from this, it does not follow that
\begin{description}
    \item[OC$_{VC}$] S ought to $\phi$ only if S has voluntary control over whether to $\phi$.
 \end{description}

 To show this, it is enough to present a counterexample. They purport “Mathilda ought to be afraid of the crocodile in front of her”, since, you know, they are dangerous \autocite[616]{Chuard2009-CHUENW}. It is in any reading of ‘possible’ possible for Mathilda to be afraid of the crocodile in front of her, yet having the emotion of fear is not something she has voluntary control over. So $OC_{AP}$ holds, while $OC_{VC}$ does not.

 The second promising interpretation is ‘can’ as ‘ability’:

 \begin{description}
    \item[OC$_{A}$] S ought to $\phi$ only if S is able to $\phi$.
 \end{description}

 The kind of ability is determined by the kind of $\phi$ at question \autocite[617]{Chuard2009-CHUENW}. Now, they propose this example: “Judy ought to understand what Nicole is going through.”, where Nicole is clinically depressed. Suppose Judy has been suffering depression as well, so she is able to understand what Nicole is going through. The deontic claim seems plausible enough. However, it doesn't follow that Judy can voluntarily chose whether to understand Nicole. Understanding is a “success term: trying to understand isn’t the same thing as actually understanding.“ What do \textcite[617]{Chuard2009-CHUENW} mean by that? According to a dispositional definition of ability due to \textcite[848]{Fara2008-FARMAA}, “an agent has the ability to A in circumstances C if and only if she has the disposition to A when, in circumstances C, she tries to A“. So in order for July to understand Nicole, she needs to be disposed to understand her in the given circumstances and that she tries to. But it is of course still conceptually possible for her to fail to understand Nicole, in exactly those cases in which her disposition doesn't manifest. Thus there are cases in which Judy has the ability to, but does not have any voluntary control over whether she understands Nicole (since she may fail to do so). Hence, OC$_A$ does not entail OC$_{VC}$.

This line of reasoning suffices for \textcite[618]{Chuard2009-CHUENW} to show that no plausible interpretation of ‘can’ implies voluntary control. What they achieved, in my opinion, is to argue in a sufficiently convincing manner as to shift the burden of proof to anyone claiming that OC$_{VC}$ \emph{does} follow from ‘S can $\phi$’. There is still a possibility to defend the original argument by stating that it does not have to hold in the general case, but merely in the special case of belief formation. So it is only for beliefs that ‘S can believe that $P$’ means ‘S has voluntary control over whether to believe that $P$’.

So one might propose that it is only in this case that ‘S can believe that $P$’ means ‘S is able to believe that $P$’. On Fara's definition of ability, it is trivially true that S is able to believe, since, plausibly, forming a belief is not analogous to actions in that it is formed through an \emph{intention}, and an intention plausibly is a necessary condition for trying. It is then trivially true that S has the ability to believe since there simply are no circumstances in which she tries to believe something, and hence cannot fail to do so\footnote{Fara's definition could also be read as that trying is also a necessary condition for the ability, which would entail that there are no belief-abilities, given one does not form believes intentionally.}. Perhaps the definition could be altered to more plausibly incorporate cognitive abilities and their conditions. \textcite[618]{Chuard2009-CHUENW}, anyway, suggest that S has the ability to believe: “S must possess the various concepts necessary for understanding the proposition that p, and this might demand enough cognitive sophistication on her part. But the further ability to form an intention to believe p, and to carry out that intention, is not one S must have in order to be able to believe that p.” They say that all other interpretations of ‘can’ don't yield anything close to voluntary control, so this response fails to undermine their argument. I think again that here, this challenge is strong enough to shift the burden of proof. Given their argument, I'm quite convinced that deontic norms can apply in spite of lack of voluntary control over belief formation.

To sum up, while there certainly is a profound challenge from doxastic voluntarism to the applicability of norms of belief, there also are ways out of it. Hence it is time to take a look at what those norms might look like.

sources

http://www.iep.utm.edu/doxa-vol/

http://plato.stanford.edu/entries/ethics-belief/

\textcite{KruseForthcoming-KRUWDR}

\textcite{Steup1988-STETDC Steup2000-STEDVA}

\textcite{Alston1985-ALSCOE}

\textcite{Feldman2001-FELVBA}

\textcite[238-246]{grundmann2008}

\subsection{ Norms of Belief }\label{sec:normsofbelief}
What is the actual content of a plausible norm of belief, supposing that veritism is true? Is there more than one true norm of belief? What conditions should they satisfy to plausibly generate any normative power? These are the questions I want to address in this section.

The literature on this topic seems sheer endless. For every norm of belief proposed, there are counterexamples presented that render the norm implausible. I discuss some of this reasoning, but of course am far from actually respecting each and every objection. However, I hope to take into account at least the most common responses, and make a norm of belief somewhat plausible.

An important qualification is the following: I mostly speak of epistemic value and propose that it is only epistemic value that has any impact here. It may be that there are actual reasons to belief something that promote prudential or moral value. And there may be corresponding norms for it generating reasons to believe something. These reason, however, are not epistemic reasons; they do not promote epistemic value.

Let's be also clear on what I want to propose here. I want to say that there are actual epistemic deontic norms to the effect that given some circumstances, I should believe some proposition, and these norms are explained by appeal to true or accurate belief as the only epistemic value. I do not want to establish a kind of deontological conception of justification. In fact, I scarcely make reference to justification at all. How and if this might be developed into such an account, or if both are in some strong plausible sense compatible is left open as a task for another time.

The most straightforward way to postulate norm of belief is to take the epistemic value at face value. How do we get from a statement about value to a statement about obligations? That is what a consequentialist theory entails. The goals of our epistemic undertakings is defined by an ideal state of maximum value, and what we ought to ‘do’ epistemically is what “conduces toward the fulfilling of those goals“ \autocite[340]{Berker2013-BERETA-2}. And in the epistemic case, what conduces towards the goal is–in some way or other–believing truly. What might this goal look like? Most naturally it is a state in which all of my beliefs are true, simple as that. Now, this does not yet take into account the need to form any belief, if I don't believe anything, trivially all my beliefs are true, since there are none. This problem amongst others will be addressed later on when talking about norms of belief.

So a veritist norm of belief should somehow promote towards the veritist goal, in that if I follow the norm, I reach the goal, or if I partially follow the norm, I take a step towards reaching the goal.

TN For all propositions P and for every subject S: S ought to believe that P if and only if P is true

This first proposal obviously promotes towards the veritist goal. If I comply with the norm for each and ever proposition, I end up believing all and only true propositions. It partially promotes towards the goal in that for each true proposition that I believe, my epistemic state is closer to the one described by my epistemic goal. So on this point at least, the norm seems to agree with veritism.

-- other evaluations

When I ponder what I should believe, maybe the most important epistemological question, and consider some proposition $P$, what settles the question whether I should believe $P$ is just whether $P$ is true. That's it. This is so obvious and straightforward that it seems like a truism. Is this immediately normatively relevant, however? It is not that I now spontaneously decide, that, yes, $P$ is true, so I'm going to believe it. Of course, I don't have direct connection to the truth, some intuit sense that just spurts out the right answer. If I would have, we wouldn't need any epistemology to study our cognitive faculties, instead, we'd all be omniscient. The truth is neither \emph{accessible} to me nor \emph{guiding} in any sense. All I can do is ask myself what speaks in favor of $P$

This could be made a bit more precise by saying ought implies can and maybe it is an epistemic possibility

Indeed, many philosophers think so \cite cite cite. 

Desiderata for a Norm of belief:
- It should take into account the value of true belief, false belief, disbelief, and suspension of judgment.
- It should provide guidance
- It should be a norm that at least conceptually can be fulfilled
- It should be plausible with respect to its consequences (perhaps add these as additional desiderata)
- trivial truths should be accounted for

could be two strategies: say that the j-norm follows directly from the value via bridge principle in \textcite[21]{Mchugh2012-MCHTTN}.

OR first derive truth norm. then derive j-norm.

literature the truth norm:

\textcite{Engel2001}
I try to argue that there is a reasonable sense in which we can, and must, say that truth is a norm of belief, and that most of our epistemic norms are grounded in this one.

\textcite{David2001-DAVTAT-7}
Content of the truth \emph{goal}

\textcite[101]{Boghossian2008-BOGCAJ}
The rationality Norm is derived from the truth norm or stomething

\textcite{Mchugh2012-MCHTTN}
if belief is subject to a norm of truth, then that norm is evaluative rather than prescriptive in character.

\textcite{Greenberg2016-GREITN}
Argues for a prescriptive truth norm and against mchugh

\textcite{Whiting2013-WHITTA-3}
The truth norm is actually prescriptive, defending against objections

\textcite{Bykvist2007-BYKDTI}
arguieng against consitutivity: we will argue that it is not constitutive of belief that one ought to believe that p if and only if p is true.
also gives some additional objections

Objection from Moore-paradoxical propositions that 'p and i don't believe p': If true, one ought not to believe it, but standard oughts say to belief it

\textcite{Wedgwood2013-WEDTRT}
Interesting account with suspension etc. Response to Bykvist 2007. Semantics of ought, doxastic vs propositional justification. VERY complicated account, but ties in very nicely with the accuracy account given earlier in this paper

objective and subjective oughts
\textcite{Gibbard2005-GIBTAC} 
\textcite{Wedgwood2016-WEDOAS}
\textcite[Ch. 2]{Gibbons2013-GIBTNO}

Teleology
\textcite[Ch. 5]{Gibbons2013-GIBTNO}

If so, how plausible are they? Perhaps see Chisholm (quoted in \textcite{Goldman2002-GOLTUO-2})


Shit. It may actually turn out that, following the argument where it leads, that the truth norm is the real deal and reasons-justification-evidence-etc norms are only derivate in that sense. So in cases of conflict, the truth norm might take priority. Goldman says that reachable, accessible etc may even generate more normative force on the J-Norm, but i don't see it at the moment if one takes this role ought seriously. STRANGE result, man. In any case make this point very clear!!

OR MAYBE. What is at stake in the doxastic voluntarism thingy are only subjective norms. then the objective oughts could still apply, but only subjectively. Attention: i'd have to clean up with this terminology in the section on significancee! see 

basic idea: the truth norms applies objectively in a hypothetical state of perfect information. this is the best. belief it!
the truth norm applies subjectively in the sense that, from the subject's perspective, everything suggests that this is the best. belief it!
its still possible that the subjects perspective suggests that, but the subject doesn't see that or something.so i hope it is compatible with externalism

strategy could be:

first propose simple notions of truth norm and j-norm. say that it is because of guidance that truth norm has no effect etc. Then say that because of doxastic voluntarism solution earlier, the truth norm still applies!!!!! then show the solution with objective oughts and subjective oughts

once that is working, discuss how plausible such norms can be! all the gritty details. Make a list of desiderata that these norms should fulfil (perhaps do that first)

% \section{Veritistic Accounts of Epistemic Justification}
% \subsection{General Structure of a Veritistic Normative Framework (Berker)}
% \subsection{Compatibility with Interalism / Externalism}

% \textcite{doi:10.1080/00048402.2015.1038283}
% \subsection{Process Reliabilism}


% \subsection{The Value Problem}

% One fundamental premise of the argument is that knowledge is more valuable than truth. Or more specifically, that the situation in which S knows that $P$ has more value than a situation in which S merely truly beliefs that $p$. As argued by \textcite[35-38]{grundmann2008} in fregean spirit, the truth of a belief is a property of the belief provided by its propositional content. Since a belief can only have one particular propositional content, the truth value of the proposition it is expressing is its sole source of direct value. Knowledge, on the other hand, does not seem to be a mere property of belief. Knowing that $P$ entails something else than just to have some belief that has the property of \emph{qualifying as knowledge}, or however you want to call it. Knowledge is a complex composite with one of its components being true belief\footnote{Of course, this is a highly controversial claim, argued against by many, most prominently by \textcite{Williamson2000-WILKAI}}. On one of the more plausible conception of knowledge compatible with process reliabilism and hence veritism called safety, a necessary condition for S to know that $P$ is that in each of the nearby relevant worlds, If S believes that $P$, $P$ is true. Bracketing out what is meant by \emph{relevant} and \emph{nearby}, what strikes me as uncontroversial is that the truth of $P$ in other possible worlds can't be plausibly said to constitute properties of the belief that $P$. Instead, it is something else (whatever it is), that in conjunction with the true belief that $P$ instantiates knowledge. So when asking if knowledge is more valuable than true belief, instead of evaluating just the belief and its properties one needs to evaluate this

% against \textcite[291]{Stapleford2016}:

% The epistemic domain is defined by the concept of knowledge. The concept is complex and it corre- sponds not to a single fundamental value but to a configuration of values, one fundamental and one (or more) derivative. Epistemic value is best thought of as a hybrid comprising the fundamental value of truth and the derivative value of justification.24 If this is right, then the objection misses the mark. Just as an authenticated painting has no more aesthetic value than an unauthenticated one, a justified true belief has no more alethic value than an unjustified true belief. But it does have more epistemic value.

% This is incompatible with veritism according to with true belief is the only fundamental epistemic value. In Staplefords account, it would only be the only fundamental alethic value. The hybrid comprising truth and justification is of fundamental epistemic value, and truth has epistemic value only derivatively, just as justification only has epistmic and althic value derivatively.

% With \textcite{Stapleford2016}:
% To see how justification secures true belief against loss, con- sider what it’s like to believe P truly without justification. Suppose that you believe P because P was suggested to you when you were drunk and it seemed like fun at the time.29 This leaves your belief that P particularly susceptible to non-evidential influences and thus liable to random fluctuation. Say we try to coax you out of it: ‘Hey, we all believe not-P. Why don’t you? P sucks!’ If you didn’t have any good reason for believing P in the first place – if you just took it on a whim – then you have no good reason to stick with it now. You can drop P without a second thought. Whereas if you believe P with justification, you’ll think: ‘Why should I do that? P seems to be true. I am justified in keeping it.’ Justification thus gives your belief that P an added layer of protection against inad- vertent or arbitrary loss.

% This seems to be a reasonable approach to stability of belief.

% \subsubsection{ Value Problem aka Meno Problem aka Swamping Problem (Show differences). Explain what it is and why it hurts veritism }
% \subsubsection{ Discuss how it could be solved in combination with accounts of justification (for example reliabilism) see Goldman/Olsson}
% \subsubsection{ Own Discussion of a solution (perhaps from modal stability)}
% \subsubsection{ The solution? Is there a solution without reliabilism, i.e. generic?}
% \subsection{The Berker/Goldman debate on veritism}
% \todo[inline]{Maybe pull this part from here and reduce it to the argument pertaining to ONLY veritism and not reliabilism}
% \subsubsection{ Problems that \textcite{Berker2013-BERETA-2} raises}
% The case intuition seems uncontroversial, however, it has also been empirically corroborated \textcite{Andow2016}
% \subsubsection{ Answers that Goldman gives}
% \subsubsection{ This does not solve the problem for the veritist in general. Even without deontic definition, this problem arises(?)}
% \subsubsection{ Berker divides consequentialist theories in three parts: theory of final value, theory of overall value, and a deontic theory, as opposed to just }divide into axiological and deontological considerations. 
% \subsubsection{ Explain IN DETAIL what this means. }
% \subsubsection{ For all (most) consequentialist theories a problem arises with firth-style cases. }
% \subsubsection{ Explain IN DETAIL what these cases are, what and how the problem arises for the consequentialist theories}
% \subsubsection{ Explain shortly what the reliabilist (i.e.) Goldman replies.  }

% Firth-style problem: The unjustified false belief that causes other true beliefs should be valuable as a trade-off. Distinguish between what is of epistemic value here. If only the unjustified true belief: It has only final epistemic value if it is true itself. It isn't, so it hasn't. It attains intrumental epistemic value by causing the other beliefs. But that doesn't add anything to its final epistemic value. Which is relevant here? think about it hard. Is it the belief - and the other beliefs? than that set has fundamental value bc the other beliefs are true. which situation is more valuable? that in which I know lots of profound stuff. does that justify any old belief in the set? i dont think so.



\section{Conclusion}

% \nocite{*}
\begingroup
\setstretch{1}
\printbibliography
\endgroup
% \printbibliography

\end{document}