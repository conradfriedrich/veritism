%!TEX program = xelatex
\documentclass[12pt,numbers=noenddot]{scrartcl}
% \usepackage[ngerman]{babel}
\usepackage[a4paper,lmargin={3cm},rmargin={3.5cm}, tmargin={2.5cm},bmargin = {2.5cm}]{geometry}
\usepackage{amsmath}
\usepackage{setspace}
\onehalfspacing
% Entfernt Blocksatz!
\usepackage[document]{ragged2e}

\usepackage[authordate, ibidtracker=context]{biblatex-chicago}
\addbibresource{veritism.bib}
\usepackage{url}
\urlstyle{same}

% \usepackage[square]{natbib}
% \usepackage{jurabib}
% \usepackage {algorithm2e}

% Package, das die Benutzung von Old Standard erlaubt
\usepackage{fontspec}

% \setmainfont{OldStandard-Regular.otf}[
% Path = /usr/local/texlive/texmf-local/opentype/,
% BoldFont = OldStandard-Bold.otf,
% ItalicFont = OldStandard-Italic.otf]


% \setmainfont{Old Standard TT}
\setmainfont{Baskerville}
\newfontfamily\osfamily{Old Standard TT}
% \newfontfamily\osfamily{Helvetica}


\setkomafont{subject}{\small}
\addtokomafont{title}{\osfamily}

\addtokomafont{publishers}{\normalfont\small}
\addtokomafont{date}{\normalfont\small}
\addtokomafont{author}{\normalfont\small}
\addtokomafont{descriptionlabel}{\normalfont\bfseries}

\addtokomafont{disposition}{\osfamily}

% \renewcommand{\thesection}{\Roman{section}} 
% \renewcommand{\thesubsection}{\thesection.\Roman{subsection}}


\subject{}
\title {    \begin{center}
        \includegraphics[scale=0.5]{philfak.jpg}
    \end{center}
     Veritism and Epistemic Normativity }
\author{C. Friedrich}
\date{\today}

\publishers{ University of Cologne \\
    Department of Philosophy \\
    Bachelor Thesis \\

}

\makeatletter
\renewbibmacro*{cite:ibid}{%
  \iftoggle{cms@noibid}%
    {\blx@ibidreset\usebibmacro{cite}}%
    {\printtext[bibhyperref]{\bibstring[\mkibid]{ibidem}}}}
\makeatother


\begin{document}
\begin{titlepage}
\maketitle

\abstract{\textbf{Abstract.} Can epistemic normativity be plausibly explained by appealing to truth as the single final epistemic value? [All structural elements and section titles are placeholders.]}

\thispagestyle{empty}
\end{titlepage}

\tableofcontents
\newpage

\section{Introduction}

\begin{quote}
    It is not that hard to be skeptical about value. You can't see values. You can't touch them. And many think we can't even define them. Partly as a result, our disagreements about value have an interminable and intractable feel. Philosophically speaking, skepticism about value seems an easy sell.

    At least this is the case when we are talking about the type of value that most philosophers worry about—moral value. But we have other values besides moral values. And one of the most basic of these other values is the value of truth. \autocite[225]{Lynch2009}
\end{quote}


In a certain sense, and one that, for example, \textcite{Berker2013-BERETA-2} holds, epistemology is an inherently normative endeavor. To Berker the most fundamental question of epistemology is, despite the etymological origins of its name, not concerned with what makes a belief count as knowledge but is instead purely normative: “What should I believe“?

In pondering this question the similarities of epistemology to the theory of normativity and especially ethics are strikingly obvious, if only at a first superficial glance. It makes a lot of sense then, if one is to undergo the project of viewing epistemology through the lens of normativity, to at first take a look at ethical theories of value and normativity. While the properties of epistemic and ethical objects of interest may stop sooner than expected, the groundwork in ethical theory shouldn't be completely disregarded, either.

Go on and on

It's really unübersichlicht out there, norms and values seem to be thrown around. I wanted to put things a bit in perspective and say where this debate takes place in epistemology, what the relevant questions are, and if there 
\begin{itemize}
    \item Where are we in epistemology? Why is a normative approach important / interesting?
    \item explain very clearly and in formidable prose what it is this paper is trying to do, how it will be done, and what it accomplished (obiovusly, write it at the end).
\end{itemize}

\begin{quote}
 I suggest that the primary function of cognition in human life is to acquire true rather than false beliefs about matters that are of interest or importance to us. \textcite[29]{Alston2005-ALSBJD}
\end{quote}

\begin{quote}
What makes us cognitive beings at all is our capacity for belief, and the goal of our distinctively cognitive endeavors is truth: we want our beliefs to correctly and accurately depict the world. \textcite[7]{Bonjour1985}
\end{quote}

[More quotes can be found in \textcite{Goldman2002-GOLTUO-2}.]
\footnote{As may be obvious, I presuppose a realist notion of truth. As perhaps less obvious, throughout this paper I assume that every proposition is either true or false and not neither or both. I won't discuss complications that arise for the veritist account if one supposes, say, dialetheism. Let's further suppose that eternalism about propositions is true, that is, one and the same proposition doesn't change its truth value, ever, in order not to run into complications when faced with a challenge from relativism brought forth by \textcite{Brogaard2008-BROTTA-3}.}

\section{Values and Norms}\label{sec:values}

\begin{quote}
I don't know how to prove that the acquisition, retention, and use of true beliefs about matters that are of interest and/or importance is the most basic and most central goal of cognition. I don't know anything that is more obvious from which it could be derived. But I suggest that anyone can see its obviousness by reflecting on what would happen to human life if we were either without beliefs at all or if our beliefs were all or mostly false. Without beliefs we would be thrown back on instinct as our only guide to behavior. And as far as thought, understanding, linguistic communication, theorizing, science, art, religion-all the aspects of life that require higher-level cognitive processes-are concerned, we would be bereft of them altogether. And if we had beliefs but ones that were mostly false, we would constantly be led astray in our practical endeavors and would be unlikely to survive for long. \textcite[30]{Alston2005-ALSBJD}
\end{quote}

Why is it ever that anyone \emph{should} believe something, or is \emph{required} to believe, or is \emph{justified} in believing, or is \emph{rational} to believe? What is that yields some kind of normative requirement to the believer? A most natural answer, one that immediately springs to mind, is it's connection to the truth. You should believe that proposition because it is most likely to be true! (You are justified in believing that proposition because it is the only one that your evidence supports. It would be completely irrational to believe that other proposition, you don't have any reason to believe it.) What else than true belief is the epistemic goal of anyone in the business of believing anything, that is, a cognitive agent? It is this very intuitive thought that motivates postulating truth as an–or the–epistemic value.

When talking about values, the deontic or about normativity in general, it is helpful to look for established theories as a guideline. Naturally, these theories have been developed in the realm of practical philosophy, so that's where I'll turn for some clarificatory groundwork.

A good starting point to look for some guidance when talking about values are the fields of value theory or, more specifically, axiology.

\subsection{ Taxonomy of the Normative}

To clarify the notions, I will very shortly introduce the normative terminology and how it will be used it in this paper.

I divide normativity roughly into axiology, the study of values, and deontology, the study of rules and norms. Axiology is concerned with what is good or bad and what is better or worse. Axiological claims are therefore \emph{evaluative}. Deontology is concerned with what is required, forbidden, permissible etc. Deontic claims are therefore prescriptive or \emph{regulative}. There can be evaluative norms, specifying what is of value, as well as deontic norms, specifying what is to be done or believed.

Any account of normativity aspires to explain the axiological and deontological questions. The tension between different types those theories will be of some import for the present paper, and I will circle back to this topic frequently.

\subsection{ Values }
What \emph{is} a value, then? \textcite[79]{scanlon1998} puts it thusly: “ ... a notion of how it would be best for the world to go, or of what would be best for particular people”. In addition, it is what makes evaluative sentences like “pleasure is good” come out (perhaps) true. There is something that corresponds to these sentences, for instance a property of pleasure, that gives “pleasure is good“ a status as a true sentence. Since axiology is defined above in terms of claims of the form “pleasure is good” this approach seems alarmingly circular. It is such a simple and perhaps primitive notion. What's left is the appeal to intuition–certain things are intuitively good, or are better than others. That's what \emph{value} refers to.

- value come in degrees
- Intrinsic Value vs Extrinsic value

\subsection{ Instrumental Value vs. Final value }
\subsection{ Good vs. right }
\subsection{ value vs goal (can be footnote)}
\subsection{ Why talk about value? To get an account of normativity, of course! }

\subsection{ Teleology / Consequentialism} \label{subsec: teleology}
A theory of Normativity with a central value / central values as its fundamental principle.

The guiding motivation behind painstakingly creating a theory of value is not to stop there an be done with it, of course, but instead developing it into a 
full-fledged theory of normativity. Normativity includes the deontic as well, that is, the concepts of \emph{right}, \emph{reason}, \emph{rational}, \emph{ought} and so forth \autocite[21]{sep-value-theory}. Values and deontic norms are two central concepts in normative theory. Which of these two is the basic or fundamental one through which the other concept can be explained or derived is an ongoing debate and roughly divides normative theorists into two camps: consequentialists (or teleologists) and deontologists. 

Consequentialists hold that all regulative norms usually phrased with \emph{oughts} and the like obtain in virtue of being directed at a value or goal. In slogan form, they put the good prior to the right. The basic idea is to first ask “what is the best (i.e. most valuable) action / belief / situation?” and then ask “what do I do to achieve it?” to arrive at what one \emph{ought} to do. „However much consequentialists differ about what the Good consists in, they all agree that the morally right choices are those that increase (either directly or indirectly) the Good.“ \autocite{sep-ethics-deontological}

Deontologists hold that all evaluative norms obtain in virtue of being directed by a regulative norm. In slogan form, they put the right prior to the good. They start out with asking “what should I do?” and then determine the most valuable action / belief / situation on this grounds. 

So the direction of explanation is reversed in these types theories, but instances of both types aspire to account for the \emph{good} as well as the \emph{right}, in other words, aspire to present a complete account of normativity \parencite[341]{Berker2013-BERETA-2}. This distinction is, as in most fields of philosophy, not as clear-cut as it might seem though, and there are theories that neither fall clearly on one side or the other of the divide and may incorporate features of the opposed theory.

This is a very rough sketch of teleological / deontological accounts of normativity that leaves most everything to be desired, to be sure, but for my purposes it's enough give a quick overview what this distinction is all about.

\subsection{Epistemic Normativity} \label{subsec: epistemic}

Veritism is a theory about value, so in that sense it determines the questions regarding axiology in a normative framework. To be compatible with veritism, a deontic theory has to accept that the only fundamental epistemic value is truth or true belief. Veritism combined with such a deontic theory amounts to a teleologist or consequentialist account of epistemic normativity in which deontic concepts are explained by appeal to veritism.

The discussion of the normative has been quite general so far, in order to be compatible with ethics as well as epistemology. Veritism is an epistemological position, so what does that mean for normative theories about it? Most of what has been said translates directly into the \emph{epistemic domain}, meaning anything that pertains to concepts like belief, truth, knowledge and justification \autocite{David2001-DAVTAT-7}. Even \textcite[7]{grundmann2008}, who holds that conceptual analysis in epistemology precedes the normative inquiry, emphasizes the normative character of justification, one of \emph{the} central concepts in epistemology \autocite[226]{grundmann2008}. Still more prevalent is the connection of epistemological concepts to the normative in the \emph{deontic conception of epistemic justification}, presented by \textcite{Steup1988-STETDC} and argued against by \textcite{Alston1988-ALSTDC}. The deontic conception states that epistemic justification can be “cashed out“ out in terms of permissions, obligations and so forth. This goes one step beyond the claim of Grundmann, who merely notes the evaluative normative character of epistemic justification.

So the interesting question is whether veritism can account for and explain the normative character of epistemology. What are those normative concepts, and how can it be done? I see two basic different approaches:

\begin{enumerate}
    \item Determine the exact content of the goal stipulated by veritism. What exactly is the goal of epistemology? By finding out the most plausible version an exact description of the goal, one thereby acquires a putative description of the ideal epistemic state of a believer. Taking this description as a regulative ideal, one can try and work out the most plausible epistemic regulative norms, that, if followed, brings one closer to being in the ideal epistemic state
    \footnote{That this is not as easy as it might sound is \textcite{Gibbons2013-GIBTNO} major point. There is a huge gap in explanation between the objective goal of truth and regulative norms pertaining to the individual subject. These even come in conflict, in what he calls \emph{the puzzle}. However, \textcite{Goldman2002-GOLTUO-2} does not think this problem grave at all. [Go on A BIT and clarify own position]}.
    If and how these regulative norms so construed pertain to the concept of justification or knowledge is a different question, and, depending on one's assumptions, can follow directly or not at all.
    \item Stipulate that the normative conception of epistemology just is a feature of what the central epistemic concept, depending on your theory of choice \emph{justification, reason, evidence, rationality, warrant} or \emph{intellectual virtue} \autocite[153]{David2001-DAVTAT-7}, amount to. On this approach, what it means to be justified (or rational, reasonable..., from now on justified for short) is in some way or other derived from veritism. Then, the assumption that these concepts are inherently normative in nature provide a basis to reason for the binding normative power of this concept. That is, norms stating that it is good to be justified in believing, or that one ought to be justified in believing.

    It is important to note that this is \emph{not} the same or similar to a deontic conception of justification. Whereas the deontic conception holds that  justification obtains in virtue of compliance with deontic norms,
\end{enumerate}

On many accounts[citation?], (2) is the preferred strategy, as is grants an important role to the central epistemic concept of justification. In strategy (1), justification seems like a mere afterthought and there are no attempts at explaining it built in.

Note that neither of both approaches is necessarily stuck to the position that there can only be evaluative or only regulative norms. This is a different question which will be addressed later on (-> ref!).

\subsubsection{How Can Veritism Ground Epistemic Normativity?}

It is perhaps not obvious why it is that, given veritism, regulative norms of belief have any normative force. How is the normative force grounded in veritism? In virtue of what are these norms authoritative? It may be that true beliefs are good, or valuable, but why should that require me, as a believer, to follow norms given provided by veritism? This problem is especially salient when additionally stating that beliefs have a constitutive aim of being true, as \textcite{Cote-BouchardForthcoming-CTBCTA} points out. His challenge goes something like this: Suppose that there is a true evaluative norm of belief, of the form
\begin{description}
    \item[TN] A belief is correct if and only if it is true. \autocite{Wedgwood2002-WEDTAO}
\end{description}
For epistemic normativity to be binding, it needs to be the case the there is necessarily a good reason to follow epistemic norms \textcite[13]{Cote-BouchardForthcoming-CTBCTA}. For TN to deliver such a verdict, we need to make the additional assumption that
\begin{description}
    \item[RTN] There is necessarily a good reason to believe correctly.
\end{description}
However, this does not follow from TN. And is has to follow from TN, since TN is supposed to–on its own–explain the force of epistemic normativity. So to say that there is a true norm of belief of this kind does not imply anything about there necessarily providing me with good reasons to believe according to that norm. For consider the analogy: “writing the same number twice in one of the columns of a Sudoku grid is incorrect relative to the constitutive norms or rules of Sudoku. Yet there might be no good reason for me to avoid that incorrect Sudoku move.” \autocite[12]{Cote-BouchardForthcoming-CTBCTA}. So although one may grant TN, RTN still does not have to obtain.

One objection that springs to mind is that there is a role-ought that applies here [cite feldman]: It is (i) our role \emph{qua believer} that forces us to obey the norms and (ii) occupying the role of a believer is inescapable for any human being, or more abstractly, anyone capable of intelligence or agency, so most anyone that epistemology has anything to say about.

\textcite[9]{Cote-BouchardForthcoming-CTBCTA} presents some counterexamples as a reply to this: first, being in the role of something does not imply normativity. Consider someone in the role of a torturer. \emph{Qua torturer} he ought to make his victims suffer. But it would be rather cynical to ascribe good reasons to do so to the torturer. Secondly, inescapability of a situation does not make it normatively relevant. Consider the alcoholic, who can't escape wanting a drink. But does she have good reason or is justified in “binge drinking”? So neither of those features conduce towards normative requirements.

However, what has to be addressed here is that the role of a believer is aimed at a good thing, at something valuable. That is what veritism states, after all. The \emph{combination} of these claims makes all the difference: any epistemologically relevant subject is in the position that (i) she is in the role of a believer (ii) this is necessarily and not only contingently so and (iii) only true belief is good belief. What Cote-Bouchard has shown is that neither of these conditions alone is sufficient to justify RTN. But as a believer TN applies to me, hence I have good reasons to believe correctly \emph{qua believer}. Since I'm necessarily in the role of a believer, and believing correctly is a good thing (as stated in veritism), I necessarily have a good reason to believe correctly.

If this objection works, it lifts the normativity of belief \emph{out} of the merely epistemic domain. So I do not merely have an epistemic reason, but a good reason, \emph{simpliciter}
\footnote{This does not amount to all-things-considered reason, however, as there might be other, heftier reasons that outweigh my good reason derived from truth. This, though, is another contentious topic, and some hold the position that only epistemic reasons can give me a reason to belief anything \autocite{Kelly2003-KELERA}}.
But not much hangs on this claim for the purposes of this paper, since for epistemic normativity to be relevant, it is enough to just apply to the domain of normativity. This question becomes relevant again as soon as one tries to compare epistemic (or intellectual) reasons with practical (or other minds of)reasons to determine an all-things-considered reason. That is not the target of this paper, though. If this norm is in fact  only hypothetical on my role as a believer–so be it.

\subsection{ What is the bearer of epistemic value? -> states of affairs}

\section{Veritism}
\subsection{Varieties of Veritism}\label{sec:varieties}

\textcite[54]{Goldman2002-GOLTUO-2} defines veritism as  “... the unity if epistemic virtues in which the cardinal value, or underlying motif, is something like true, or accurate, belief”. Borrowing the terminology from Goldman, \textcite[360]{Berker2013-BERETA-2} proposes to define veritism as the position that „ ... our only epistemic goals are (i) the accumulation of true beliefs and (ii) the avoidance of false beliefs”. \textcite{Zagzebski2004-ZAGEVM-2} has a different name for a very similar position, what she calls epistemic value monism: “Any epistemic value other than the truth of a belief derives from the good of truth“. 

The unifying theme of these quotes is apparent: What’s central to all three accounts clearly is the notion of \emph{true belief}. True, or accurate, belief has to be a value, goal, function or motif with a distinct role in the theory. This role is determined as cardinal, single, primary and non-derivative. In other words, veritism says that true belief has to be regarded as the only \emph{final} value. No other epistemic concept like knowledge or justification has final epistemic value. It is important to note, however, that this does \emph{not} entail that other epistemic concepts have \emph{no} epistemic value, they just don’t have \emph{final} epistemic value.

So it is perfectly compatible with veritistic accounts to say that epistemic justification is valuable and that it has a value distinct from the value of true belief. Justification does not, however, have final or intrinsic epistemic value.

\subsubsection{The Value of Believing}
That is the gist of the position: veritism claims that true belief is the only epistemic value or goal. In some sense it is an ontological claim about the domain of epistemology: Veritist describe the property of value to true beliefs, or the existence of of a unifying goal for all believers. It is somewhat removed from psychological considerations, but that was to be expected, given this inquiry into the normative.

\begin{description}
    \item[Simple Veritism:] True belief is the only state of affairs that has final epistemic value.
\end{description}

\begin{itemize}
    \item of course, this is still underspecified. What exactly is of value? Having as many true beliefs as possible? How are true and false aggregated to accommodate something like a total value?
    \item add a comparative notion to it.
\end{itemize}

Can this account explain intuition about scenarios concerning epistemic value?  A clear-cut case: two propositions $P$ and $Q$ where $P$ is true and $Q$ is false–it couldn't be more obvious: believing $P$ is more valuable than believing $Q$. This very natural thought directly leads to a comparative notion which holds that true belief is more valuable or better than false belief. Simple enough.

Now what if we want to evaluate the doxastic system of a person, or her intellectual attainment, as \textcite[58]{Goldman2002-GOLTUO-2} puts it? This seems just as straightforward. The more true beliefs a person has, the more valuable the position that she occupies. So no matter the epistemic situation she am in, by acquiring more true beliefs she can better her epistemic standing. So far, so good. Goldman\textcite[59]{Goldman2002-GOLTUO-2}

All is not so peachy, though. Consider these two:
\begin{description}
    \item \emph{Gilbert Gullible} believes everything anyone tells tells him, anything he thinks about, in general any proposition he encounters.
    \item \emph{Priscilla Precise} is more careful in forming her beliefs. She carefully weighs her evidence and, as a result, has a lot fewer beliefs than Gilbert, but her beliefs are largely true.
\end{description}

Intuitively\footnote{I make a lot of assumptions about intuitions here. Granted, this is not ideal, but I tried to only incorporate cases where the intuitions seem uncontroversial. As these are empirical claims, of course all of my assumptions about clear-cut intuitions are open to challenges from experimental philosophy.}, it is obvious that Priscilla occupies the more valuable epistemic state. Yet on the account just sketched, Gilbert amasses lots and lots of beliefs, among these many true beliefs. Hence his epistemic situation is immensely valuable. Priscilla on the other hand has fewer true beliefs to show for, her epistemic situation is therefore less valuable. This is a terrible result for the simple account.

Gilbert’s situation is an instance of what \textcite[360]{Berker2013-BERETA-2} calls \emph{epistemic recklessness}. 

To avoid this result, it seems only natural to expand the epistemic goal to consist of to goals, really: ”...the twin goals of acquiring true beliefs and avoiding false ones“ \textcite[339]{Berker2013-BERETA-2}. This dualistic rendition of epistemic value has already been proposed by \textcite[17]{James1896-JAMTWT-19}, who coined the phrase: ”Believe truth! Shun error!“.

\begin{description}
    \item[Dualistic Veritism:] True belief and the avoidance of error are the only two states of affairs that have final epistemic value.
\end{description}
\label{SECTION SO UND SO}
What does it mean to avoid error? An error in this sense \textcite[362]{Berker2013-BERETA-2} is a false belief. But in everyday language, disbelieves in true propositions are errors as well. I will discuss this complication later on \ref{SECTION SO UND SO}. What is avoidance, then? In a narrow reading, only disbelieving or suspension of judgment towards a false proposition count as avoiding errors. Interpreted more loosely, just being ignorant about a given proposition can count as avoidance, too. This relates to the criterion that determines which propositions actually contribute to the epistemic value, I will discuss it in \ref{SECTION SO UND SO}.

So to maximize my epistemic value, I seek to have as many true beliefs and as few false beliefs as possible.

This account fares better with respect to Gilbert and Priscilla. As Gilbert does have some true beliefs that count towards a valuable position, Priscilla’s carefulness now brims with epistemic value as she avoids most of the errors that Gilbert makes. So this intuition, at least, can be explained. Splendid!

But let's revisit one of the key merits of veritism at the core of any deontic framework. One central motivation is to unify all epistemic evaluation. This approach reaches as far back as Socrates, who proclaimed: ”Virtue is one!” and meant it quite literally \textcite{penner1973}, entailing that what Socrates regarded as virtues where really all the same thing. So bravery, wisdom, temperance, justice, piety and even knowledge amount to something equivalent, if not identical.
\textcite{Goldman2002-GOLTUO-2} picks up the ball in his self-appointed task to unify all epistemic virtues\footnote{Now, virtues are not the same thing as values or goals, and I don't want to get into a discussion about virtue epistemology at all here. What Goldman concedes, though, is that his conception of epistemic virtues builds upon or at the very least is compatible with that form of consequentialism value-directed accounts present us with.}. As one might think, he is not so quick to drop value monism to make it two, that is to switch to a dualistic account, which would pretty much mean to give up the idea of a single unifying epistemic virtue.

\textcite[58]{Goldman2002-GOLTUO-2} proposes an a little more complex form of veritism which takes its motivation from the model of partial belief, or degree of belief, or levels of confidence, or credences. Let's first get clear on the different notions of belief he employs:

\begin{quote}
    First, we can use the traditional classification scheme which offers three types of credal attitude toward a proposition: believe it, reject it (disbelieve it), or withhold judgment. I call this the trichotomous approach. Second, we can allow infinitely many degrees or strengths of belief, represented by any point in the unit interval (from zero to one). I call this scheme the degree of belief (DB) scheme. \textcite[88]{Goldman1999-GOLKIA}
\end{quote}

So nothing non-standard here. Note, that we want to account for each of the doxastic attitudes belief, disbelief and suspension of judgment.

In the model of degree of beliefs, it is not mere true belief that has final epistemic value. Instead, the value of a credence in a true proposition derives its value as a function of its degree: maximal value if the degree is maximal, and minimal value if the degree is minimal. This yields a slightly different comparative notion than before: A degree of belief in a true proposition is more valuable than another degree of belief simply if it is higher. Given a true proposition $P$, your credence of $0.9$ is more valuable than mine of $0.4$, simple as that.
Goldman then supposes a workable way to translate degrees of beliefs into full beliefs, suspension of belief, and disbelieve, by some threshold measure that is left unspecified. He can then compare: ”... believing a truth carries more veritistic value than suspension of judgment; and suspension of judgment carries more veritistic value than disbelief“. This leads to a comparative notion:

\begin{quote}
    If a person regularly has a high level of belief in the true propositions she considers or takes an interest in, then she qualifies as “well‐informed.” Someone with intermediate levels of belief on many such questions, amounting to “no opinion,” qualifies as uninformed, or ignorant. And someone who has very low levels of belief for true propositions—or, equivalently, high levels of belief for false propositions—is seriously misinformed. \textcite[58]{Goldman2002-GOLTUO-2}
\end{quote}

In this way, Goldman concludes, the veritistic account can accommodate everything an account with two epistemic goals instead of one can. In particular, he claims that that having very low levels of belief for true propositions is \emph{equivalent} to having high level of belief in false propositions, at least insofar as the epistemic value is concerned. On this account, a credence of $0.2$ in a true proposition has the same epistemic value as a credence of $0.8$ in a false proposition, and both counts equally towards a person being seriously misinformed. For Goldman, this notion of comparative value does justice to our intuitions regarding relevant cases.

\begin{description}
    \item[Accuracy Veritism:] Accurate belief is the only state of affairs that has final epistemic value.
\end{description}

\subsubsection{The Value of Disbelieving}

However, as I see it, there are problems with this translation of conclusions drawn with the notion of degree of belief to the notion of full belief that Goldman does not seem to take into account. It might be interesting to have a look at the notion of belief and disbelieve that Goldman employs, as it does a lot of work in explaining the value of the doxastic attitudes.

Goldman claims that disbelieving a proposition $P$ is equivalent to believing a proposition non-$P$ \textcite[58]{Goldman2002-GOLTUO-2}, so that I disbelieve a proposition $P$ if and only if I believe a proposition non-$P$. That is, whenever it is true that i disbelieve $P$, it is also true that I believe non-$P$, and whenever it is false that i disbelieve $P$, it is also false that i believe non-$P$, and \emph{vice versa}. I presume this is a consequence from the translation of the notion of degree of beliefs to the notion of full beliefs. Given scaling of degree of beliefs on the unit interval, whenever I believe $P$ to the degree $c$, I also believe non-$P$ to the degree $1-c$. For example: I am pretty unconvinced of $P$, my credence is $0.1$. Hence, my credence in non-$P$ is $0.9$. Translated back to full beliefs, this would then amount to belief and disbelieve, if $c$ is sufficiently low for disbelief. So Goldman's claim about equivalence would follow. What shouldn't be disregarded, though, is that this only applies to agents that obey the axioms of probability theory, in many theories a necessary condition on rationality of degree of beliefs. It is still very much conceptually possible to believe $P$ to the degree $0.8$ \emph{and also} believe non-$P$ to the degree, say $0.7$. I wouldn't be quite rational to do so, of course, but it is certainly possible. Now, Goldman does not state conceptual identity, granted. Equivalence is still a very strong claim, but: my believing $P$ to the degree $c$ \emph{does not entail} my believing non-$P$ to the degree $1-c$ (or vice versa). So, translated back to full beliefs, the equivalence of disbelieving that $P$ and believing that non-$P$ does not follow from the notion of degree of beliefs\footnote{The notion of translating between different models of belief formation is not self-explanatory and seems in need of an independent argument. Let's suppose, for the moment, that such a sufficient argument has been presented.}.

One could object to this that this is not at all a move from degree of beliefs to full beliefs. Instead, is just very naturally follows from thinking about doxastic attitudes. What else should disbelieving $P$ amount to, if not believing that non-$P$? When I disbelieve $P$, I hold that $P$ is false. But believing that $P$ is false \emph{just is} believing that non-$P$ is true. The argument looks something like this:

\begin{enumerate}
\item [P1] If I disbelieve that $P$, I believe that <$P$ is false>. 
\item [P2] <$P$ is false> is equivalent to <non-$P$ is true>.
\item [P3] Belief is closed under single premise logical entailment.
\item [C1] If I disbelieve that $P$, I believe that <non-$P$ is true>. (From P1 to P3)
\item [P4] Believing that <$Q$ is true> \emph{just is} to believe that $Q$.
\item [C] Therefore, If I disbelieve that $P$ then I believe that believe that non-$P$. (From C1 and P4)
\end{enumerate}

A similar argument may be sketched for the opposite direction of entailment.

But I would hold that I presumably can disbelieve $P$ and also not believe non-$P$, that is, disbelieve non-$P$ or suspend judgment towards non-$P$. There is no entailment relation between either of these doxastic attitudes. What goes wrong in the argument? P1 may just follow from your definition of disbelieve. I won't argue against that here, or ever, so this premise seems fine. P2 just is a necessary statement given that every Proposition is either true or false. P4 seems like a truism, although one might argue against it on grounds that a conception of truth or true belief is not necessary for a subject to hold true beliefs, however, let's grant this as well. P3, on the other hand, is the most dubious candidate. While it is a hot topic whether knowledge is closed under entailment\footnote{See, for example, \textcite{Dretske2005-DREIKC}.}, it is a lot less plausible that \emph{mere belief} is closed under (single premise) entailment. For I can, of course, believe that $P$, where it is some necessary truth that If $P$, then $Q$, but still fail to grasp that $Q$ obtains. Without this or a similar premise\footnote{Perhaps the premise that it is closed under \emph{obvious} single premise closure, but then, I think, the same counterexamples work.} the argument just isn't sound.

So, I suggest that yes, of course, <$P$ is false> is equivalent to <non-$P$ is true>, however, disbelieving that $P$ is not equivalent to believing that non-$P$.

[The following is BS]
Would it be so, I could never believe a direct contradiction, not only not rationally, but not with good reasons, either\footnote{As in the case of certain paradoxes, for example the liar paradox, as one might argue \textcite[415]{Priest1998-PRIWIS}}.

Goldman can still retreat to the position that the \emph{epistemic value} of disbelieving $P$ is equivalent to the \emph{epistemic value} of believing non-$P$. Let's suppose $P$ is false, and hence non-$P$ true, then on this account, disbelieving $P$ has epistemic value–the corresponding credence is very close to the actual truth value– and believing non-$P$ has value, since one believes a true proposition, namely non-$P$. So what can be accounted for is the dual value of disbelieving a falsehood and believing a truth. This position seems a lot more plausible, then. Perhaps it is what Goldman had in mind, anyway. What does it help veritism with? Consider the case:

\begin{description}
   \item \emph{Stephanie Skeptical} is in a very unfortunate situation. Her questionable colleagues all present her with unconvincing propositions. As a result, she comes to disbelieve most of them. And rightly so, all of them are false.
\end{description}

We intuitively want to say that Stephanie's disbeliefs contribute something epistemically valuable to her situation. Yet, on the simple account of veritism, I could only ascribe value to her disbeliefs if I presume the equivalence stated by Goldman, that her disbelief in the false proposition $P$ entails a believe in the true proposition non-$P$. On the dualistic account, her disbeliefs would count as an avoidance of error only if (a) the equivalence claim above is supposed true, or (b) error is meant to encompass disbelieve in false propositions as well. Option (a) would presuppose to much to be incorporated into the value directly, for my taste, whereas option (b) seems \emph{prima facie} to be a plausible candidate to deal with this problem, however, isn't emphasized that much in the literature.

Goldman's accuracy account is partly designed to handle exactly this type of case. Stephanie's disbeliefs all have epistemic value equivalent to that of true beliefs. Hence, her epistemic situation is valuable, agreeing with intuition.

\subsubsection{The Value of Suspending Judgment}

What about suspension of judgment? Given a true proposition $P$, it is most valuable to belief that $P$. It seems less valuable to suspend judgment towards $P$, and it is even less valuable to disbelieve $P$. Conversely in the case of a false proposition $Q$: disbelieving is most valuable, suspension less so, and believing that $Q$ least valuable. Intuitively then, there is a clear ranking of these types of doxastic attitudes. On the simple account, none of the features of the ranking save true belief can be explained. The dualistic account with the above modification regarding error gets two out of three right, it does not say anything about suspension of judgment though. What should it count as? Is it some form of true-at-least-a-bit belief, or should it be regarded as an error? Neither of these seem plausible at all. So if we are to include suspension of judgment into our notion of doxastic attitudes, both the simple and the dualistic veritistic account can't seem to handle the intuitions regarding its value. The accuracy veritist is able to cope a lot better, as suspension is directly incorporated into the comparative notion that Goldman spelled out.

However, there might still be a problem: [Insert Dominiks Argument here] i suggest that the translation to a model of full belief does not work in all respects: A degree of belief of 0.5 is not the same a suspension of judgment, which does not entail a doxastic attitude towards a proposition \footnote{I owe this point to \textcite{Balg2018} }

So far, it seems that accuracy veritism comes out on top, at least if we only consider cases in which there is a fixed set of propositions against which we evaluate different intellectual attainments. Consider again the case of Gilbert and Priscilla: Gilbert is right quite often, but wrong even more often. Priscilla is right in most of her beliefs, but has fewer of them, so is not right as often as Gilbert is. On Goldman's account then, one accumulates epistemic value by believing truths and disbelieving falsehoods, but also by suspending judgment, regardless of the truth value of the proposition. Isn't it still the case that Gilbert accumulates more value by believing all those true propositions? He accumulates a lot, that much is correct, but reading Goldman charitably we might ascribe to Priscilla that she withholds judgments in many of the cases which Gilbert gets wrong. Thus, she accumulates lots of additional value by being indecisive. I would concede that this point is controversial, for it is not the same thing to withhold judgment and to be ignorant about a proposition. Being ignorant does not promote any epistemic value, and isn't Priscilla ignorant about those proposition, really? Whatever the answer, one might object that there is a slight alteration to the Gilbert and Priscilla case that might pose a new problem. Consider:

\begin{description}
    \item \emph{Hesitant Howard} is too indecisive to belief most anything, he withholds judgment on every single proposition he is not perfectly sure of. And he is almost never sure of anything! As a result, most (if not all) of his doxastic attitudes are suspensions of judgment.
\end{description}

Of course, Howard is not in an epistemically notably valuable position. And his situation certainly is not better than that of Priscilla. But doesn't the accuracy account produce the verdict that Howard's situation is more valuable, given that he accumulates value for each suspension of judgment, while Priscilla doesn't? The response is analogous to the previous case. If Priscilla is ignorant or without doxastic attitude towards most of the propositions that Howard withholds judgment on, the accuracy account might produce the verdict that Howard accumulates more epistemic value than Priscilla does. If we grant, however, that Priscilla herself suspends judgment on these propositions, then her overall positive track-record will give her a slight to large edge over Howard and therefore make the account's verdict agree with intuition.

If we stipulate, though, that Priscilla has doxastic attitudes towards a given limited set of propositions, but which are mostly accurate, whereas Howard's set of proposition he withholds judgment on is significantly larger, the accuracy account would value Howard's situation as the more epistemically valuable. I argue that these cases in which the set of propositions in question differ in such a significant way may yield less clear intuitions about the epistemic value. Intuitively, there might be something of value in the mere range of propositions that Howard is familiar with, and even though he is not knowledgeable about most of them, just having this many doxastic attitudes may count some way towards epistemic value.

I would propose to set aside such edge cases and predominantly compare epistemic situations on opinionated, fixed or comparably-sized sets of propositions–doxastic attitudes are, after all, what we want to compare the value of.

\subsubsection{Summary}

In this section, I present simple and straightforward forms of spelling out what a veritist account of epistemic value might look like, and compare their verdicts against intuitions about epistemic value in ready-made and non-obscure cases. I then evaluate which, if any, of the presented accounts seem plausible enough on this first challenge to then be incorporated into a theory of epistemic normativity. It becomes apparent that getting clear on what it is, exactly, that veritism states, is not as straightforward as one might think. Simple veritism can't accommodate the value of disbelieving a false proposition. Dualistic veritism has problems explaining intuitions about the value of suspending judgment. Accuracy veritism is not able to properly explain intuitions about comparing the value of wildly diverging sets of beliefs, however, that problem might not be of so much import after all.

To make headway, I propose to accept the account of accuracy veritism for the time being as a predominantly plausible one. To avoid complications, I will still talk of veritism as employing true belief as the only final epistemic value with the notion of accuracy in mind.

Next, [what is next?]

\subsection{Veritism and Other Accounts}

Veritism proposes that true belief is the only final epistemic value. This would not be saying much if there were not different accounts of value that try to explain epistemic normativity from a different starting point. In this section, I will outline some of them and give an overview of the different accounts in use in contemporary normative epistemology.

\subsubsection{ Truth Value Monism }
I stated in section \ref{sec:varieties} that \textcite[191]{Zagzebski2004-ZAGEVM-2} has a different name for a very similar position to veritism: epistemic value monism. I will label this position \emph{truth value monism}, however, since epistemic value monism seems to be somewhat of a misnomer for the position that Zagzebski sketches: “Any epistemic value other than the truth of a belief derives from the good of truth.” Although this is a case of epistemic value monism, that is, a position that stipulates a single final epistemic value, other positions may also stipulate a single final epistemic value, most naturally knowledge, and thereby classify as epistemic value monism. So the term is ambiguous, hence truth value monism.
Is truth value monism the same as veritism? In section \ref{sec:varieties} is presented different renditions of what veritism might amount to, one of these simply stating truth as the final epistemic value. The other versions may still classify as truth value monism, depending on how tight you want to draw the concept, but i think the label veritism with the various addenda such as \emph{dualistic} much more appropriate as working name, so that is what I am going to use.
Terminology out of the way, Let's see what reasons there are to endorse veritism as a theory.

First, it is a most natural notion. It is not controversial at all to say that truth is valuable one way or the other, or that having a true belief about some matter is better than having a false belief about that matter. This much is conceded by pluralist notions as well a other epistemic value monism, such as a knowledge-first account, even if in the latter case true belief would be valuable only derivatively. This much is even conceded by deontic conceptions of epistemic normativity, which grant that there is value to true belief, but that true belief derives it value trough norm-compliance. The concept is so natural in fact, that some even argued that belief constitutively aims at the truth \autocite{Shah2003-SHAHTG,Velleman2000-VELOTA}. To say that much is not necessary for a veritist position, however.

Secondly, veritism packs a bunch of explanatory power. To see this, consider a very successful theory in traditional epistemology: process reliabilism. This theory is, as Goldman point out, consequentialist in nature \autocite{Goldman2002-GOLTUO-2} employing veritism as axiological groundwork. I discuss extensively that this is not without problems in section (REF->). But even in formal epistemology, veritism has explanatory power and can help vindicate Bayesianism, or so at least \textcite{Joyce2009-JOYAAC} proposes. Trying to vindicate probabilistic coherence has long mostly been a project taking its argumentative force from pragmatic considerations such as intuitions about pragmatic betting behavior. Setting it on alethic foundations, as Joyce puts it, is to present a purely epistemic argument for probabilistic coherence. He then goes on to assume Accuracy of degrees of belief as the epistemic value, which is quite strikingly a veritist variant.

Thirdly, it may be argued that parsimony is a veritable virtue of explanatory theories \footnote{In fact, \textcite[342]{Ahlstrom-Vij2013} posit some key desiderata for a satisficing epistemic axiology, chiefly among them a requirement for parsimony. They argue for it on the grounds that “one should prefer ontologies with fewer rather than more existential commitments, \emph{ceteris paribus}” \autocite{Ahlstrom-Vij2013}. This means that when comparing two theories with similar attractive features then, other things being equal, one ought to choose the theory with the fewer different entities assumed existing as possible. This intuitive principle in philosophy and the sciences was already proposed by Aristotle, Kant, Newton, Einstein, and Lewis among many \autocite[3]{sep-simplicity} and has its most famous philosophical realization in Occam's Razor, which can be stated as “other things being equal, if $T_1$ is more ontologically parsimonious than $T_2$ then it is rational to prefer $T_1$ to $T_2$” \autocite[7]{sep-simplicity}. Arguments for it range from just stating that parsimony is a primitive value over analyzing the quality of actual empirical theories vis-à-vis parsimony to probabilistic arguments showing that rational agents assign distinctive prior probabilities to simpler laws \autocite[11-26]{sep-simplicity}. For the purposes of this paper I take it to be sufficient to assume this quite plausible principle without going any deeper into this discussion.},
and stating just one single epistemic value to explain all of epistemic normativity is of course quite parsimonious.
Forthly, stating a single epistemic value is one way of avoiding the incommensurability problem \autocite{sep-value-theory}. zZzzz...


    \subsubsection{ Truth-Value-Monism vs. Pluralism about values }
    \subsubsection{ Other-Value-Monisms (Knowledge, ...) }
    Veritist propose truth as the only epistemic goal. But why is that so? Woulnd't other candidates be a lot more plausible? After all, this is epistemology, and not alethology, or something like that.

\subsubsection{ TVM vs. Non-Value-based accounts of normativity }
    Deontological account of justification has the problem of explaining intuitions of the form you should believe it because its most likely to be true!

    Deontological accounts also need to provide an explanation for the problems posed by a false doxastic voluntarism and ought-implies-can. Veritism gets evaluative notions for free! Docastic voluntarism still is a problem for deontic requirements, though. Strategies: Flat-out deny deontic (regulative) requirements, or adopt one of the deontological accounts position and distinguish between different forms of doxastic voluntarism or deny ought-implies-can in one of its different forms. (Perhaps) we'll see later what these strategies exactly amount to .

\subsection{Objection from Significance}
    \subsubsection{ The content of the epistemic goal: An ideal state.}
    \subsubsection{ Objection from significance - Some true beliefs seem more valuable than other true beliefs.}

    Compare these two beliefs:
    \begin{enumerate}
        \item The universe is expanding at an accelerating rate.
        \item The number of people ever to have visited the David Hume memorial up until now is even. \footnote{Examples from \textcite{Ahlstrom-Vij2013}}
    \end{enumerate}


    How can the truth monist account for that? - Novel Idea: Perhaps not truth, but INFORMATION is the fundamental epistemic value. Look up Definition of information by shannon. The amount of information a message carries is determined by the number of alternatives that can be ruled out by the receiver. The more alternatives ruled out, the higher the information! Compare: The value of belief has is determined by the amount of possibilities (or possible worlds or other propositions) it rules out. This would explain why some true beliefs are more valuable than others, they carry more information. this carries the air of popper's verisimilitude for scientific theories, that should somehow measure its explanatory power (see keuth, my own paper).
    Moreover, it might even explain why justified belief is more valuable than mere true belief. Does it? How? Somehow the modal stability adds to the information contained? That seems a little far fetched but perhaps its worth a try. 
    What it DOES do is take the reek of practical value that comes with interest and significance! Argh! We veritists want delicious, pure epistemic value. If it is possible to define information in a way that it does not pertain to any practical evaluation and can be evaluated in epistemic terms only, its an account of significance or interest of true beliefs that is purely epistemic.

\section{Veritistic Accounts of Epistemic Normativity}
\subsection{Can There Be Norms of Belief at All?}
\subsection{ axiological / deontic part. Perhaps list desiderata from ahlstrom, or taxonomy by Berker}
\subsection{ If so, how plausible are they? Perhaps see Chisholm (quoted in \textcite{Goldman2002-GOLTUO-2})}
\subsection{Doxastic Voluntarism and Ought Implies Can}

\section{Veritistic Accounts of Epistemic Justification}
\subsection{General Structure of a Veritistic Normative Framework (Berker)}
\subsection{Process Reliabilism}

\subsection{The Value Problem}

One fundamental premise of the argument is that knowledge is more valuable than truth. Or more specifically, that the situation in which S knows that $P$ has more value than a situation in which S merely truely beliefs that $p$. As argued by \textcite[35-38]{grundmann2008} in fregean spirit, the truth of a belief is a property of the belief provided by its propositional content. Since a belief can only have one particular propositional content, the truth value of the proposition it is expressing is its sole source of direct value. Knowledge, on the other hand, does not seem to be a mere property of belief. Knowing that $P$ entails something else than just to have some belief that has the property of \emph{qualifying as knowledge}, or however you want to call it. Knowledge is a complex composite with one of its components being true belief\footnote{Of course, this is a highly controversial claim, argued against by many, most prominently by \textcite{Williamson2000-WILKAI}}. On one of the more plausible conception of knowledge compatible with process reliabilism and hence veritism called safety, a necessary condition for S to know that $P$ is that in each of the nearby relevant worlds, If S believes that $P$, $P$ is true. Bracketing out what is meant by \emph{relevant} and \emph{nearby}, what strikes me as uncontroversial is that the truth of $P$ in other possible worlds can't be plausibly said to constitute properties of the belief that $P$. Instead, it is something else (whatever it is), that in conjunction with the true belief that $P$ instantiates knowledge. So when asking if knowledge is more valuable than true belief, instead of evaluating just the belief and its properties one needs to evaluate this

against \textcite[291]{Stapleford2016}:

The epistemic domain is defined by the concept of knowledge. The concept is complex and it corre- sponds not to a single fundamental value but to a configuration of values, one fundamental and one (or more) derivative. Epistemic value is best thought of as a hybrid comprising the fundamental value of truth and the derivative value of justification.24 If this is right, then the objection misses the mark. Just as an authenticated painting has no more aesthetic value than an unauthenticated one, a justified true belief has no more alethic value than an unjustified true belief. But it does have more epistemic value.

This is incompatible with veritism according to with true belief is the only fundamental epistemic value. In Staplefords account, it would only be the only fundamental alethic value. The hybrid comprising truth and justification is of fundamental epistemic value, and truth has epistemic value only derivatively, just as justification only has epistmic and althic value derivatively.

With \textcite{Stapleford2016}:
To see how justification secures true belief against loss, con- sider what it’s like to believe P truly without justification. Suppose that you believe P because P was suggested to you when you were drunk and it seemed like fun at the time.29 This leaves your belief that P particularly susceptible to non-evidential influences and thus liable to random fluctuation. Say we try to coax you out of it: ‘Hey, we all believe not-P. Why don’t you? P sucks!’ If you didn’t have any good reason for believing P in the first place – if you just took it on a whim – then you have no good reason to stick with it now. You can drop P without a second thought. Whereas if you believe P with justification, you’ll think: ‘Why should I do that? P seems to be true. I am justified in keeping it.’ Justification thus gives your belief that P an added layer of protection against inad- vertent or arbitrary loss.

This seems to be a reasonable approach to stability of belief.


\subsubsection{ Value Problem aka Meno Problem aka Swamping Problem (Show differences). Explain what it is and why it hurts veritism }
\subsubsection{ Discuss how it could be solved in combination with accounts of justification (for example reliabilism) see Goldman/Olsson}
\subsubsection{ Own Discussion of a solution (perhaps from modal stability)}
\subsubsection{ The solution? Is there a solution without reliabilism, i.e. generic?}


\subsection{The Berker/Goldman debate on veritism}

\subsubsection{ Problems that \textcite{Berker2013-BERETA-2} raises}
\subsubsection{ Answers that Goldman gives}
\subsubsection{ This does not solve the problem for the veritist in general. Even without deontic definition, this problem arises(?)}
\subsubsection{ Berker divides consequentialist theories in three parts: theory of final value, theory of overall value, and a deontic theory, as opposed to just }divide into axiological and deontological considerations. 
\subsubsection{ Explain IN DETAIL what this means. }
\subsubsection{ For all (most) consequentialist theories a problem arises with firth-style cases. }
\subsubsection{ Explain IN DETAIL what these cases are, what and how the problem arises for the consequentialist theories}
\subsubsection{ Explain shortly what the reliabilist (i.e.) Goldman replies.  }

What is the exact distinction between Values, or goals, in general, and merely epistemic values, or goals? Epistemic Values pertain to something like Knowledge, Justification and Truth. But is that something that is really there or just some conventional means to divide and conquer the problem? Speaking from an evolutionary standpoint, it seems likely that we as humans are fitted with a natural goal of believing what maximizes our chances of survival. In most (if not all) cases, this goal coincides with believing what is true. But conceptually, these two are not the same. There may be cases where it would be best from a survival perspective, to believe that there is a predator in the surroundings, even when there is not, and so to believe falsely. So to purport the actual, final value of beliefs as being true seems, if nor fat-fetched, at least somewhat artificial. It is a very useful distinction, however, as the epistemologist can free her reasoning from any practical considerations that would otherwise be still on the table. However, if artificial, it might also not be what guides our intuitions about thought experiments. These intuitions may not share this artificial division of goals and subsume even epistemic evaluation under some, probably more practical, goals. I hold that these or some other not merely epistemic intuitions are what guides our evaluation in some cases. Consider a standard reply to a truth norm of the form: One should believe $p$ if and only if $p$ is true. This has the obvious consequence that one is required to believe every true proposition, no matter how menial or irrelevant to my current PRACTICAL purposes. I maintain that this relevance or interest is mostly practically motivated and hence should not be a part of any purely epistemic evaluation of the belief. From this perspective, every comparable true belief is comparably valuable in the epistemic sense. With comparable belief I mean beliefs that somehow have a similar epistemic status. They are of comparable generality and content. There may be differences in epistemic value, however, between a belief that is a very general (and true) law of nature and

Objection: 

The case intuition seems uncontroversial, however, it has also been empirically corroborated \textcite{Andow2016}
\section{Conclusion}

% \nocite{*}

\printbibliography

\end{document}