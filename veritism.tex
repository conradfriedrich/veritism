%!TEX program = xelatex
\documentclass[12pt,numbers=noenddot]{scrartcl}
% \usepackage[ngerman]{babel}
\usepackage[a4paper,lmargin={3cm},rmargin={3.5cm}, tmargin={2.5cm},bmargin = {2.5cm}]{geometry}
\usepackage{amsmath}
\usepackage{setspace}
\usepackage{enumitem}   
\usepackage{titling}

\onehalfspacing
% Entfernt Blocksatz!
% \usepackage[document]{ragged2e}

\usepackage[authordate, ibidtracker=context]{biblatex-chicago}
\addbibresource{veritism.bib}
\usepackage{url}
\urlstyle{same}

% \usepackage[disable]{todonotes} % notes don't show
% \usepackage[draft]{todonotes}   % notes show

% \usepackage[square]{natbib}
% \usepackage{jurabib}
% \usepackage {algorithm2e}

% Package, das die Benutzung von Old Standard erlaubt
\usepackage{fontspec}

% \setmainfont{OldStandard-Regular.otf}[
% Path = /usr/local/texlive/texmf-local/opentype/,
% BoldFont = OldStandard-Bold.otf,
% ItalicFont = OldStandard-Italic.otf]


% \setmainfont{Old Standard TT}
\setmainfont{Baskerville}
\newfontfamily\osfamily{Old Standard TT}
% \newfontfamily\osfamily{Helvetica}


\setkomafont{subject}{\small}
\addtokomafont{title}{\osfamily}

\addtokomafont{publishers}{\normalfont\small}
\addtokomafont{date}{\normalfont\small}
\addtokomafont{author}{\normalfont\small}
\addtokomafont{descriptionlabel}{\normalfont\bfseries}

\addtokomafont{disposition}{\osfamily}

% \renewcommand{\thesection}{\Roman{section}} 
% \renewcommand{\thesubsection}{\thesection.\Roman{subsection}}

\makeatletter
\renewbibmacro*{cite:ibid}{%
  \iftoggle{cms@noibid}%
    {\blx@ibidreset\usebibmacro{cite}}%
    {\printtext[bibhyperref]{\bibstring[\mkibid]{ibidem}}}}
\makeatother

\begin{document}
\begin{titlepage}
    \centering
    \vspace{0.5cm}

    \includegraphics[width=0.2\textwidth]{philfak.jpg}
    \par\vspace{0.5cm}
    { \large University of Cologne \par}
    { \large Department of Philosophy \par}
    \vspace{0.5cm}
    % { \large  \emph{supervised by} Prof. Dr. Thomas Grundmann \par }
    \par\vspace{0.5cm}
    {\centering
        \vspace*{3cm} 
        {\Huge\osfamily\bfseries Veritism and Epistemic Normativity \par}
        \vspace{1cm}
    }
    \vspace{1cm}


    \vfill

    { \large Bachelor Thesis\par}
    { \large \emph{submitted by} Conrad Friedrich \par}
    { \large \today\par}



    % {\footnotesize  \emph{conradfriedr@gmail.com} – Matrikelnummer: 4799305 – Franziskastr. 1 50733 Köln}
\end{titlepage}

\tableofcontents
\newpage

\section{Introduction}

\begin{center}
    \vspace*{2cm} 
    What should I believe?
    \vspace*{1cm} 
\end{center}

Intuitively, this question has a ready-made answer. You should believe what \emph{seems} most likely. You should believe what you have most \emph{reasons} for. Believe what the \emph{evidence} suggests. Believe what you \emph{think} \emph{is true}.

But \emph{why} should I believe that? What is the rationale behind these normative requirements? In this paper, I want to develop an account that has a seemingly simple answer to this: You should believe what the evidence suggests because that is probably what is true. True beliefs are good. Evidence is a means to the truth. So listen to it! It turns out, though, that this proposed derivation is a lot more complicated than it seems.

This consequentialist picture of normativity rests on a form of value monism. Truth or true belief is its central and only fundamental value. This position is called \emph{veritism}. It derives norms of belief from the simple assumption that only truth is of basic value. How that might work and which problems this position has will be the focus of this paper.

In the end, I hope to make a case for veritist accounts, if only in that there might be an optimistic outlook. I argue that at least insofar as plausible norms of belief are concerned, veritist positions seem to be in a good spot to deliver a convincing explanation of value and normativity. The account proposed is therefore a veritist consequentialism.

Normative questions are central to epistemology. Some even hold that they take precedence over conceptual analysis. When studying the normativity of epistemology, it is helpful to first take a look at the normative vocabulary long established in practical philosophy. There might be, after all, important analogies between both kinds of normativity. In Chapter 2, I outline the preliminaries for this paper. I introduce some basic concepts of the normative landscape and talk about what they have to do with the position of interest, veritism. I explain what a proper account of epistemic normativity amounts to, and how a consequentialist account of epistemic normativity grounded in veritism might look like.

Veritism is an axiological position with far-reaching consequences for one's epistemological position. In Chapter 3, I detail what veritism spelled out actually means and how a form of veritism can be made plausible with regard to intuitions about epistemic value. As this might be quite important from the outset, I discuss this in some detail to give a solid foundation for the normative epistemology. I highlight to positive aspect of adopting a veritist account. I present a salient objection to the veritist account based just on its axiology, look at defenses in the literature, and expand upon them to propose a veritist response to this challenge.

The axiology out of the way, in Chapter 4, I discuss the deontic or prescriptive epistemic norms in relation to veritism. I first take into consideration whether it is even possible for deontic norms of belief to be true, as has been often and famously challenged. I then discuss how epistemic norms might look like and consider problems for and objections to them. This proves to be a rather complex undertaking, and is not at all a trivial task.

I conclude by stating that there might be good \emph{prima facie} reason to believe that veritism provides an acceptable account of epistemic normativity. The theory has a lot of things going for it, and seems to be able to properly explain epistemic normativity, even if I only show it only a very general way.

Certain epistemological concepts are charged with normative intuitions, especially the concept of justification. It is most probably possible to provide an account of justification compatible with and explained by veritism. In this paper, however, I will set this question aside to focus on actual norms of belief and their relation to veritism. These norms might be closely related to the concept of justification, but I do not concern myself with this putative relationship. For now, I regard both norms and justification as independent concepts. A lot of pushback that veritism receives is due to its explanations of justification. I try to feebly evade this problem by not clearly taking sides.

As a last introductory point, the present paper is concerned with individualistic epistemology, mostly. New and interesting considerations from social epistemology play no direct role.\footnote{As may be obvious, I presuppose a realist notion of truth. As perhaps less obvious, throughout this paper I assume that every proposition is either true or false and not neither or both. I won't in detail discuss complications that arise for the veritist account if one supposes, say, dialetheism.}

\clearpage

\section{Values, Norms and Truth}\label{sec:values}

\begin{quote}
    It is not that hard to be skeptical about value. You can't see values. You can't touch them. And many think we can't even define them. Partly as a result, our disagreements about value have an interminable and intractable feel. Philosophically speaking, skepticism about value seems an easy sell.

    At least this is the case when we are talking about the type of value that most philosophers worry about—moral value. But we have other values besides moral values. And one of the most basic of these other values is the value of truth. \autocite[225]{Lynch2009}
\end{quote}

When talking about values, the deontic or about normativity in general, it is helpful to look for established theories as a guideline. Naturally, these theories have been developed in the realm of practical philosophy, so that's where I'll turn for some clarificatory groundwork.

A good starting point to look for some guidance when talking about values are the fields of value theory and, more specifically, axiology.

\subsection{ Taxonomy of the Normative}

To clarify the notions, I very shortly introduce the normative terminology and how it is used it in this paper.

Normativity is roughly divided into axiology, the study of values, and deontology, the study of rules and norms. Axiology is concerned with what is good or bad, what is desirable or valuable, and what is better or worse. Axiological claims are therefore \emph{evaluative}. Deontology is concerned with what is required, forbidden, permissible, what one one ought to do and so on. Closely related is the term \emph{correctness}. Another closely related but distinct group of terms consists of \emph{responsibility, duty, blame, praiseworthiness} and so. On some theories there is a slight, on others a significant difference between these groups of concepts. Deontic claims are prescriptive or \emph{regulative}. There can be evaluative norms, specifying what is of value, as well as deontic norms, specifying what is to be done or believed. Sometimes deontic norms of the form 'you ought to $\phi$' are meant to be interpreted evaluatively, but I will mostly leave this complication out of the paper.

Any account of normativity aspires to explain the axiological \emph{and} deontological questions. The tension between different types of those theories are of some import for the present paper, and I circle back to this topic frequently.

What \emph{is} a value, then? \textcite[79]{scanlon1998} puts it thusly: “ ... a notion of how it would be best for the world to go, or of what would be best for particular people”. In addition, it is what makes evaluative sentences like “pleasure is good” come out (perhaps) true. There might be something that corresponds to these sentences, for instance a property of pleasure, that could gives“pleasure is good“ a status as a true sentence. Since axiology is defined above in terms of claims of the form “pleasure is good”, this approach to explain the term ‘value’ seems to beg the question. It is such a simple and perhaps primitive notion. What's left is the appeal to intuition–certain things are intuitively good, or are better than others. That's what \emph{value} refers to.

There are many distinctions of value in the literature. Values can be \emph{intrinsic}, \emph{extrinsic}, \emph{fundamental}, \emph{basic}, \emph{final}, \emph{instrumental}, \emph{simpliciter}, \emph{pro tanto}, \emph{all-things-considered}, something can be valuable \emph{for its own sake} or merely \emph{derivatively}. There is some debate as to what the differences are, and which attributions overlap. For the current purposes, I need one distinction especially which I use throughout the paper. This is the difference between \emph{fundamental} value and \emph{instrumental} value. Something is of fundamental value if it is valuable for its own sake, if the value is not somehow derived from something else. Sometimes I also use the term final value to describe the same thing, although some interpret these terms as different in meaning. Something is of instrumental value if it is not valuable for its own sake, and gets it value derivatively from something else that is of fundamental value. I do not mix the terms intrinsic and extrinsic value into this at all.

There is also the idea that something is valuable versus the idea that we regard something as valuable. I presuppose that something can be valuable, and that value is not merely attributed by us. I think that the relevant arguments can be made without this distinction in mind, however, and I do not want to commit myself to a realist notion of values at this point.

So far I have not really been quite firm on the terminology when talking about values and aims or goals. Let us specify that a bit, following \textcite[344f.]{Berker2013-BERETA-2}. What is valuable in itself are not beliefs \emph{per se}, but states of affairs or situations in which someone has a true belief. But the value of that situation only obtains in virtue of the true belief someone has, so it makes still sense to speak of true belief as valuable. Compare that to a goal: a goal in this sense is a situation of ideal or maximized value, something similar to the ideally rational reasoner who comes up frequently in the formal epistemology literature or the situation science is supposed to reach granted indefinite time for research. So the content of a goal describes that situation towards which doing something valuable or having a true belief promotes. Not much hangs on this distinction, but it maybe important to be clear about these notions.

\subsection{ Teleology / Consequentialism} \label{subsec: teleology}
 Accounts of normativity includes the deontic as well, that is, the concepts of \emph{right}, \emph{reason}, \emph{rational}, \emph{ought} and so forth \autocite[21]{sep-value-theory}. Values and deontic norms are two central concepts in normative theory. Which of these two is the basic or fundamental one through which the other concept can be explained or derived is an ongoing debate and roughly divides normative theorists into two camps: consequentialists (or teleologists) and deontologists. 

\emph{Consequentialists} hold that all regulative norms usually phrased with \emph{oughts} and the like obtain in virtue of being directed at a value or goal. In slogan form, they put the good prior to the right. The basic idea is to first ask “what is the best (i.e. most valuable) action, belief, or situation?” and then ask “what do I do to achieve it?” to arrive at what one \emph{ought} to do. „However much consequentialists differ about what the Good consists in, they all agree that the morally right choices are those that increase (either directly or indirectly) the Good.“ \autocite{sep-ethics-deontological}

\emph{Deontologists} hold that all evaluative norms obtain in virtue of being directed by a regulative norm. In slogan form, they put the right prior to the good. They start out with asking “what should I do?” and then determine the most valuable action, belief, or situation on this grounds.

So the direction of explanation is reversed in these types theories, but instances of both types aspire to account for the \emph{good} as well as the \emph{right}, in other words, aspire to present a complete account of normativity \parencite[341]{Berker2013-BERETA-2}. This distinction is, as in most fields of philosophy, not as clear-cut as it might seem, though, and there are theories that neither fall clearly on one side or the other of the divide and may incorporate features of the opposed theory.

This is a very rough sketch of consequentialist and deontological accounts of normativity that leaves most everything to be desired, to be sure, but for my purposes it's enough give a quick overview what this distinction is all about.

\subsection{Epistemic Normativity} \label{subsec: epistemic}

Veritism is a theory about value, so in that sense it determines the questions regarding axiology in a normative framework. To be compatible with veritism, a deontic theory has to accept that the only fundamental epistemic value is truth or true belief. Veritism combined with such a deontic theory amounts to a teleologist or consequentialist account of epistemic normativity in which deontic concepts are explained by appeal to veritism.

The discussion of the normative has been quite general so far, in order to be compatible with ethics as well as epistemology. Veritism is an epistemological position, so what does that mean for normative theories about it? Most of what has been said translates directly into the \emph{epistemic domain}, meaning anything that pertains to concepts like belief, truth, knowledge and justification \autocite{David2001-DAVTAT-7}. Even \textcite[7]{grundmann2008}, who holds that conceptual analysis in epistemology precedes the normative inquiry, emphasizes the normative character of justification, one of \emph{the} central concepts in epistemology \autocite[226]{grundmann2008}. Still more prevalent is the connection of epistemological concepts to the normative in the \emph{deontic conception of epistemic justification}, presented by \textcite{Steup1988-STETDC} and argued against by \textcite{Alston1988-ALSTDC}. The deontic conception states that epistemic justification can be “cashed out” in terms of permissions, obligations and so forth. This goes one step beyond the claim of Grundmann, who merely notes the evaluative normative character of epistemic justification.

So the interesting question is whether veritism can account for and explain the normative character of epistemology. What are those normative concepts, and how can it be done? I see two basic different approaches:

\begin{enumerate}
    \item Determine the exact content of the goal stipulated by veritism. What exactly is the goal of epistemology? By finding out the most plausible version of an exact description of the goal, one thereby acquires a putative description of the ideal epistemic state of a believer. Taking this description as a regulative ideal, one can try and work out the most plausible epistemic regulative norms, that, if followed, brings one closer to being in the ideal epistemic state\footnote{That this is not as easy as it might sound is \textcite{Gibbons2013-GIBTNO} major point. These can come in conflict, in what he calls \emph{the puzzle}. I refer again to this problem in \ref{sec:normsofbelief}}.
    If and how these regulative norms so construed pertain to the concept of justification or knowledge is a different question, and, depending on one's assumptions, can follow directly or not at all.
    \item Stipulate that the normative conception of epistemology just is a feature of what the central epistemic concept, mostly justification or knowledge amounts to. On this approach, what it means to be justified (or knowledgeable) is in some way or other derived from veritism. The value of justification is derived from the value of true belief, but whether “justified belief has anything to do with fulfilling one's epistemic obligations” is left undecided \autocite[66]{Steup1988-STETDC}.

    It is important to note that this is \emph{not} equivalent to a deontic conception of justification. Whereas the deontic conception holds that justification can only be \emph{analyzed} in deontic terms, this account does make this claim. It would, however, not really be a proper concept of epistemic deontic normativity if there wouldn't be some claims made about the connection of justification and deontic norms.
\end{enumerate}
In this paper, as noted in the introduction, I pursue the first strategy. I will be mostly silent on whether a veritist account of justification also provides constitutive means to a concept of justification. Both approaches don't seem to be mutually exclusive.

\clearpage

\section{Veritism}

In this section, I present veritism as a theory of epistemic axiology. I start by delineating veritism and differentiating it from other accounts. I then address the question whether veritism can plausibly account for intuitions about epistemic value of relevant doxastic attitudes. Finally, I present a salient objection to veritist axiology, the challenge from significance. I argue that the veritist can respond convincingly.

\subsection{Veritism and Alternative Accounts}

Veritism proposes that true belief is the only fundamental epistemic value. \textcite[191]{Zagzebski2004-ZAGEVM-2} has a different name for a very similar position to veritism: \emph{epistemic value monism}. I label this position \emph{truth value monism}, however, since epistemic value monism seems to be somewhat of a misnomer for the position that Zagzebski sketches: “Any epistemic value other than the truth of a belief derives from the good of truth.” Although this \emph{is} a case of epistemic value monism, that is, a position that stipulates a single fundamental epistemic value, other positions may also stipulate a single fundamental epistemic value, most naturally knowledge, and thereby classify as epistemic value monism. So the term is ambiguous, hence truth value monism.
Is truth value monism the same as veritism? In section \ref{sec:varieties} I present different renditions of what veritism might amount to, one of these simply stating truth as the fundamental epistemic value. The other versions may still classify as truth value monism, depending on how tight you want to draw the concept, but I find the label \emph{veritism} a bit more appropriate, so that is what I am going to use.

As one might think, veritism is not the only account to explain epistemic normativity. One obvious alternative is to not \emph{only} stipulate truth as the fundamental epistemic value but to also allow other plausible concepts to have non-derivative epistemic value, for example \emph{justification}. Such account would be an instance of \emph{value pluralism}. This has the striking advantage of not running into the value or swamping problem which challenges veritism to explain the seemingly additional value of justification over true belief. In this paper, I will not discuss this problem, however, it is very closely tied to veritism and appears as soon as one develops an account of justification on veritist grounds. Alternatively, other epistemic concepts like \emph{knowledge} could be stipulated as the only concept of fundamental epistemic value. A different angle take non-value-bases accounts on normativity, that posit the precedence of deontic norms as fundamental epistemic normative concepts, and explain value as derivative from norms. I don't compare or elaborate on these alternative conceptions in this paper.

\subsection { Virtues of Veritism }
Terminology out of the way, Let's see what reasons there are to endorse veritism as a theory.

First, it is a most natural notion. It is not controversial at all to say that truth is valuable one way or the other, or that having a true belief about some matter is better than having a false belief about that matter. This much is conceded by pluralist notions as well a other epistemic value monism, such as a knowledge-first account, even if in the latter case true belief would be valuable only derivatively. This much is even conceded by deontic conceptions of epistemic normativity, which grant that there is value to true belief, but that true belief derives it value trough norm-compliance. So this point is not a unique selling point for the veritist position. The concept is so natural in fact, that some even argued that belief constitutively aims at the truth \autocite{Shah2003-SHAHTG,Velleman2000-VELOTA}. To go that far is not necessary for a veritist position, however \autocite[361]{Berker2013-BERETA-2}.

To drive the point home how natural this notion is, I include some quite prominent quotes. Lots of different statements like this can be cited of lots of noted epistemologist, interestingly, from many different camps. Truth as a fundamental epistemic value is held by internalists, externalists, foundationalists, coherentists, evidentialists, reliabilists, and so on.

\begin{quote}
I don't know how to prove that the acquisition, retention, and use of true beliefs about matters that are of interest and/or importance is the most basic and most central goal of cognition. I don't know anything that is more obvious from which it could be derived. But I suggest that anyone can see its obviousness by reflecting on what would happen to human life if we were either without beliefs at all or if our beliefs were all or mostly false. \autocite[30]{Alston2005-ALSBJD}
\end{quote}

\begin{quote}
 I suggest that the primary function of cognition in human life is to acquire true rather than false beliefs about matters that are of interest or importance to us. \autocite[29]{Alston2005-ALSBJD}
\end{quote}

\begin{quote}
What makes us cognitive beings at all is our capacity for belief, and the goal of our distinctively cognitive endeavors is truth: we want our beliefs to correctly and accurately depict the world. \autocite[7]{Bonjour1985}
\end{quote}

Secondly, veritism packs a bunch of explanatory power. To see this, consider a successful theory of justification in traditional epistemology: process reliabilism. This theory is, as Goldman point out, consequentialist in nature \autocite{Goldman2002-GOLTUO-2}, employing veritism as axiological groundwork. In formal epistemology, too, veritism has explanatory power and can help vindicate bayesianism, or so at least \textcite{Joyce2009-JOYAAC} proposes. Trying to vindicate probabilistic coherence has long mostly been a project taking its argumentative force from pragmatic considerations such as intuitions about pragmatic betting behavior. Setting it on alethic foundations, as Joyce puts it, is to present a purely epistemic argument for probabilistic coherence. He then goes on to assume accuracy of degrees of belief as the fundamental epistemic value, which is quite strikingly a veritist variant. This point is not of \emph{that} much relevance for the present purposes, since I do not want to develop a full account of epistemic justification. Nonetheless, when considering the general virtues of veritism it is relevant.

Thirdly, it may be argued that parsimony is a veritable virtue of explanatory theories \footnote{In fact, \textcite[342]{Ahlstrom-Vij2013} posit some key desiderata for a satisficing epistemic axiology, chiefly among them a requirement for parsimony. They argue for it on the grounds that “one should prefer ontologies with fewer rather than more existential commitments, \emph{ceteris paribus}” \autocite{Ahlstrom-Vij2013}. This means that when comparing two theories with similar attractive features then, other things being equal, one ought to choose the theory with the fewer different entities assumed existing as possible. This intuitive principle in philosophy and the sciences was already proposed by Aristotle, Kant, Newton, Einstein, and Lewis among many \autocite[3]{sep-simplicity} and has its most famous philosophical realization in Occam's Razor, which can be stated as “other things being equal, if $T_1$ is more ontologically parsimonious than $T_2$ then it is rational to prefer $T_1$ to $T_2$” \autocite[7]{sep-simplicity}. Arguments for it range from just stating that parsimony is a primitive value over analyzing the quality of actual empirical theories vis-à-vis parsimony to probabilistic arguments showing that rational agents assign distinctive prior probabilities to simpler laws \autocite[11-26]{sep-simplicity}. For the purposes of this paper I take it to be sufficient to assume this quite plausible principle without going any deeper into this discussion.},
and stating just one single epistemic value to explain all of epistemic normativity is of course quite parsimonious.

Forthly, stating a single epistemic value is one way of avoiding the incommensurability problem \autocite[16]{sep-value-theory}. Mainly pluralistic accounts of epistemic value have the problem to explain how, and why, these different types of value can be compared. Not so with veritism, as it is a lot more straightforward to explain comparison in values when there is only one type of fundamental value to refer to.

So it seems like veritism has a lot going for it. I'll go in more detail spelling out a veritistic axiological account.

\subsection{Varieties of Veritism}\label{sec:varieties}

\textcite[54]{Goldman2002-GOLTUO-2} defines veritism as  “... the unity if epistemic virtues in which the cardinal value, or underlying motif, is something like true, or accurate, belief”. Borrowing the terminology from Goldman, \textcite[360]{Berker2013-BERETA-2} proposes to define veritism as the position that „ ... our only epistemic goals are (i) the accumulation of true beliefs and (ii) the avoidance of false beliefs”. \textcite{Zagzebski2004-ZAGEVM-2} has a different name for a very similar position, what she calls epistemic value monism: “Any epistemic value other than the truth of a belief derives from the good of truth“. 

The unifying theme of these quotes is apparent: What’s central to all three accounts clearly is the notion of \emph{true belief}\footnote{It might be argued that \emph{truth} is the actual only epistemic value, assigning it to true belief would be “undue reification” \autocite{Pritchard2014}. Then, however, one has to tell a story as to why other truth-apt doxastic attitudes do not have epistemic value. This could be done by pointing out that belief is the only relevant epistemic doxastic attitude. And this would be not much different than to be clear about it and state true belief as of fundamental epistemic value.}. True or accurate belief has to be a value, goal, function or motif with a distinct role in the theory. This role is determined as cardinal, single, primary and non-derivative. In other words, veritism says that true belief has to be regarded as the only \emph{fundamental} value. No other epistemic concept like knowledge or justification has fundamental epistemic value. It is important to note, however, that this does \emph{not} entail that other epistemic concepts have \emph{no} epistemic value, they just don’t have \emph{fundamental} epistemic value. So it is perfectly compatible with veritistic accounts to say that epistemic justification is valuable and that it has a value distinct from the value of true belief. Justification does not, however, have fundamental epistemic value.

\subsubsection{The Value of Believing}
That is the gist of the position: veritism claims that true belief is the only epistemic value or goal.

\begin{description}
    \item[Simple Veritism:] True belief is the only state of affairs that has fundamental epistemic value.
\end{description}

It is still unclear how this amounts to a comparative notion. For now, all that follows from this is that for a proposition $p$, if $p$ is true, it is more epistemically valuable to believe $p$ than not to believe it. One may say that one accumulates epistemic value by adopting more true beliefs. It is epistemically better to have true beliefs than to have false ones. I leave open for now how exactly this might spell out.

Can this account explain intuition about scenarios concerning epistemic value?  A clear-cut case with two propositions $p$ and $q$, where $p$ is true and $q$ is false, couldn't be more obvious: believing $p$ is more valuable than believing $q$. This very natural thought indicates a comparative notion which holds that true belief is more valuable or better than false belief. Simple enough.

Now what if we want to evaluate the doxastic system of a person, or her intellectual attainment, as \textcite[58]{Goldman2002-GOLTUO-2} puts it? This seems just as straightforward. The more true beliefs a person has, the more valuable the position that she occupies. So no matter the epistemic situation she is in, by acquiring more true beliefs she can better her epistemic standing. So far, so good.

All is not well, though. Consider these two:
\begin{description}
    \item \emph{Gullible Gilbert} the believes everything anyone tells tells him, anything he thinks about, in general any proposition he encounters.
    \item \emph{Precise Priscilla} the is more careful in forming her beliefs. She carefully weighs her evidence and, as a result, has a lot fewer beliefs than Gilbert, but her beliefs are largely true.
\end{description}

Gilbert’s situation is an instance of what \textcite[360]{Berker2013-BERETA-2} calls \emph{epistemic recklessness}.

Intuitively\footnote{I make a lot of assumptions about intuitions here. Granted, this is not ideal, but I tried to only incorporate cases where the intuitions seem uncontroversial, or if the intuitions seemed problematic, noted that they might be so and tried not to put too much argumentative weight on them. As these are empirical claims, of course all of my assumptions about intuitions are open to challenges from experimental philosophy.},
it is obvious that Priscilla is in the more valuable epistemic situation. Yet on the account just sketched, Gilbert amasses lots and lots of beliefs, among these many true beliefs. Hence his epistemic situation is comparably quite valuable. Priscilla on the other hand has fewer true beliefs to show for, her epistemic situation is therefore less valuable. This is not a good result for the simple account.

To avoid this result, it seems only natural to expand the epistemic goal to consist of to goals, really: ”...the twin goals of acquiring true beliefs and avoiding false ones“ \textcite[339]{Berker2013-BERETA-2}. This dualistic rendition of epistemic value has already somewhat been proposed by \textcite[17]{James1896-JAMTWT-19}, who coined the phrase: ”Believe truth! Shun error!“.

\begin{description}
    \item[Dualistic Veritism:] True belief and the avoidance of error are the only two states of affairs that have final epistemic value.
\end{description}

What does it mean to avoid error? An error in this sense is a false belief \autocite[362]{Berker2013-BERETA-2}. But in everyday language, disbelieves in true propositions are errors as well. What is avoidance, then? In a narrow reading, only disbelieving a false proposition count as avoiding errors. Interpreted more loosely, suspending judgment or just being ignorant about a given proposition can count as avoidance, too. This relates to the criterion that determines which propositions actually contribute to the epistemic value, which I hint at later in this section.

So to maximize my epistemic value, I aim to have as many true beliefs and as few false beliefs as possible.

This account fares better with respect to Gilbert and Priscilla. As Gilbert does have some true beliefs that count towards a valuable position, Priscilla’s carefulness now brims with epistemic value as she avoids most of the errors that Gilbert makes. So this intuition, at least, can be explained.

But let's revisit one of the key merits of veritism at the core of any deontic framework. One central motivation is to unify all epistemic evaluation. This approach reaches as far back as Socrates, who proclaimed ”virtue is one” and “meant it quite literally” \autocite{penner1973}, entailing that what Socrates regarded as virtues where really all the same thing. So bravery, wisdom, temperance, justice, piety and even knowledge amount to something equivalent, if not identical.
\textcite{Goldman2002-GOLTUO-2} picks up this point in his search to unify all epistemic virtues\footnote{Now, virtues are not the same thing as values or goals, and I don't want to get into a discussion about virtue epistemology at all here. What Goldman concedes, though, is that his conception of epistemic virtues builds upon or at the very least is compatible with that form of consequentialism value-directed accounts present us with.}. As one might think, he is not so quick to drop value monism to switch to a dualistic account, which would pretty much mean to give up the idea of a single unifying epistemic virtue.

Instead, \textcite[58]{Goldman2002-GOLTUO-2} proposes an a little more complex form of veritism which takes its motivation from the model of partial belief, or degree of belief, or levels of confidence, or credences. Let's first get clear on the different notions of belief he employs:

\begin{quote}
    First, we can use the traditional classification scheme which offers three types of credal attitude toward a proposition: believe it, reject it (disbelieve it), or withhold judgment. I call this the trichotomous approach. Second, we can allow infinitely many degrees or strengths of belief, represented by any point in the unit interval (from zero to one). I call this scheme the degree of belief (DB) scheme. \autocite[88]{Goldman1999-GOLKIA}
\end{quote}

So nothing non-standard here. Note that we want to account for each of the doxastic attitudes belief, disbelief and suspension of judgment.

In the model of degree of beliefs, it is not mere true belief that has final epistemic value. Instead, the value of a credence in a true proposition derives its value as a function of its degree: maximal value if the degree is maximal, and minimal value if the degree is minimal. This yields a slightly different comparative notion than before: A degree of belief in a true proposition is more valuable than another degree of belief simply if it is higher. Given a true proposition $p$, your credence of $0.9$ is more valuable than mine of $0.6$, simple as that.
Goldman presupposes a viable way to translate degrees of beliefs into full beliefs, suspension of belief, and disbelief, by some threshold measure that is left unspecified. That means that for a sufficiently high degree of belief over some threshold, this can be translated to full belief, and a sufficiently low degree of belief with a mirrored threshold can translate into disbelief. Suspension of judgment is then something intermediary, without clear specification as to exactly which degrees of belief correspond to it.

He can then compare: ”... believing a truth carries more veritistic value than suspension of judgment; and suspension of judgment carries more veritistic value than disbelief“. This leads to a comparative notion:

\begin{quote}
    If a person regularly has a high level of belief in the true propositions she considers or takes an interest in, then she qualifies as “well‐informed.” Someone with intermediate levels of belief on many such questions, amounting to “no opinion,” qualifies as uninformed, or ignorant. And someone who has very low levels of belief for true propositions—or, equivalently, high levels of belief for false propositions—is seriously misinformed. \autocite[58]{Goldman2002-GOLTUO-2}
\end{quote}

In this way, Goldman concludes, the veritistic account can accommodate everything an account with two epistemic goals instead of one. In particular, he claims that having a very low degree of belief for true propositions is \emph{equivalent} to having a high degree of belief in false propositions, at least insofar as the epistemic value is concerned. On this account, a credence of $0.2$ in a true proposition has the same epistemic value as a credence of $0.8$ in a false proposition, and both counts equally towards a person being seriously misinformed. For Goldman, this notion of comparative value does justice to our intuitions regarding relevant cases. It is very reminiscent of the concept of utility in utilitarianism, which is “sum of the action's positive and negative consequences, that is, the pain and pleasure caused by the action” \autocite{depaul_value_2001}, in that it subsumes the positive value of true belief and the negative value of false belief under one unifying concept.

This leads to this alternative formulation:
\begin{description}
    \item[Accuracy Veritism:] An accurate doxastic attitude is the only state of affairs of final epistemic value\footnote{This is a formulation very close to \textcite[9]{Pettigrew2016-PETAAT-7}. I adopted the phrasing as ‘accuracy’ to incorporate Goldman's approach to the value of full belief, disbelief and suspension of judgment.}.
\end{description}

\subsubsection{The Value of Disbelieving}

However, as I see it, there are problems with this translation of conclusions drawn with the notion of degree of belief to the notion of full belief that Goldman does not seem to take into account. It might be interesting to have a look at the notion of belief and disbelieve that Goldman employs, as it does a lot of work in explaining the value of the doxastic attitudes.

To be clear, I still argue mainly \emph{with} Goldman in the end, but I want to take a closer look at the argument supporting the position.

\textcite[58]{Goldman2002-GOLTUO-2} claims that disbelieving a proposition $p$ is equivalent to believing a proposition non-$p$, so that I disbelieve a proposition $p$ if and only if I believe a proposition non-$p$. That is, whenever it is true that I disbelieve $p$, it is also true that I believe non-$p$, and whenever it is false that I disbelieve $p$, it is also false that I believe non-$p$, and \emph{vice versa}. 
I assume this is a consequence from the translation of the notion of degree of beliefs to the notion of full beliefs. Given scaling of degree of beliefs on the unit interval, whenever I believe $p$ to the degree $c$, I also believe non-$p$ to the degree $1-c$. For example: I am pretty unconvinced of $p$, my credence is $0.1$. Hence, my credence in non-$p$ is $0.9$. Translated back to full beliefs, this would then amount to belief and disbelieve, if $c$ is sufficiently low for disbelief. 
So Goldman's claim about equivalence would follow. What shouldn't be disregarded, though, is that this only applies to agents who obey the axioms of probability theory, in many theories a necessary condition on rationality of degree of beliefs. It is still very much conceptually possible to believe $p$ to the degree $0.8$ \emph{and also} believe non-$p$ to the degree, say $0.7$. I wouldn't be quite \emph{rational} to do so, of course, but it is certainly \emph{possible}. Now, Goldman does not state conceptual identity, granted, though equivalence is still a very strong claim. But, my believing $p$ to the degree $c$ \emph{does not entail} my believing non-$p$ to the degree $1-c$ (or vice versa). 
So, translated back to full beliefs, the equivalence of disbelieving that $p$ and believing that non-$p$ does not follow from the notion of degree of beliefs\footnote{The notion of translating between different models of belief formation is not self-explanatory and seems in need of an independent argument. Let's suppose, for the moment, that such a sufficient argument has been presented.}.

One could object to this that this is not at all a move from degree of beliefs to full beliefs. Instead, it just very naturally follows from thinking about doxastic attitudes. What else should disbelieving $p$ amount to, if not believing that non-$p$? When I disbelieve $p$, I hold that $p$ is false. But believing that $p$ is false \emph{just is} believing that non-$p$ is true. The argument looks something like this:

\begin{enumerate}
\item [P1] If I disbelieve that $p$, I believe that <$p$ is false>. 
\item [P2] <$p$ is false> is equivalent to <non-$p$ is true>.
\item [P3] Belief is closed under single premise logical entailment.
\item [C1] If I disbelieve that $p$, I believe that <non-$p$ is true>. (From P1 to P3)
\item [P4] Believing that <$q$ is true> \emph{just is} to believe that $q$.
\item [C] Therefore, if I disbelieve that $p$ then I believe that believe that non-$p$. (From C1 and P4)
\end{enumerate}

A similar argument may be sketched for the opposite direction of entailment.

The argument seems perfectly valid. But is it sound? I would hold that I presumably can disbelieve $p$ and also not believe non-$p$, that is, disbelieve non-$p$ or suspend judgment towards non-$p$. There is no entailment relation between either of these doxastic attitudes. To take a familiar example: I can disbelieve that Hesperus shines without believing that Phosphorus doesn't shine, even though, as is common knowledge, Hesperus and Phosphorus denote the same object and therefore <'Hesperus shines' is false> and <'Phosphorus does not shine' is true> are equivalent propositions. What goes wrong in the argument? P1 may just follow from your definition of disbelieve. I won't argue against that here, or ever, so this premise seems fine. P2 just is a necessary statement given that every proposition is either true or false. P4 seems like a truism, although one might argue against it on grounds that a conception of truth or true belief is not necessary for a subject to hold true beliefs, however, let's grant this as well. P3, on the other hand, is the most dubious candidate. While it is a hot topic whether knowledge is closed under entailment\footnote{See, for example, \textcite{Dretske2005-DREIKC}.}, and even so for rational belief\footnote{See \textcite{KyburgJr1970-KYBC-2} for a case against.}, it is a lot less plausible that \emph{mere belief} is closed under (single premise) entailment. For I can, of course, believe that $p$, where it is some necessary truth that If $p$, then $q$, but still fail to grasp that $q$ obtains. Without this or a similar premise\footnote{Perhaps the premise that it is closed under \emph{obvious} single premise closure, but then, I think, the same counterexamples work. But isn't closure under logical entailment too strong a requirement? What's really at issue is closure under logical equivalence, and that seems a lot more plausible? Well, even in this case, the presented counterexample about Hesperus and Phosphorus would work.} the argument just isn't sound.

So, I suggest that yes, of course, <$p$ is false> is equivalent to <non-$p$ is true>, however, disbelieving that $p$ is not equivalent to believing that non-$p$.

Goldman can still retreat to the position that the \emph{epistemic value} of disbelieving $p$ is equal to the \emph{epistemic value} of believing non-$p$. Let's suppose $p$ is false, and hence non-$p$ true, then on this account, disbelieving $p$ has epistemic value–the corresponding credence is very close to the actual truth value–and believing non-$p$ has value, since one believes a true proposition, namely non-$p$. So what can be accounted for is the dual value of disbelieving a falsehood and believing a truth. This position seems a lot more plausible, then. Perhaps it is what Goldman had in mind, anyway. What does it help veritism with? Consider the case:

\begin{description}
   \item \emph{Skeptical Stephanie} is in a very unfortunate situation. Her questionable colleagues all present her with unconvincing propositions. As a result, she comes to disbelieve most of them. And rightly so, all of them are false.
\end{description}

We intuitively want to say that Stephanie's disbeliefs contribute something epistemically valuable to her situation. Yet, on the simple account of veritism, I could only ascribe value to her disbeliefs if I presume the equivalence stated by Goldman, that her disbelief in the false proposition $p$ entails a believe in the true proposition non-$p$. On the dualistic account, her disbeliefs would count as an avoidance of error only if (a) the equivalence claim above is supposed true, or (b) error is meant to encompass disbelieve in false propositions as well. Option (a) would presuppose too much to be incorporated into the value directly, for my taste, whereas option (b) seems \emph{prima facie} to be a plausible candidate to deal with this problem, however, this is not how the term is usually used.

Goldman's accuracy account is partly designed to handle exactly this type of case. Stephanie's disbeliefs all have epistemic value equal to that of true beliefs. Hence, her epistemic situation is valuable, agreeing with intuition.

\subsubsection{The Value of Suspending Judgment}

What about suspension of judgment? Given a true proposition $p$, it is most valuable to believe that $p$. It seems less valuable to suspend judgment towards $p$, and it is even less valuable to disbelieve $p$. Conversely in the case of a false proposition $q$: disbelieving is most valuable, suspension less so, and believing that $q$ is least valuable. Intuitively then, there is a clear ranking of these types of doxastic attitudes. On the simple account, none of the features of the ranking save true belief can be explained. The dualistic account with the above modification regarding error gets two out of three right, it does not say anything about suspension of judgment though. What should it count as? Is it some form of true-at-least-a-bit belief, or should it be regarded as an error? Neither of these seem plausible. So if we are to include suspension of judgment into our notion of doxastic attitudes, both the simple and the dualistic veritistic account can't seem to handle the intuitions regarding its value. The accuracy veritist is able to cope a lot better, as suspension is directly incorporated into the comparative notion that Goldman spelled out.

So far, it seems that accuracy veritism comes out on top, at least if we only consider cases in which there is a fixed set of propositions against which we evaluate different intellectual attainments. Consider again the case of Gilbert and Priscilla: Gilbert is right quite often, but wrong even more often. Priscilla is right in most of her beliefs, but has fewer of them, so is not right as often as Gilbert is. On Goldman's account then, one accumulates epistemic value by believing truths and disbelieving falsehoods, but also by suspending judgment, regardless of the truth value of the proposition. Isn't it still the case that Gilbert accumulates more value by believing all those true propositions? He accumulates a lot, that much is correct, but reading Goldman charitably we might ascribe to Priscilla that she withholds judgments in many of the cases which Gilbert gets wrong. Thus, she accumulates lots of additional value by being indecisive. I would concede that this point is controversial, for it is not the same thing to withhold judgment and to be ignorant about a proposition. Being ignorant does not promote any epistemic value, and isn't Priscilla ignorant about those proposition, really? Whatever the answer, one might object that there is a slight alteration to the Gilbert and Priscilla case that might pose a new problem. Consider:

\begin{description}
    \item \emph{Hesitant Howard} is too indecisive to belief most anything, he withholds judgment on every single proposition he is not perfectly sure of. And he is almost never sure of anything! As a result, most (if not all) of his doxastic attitudes are suspensions of judgment.
\end{description}

Of course, Howard is not in an epistemically notably valuable position. And his situation certainly is not better than that of Priscilla. But doesn't the accuracy account produce the verdict that Howard's situation is more valuable, given that he accumulates value for each suspension of judgment, while Priscilla doesn't? The response is analogous to the previous case. If Priscilla is ignorant or without doxastic attitude towards most of the propositions that Howard withholds judgment on, the accuracy account might produce the verdict that Howard accumulates more epistemic value than Priscilla does. If we grant, however, that Priscilla herself suspends judgment on these propositions, then her overall positive track-record will give her a slight to large edge over Howard and therefore make the account's verdict agree with intuition.

If we stipulate, though, that Priscilla has doxastic attitudes towards a given limited set of propositions, but which are mostly accurate, whereas Howard's set of proposition he withholds judgment on is significantly larger, the accuracy account would value Howard's situation as the more epistemically valuable. I argue that these cases in which the set of propositions in question differ in such a significant way may yield less clear intuitions about the epistemic value. Intuitively, there might be something of value in the mere range of propositions that Howard is familiar with, and even though he is not knowledgeable about most of them, just having this many doxastic attitudes may count some way towards epistemic value.

I would propose to set aside such edge cases and predominantly compare epistemic situations on opinionated, fixed or comparably-sized sets of propositions–doxastic attitudes are, after all, what we want to compare the value of.

To make headway, I propose to accept the account of accuracy veritism for the time being as a predominantly plausible one. To avoid complications, I still talk of veritism as employing true belief as the only final epistemic value with the notion of accuracy in mind.

\subsection{The Significance Problem}

In this section I discuss that there might be a problem posed to value-based accounts of epistemic normativity by the clear intuition that some true beliefs seem intuitively more epistemically valuable than others.

Let's compare these two sentences:
\begin{enumerate}
    \item The universe is expanding at an accelerating rate.
    \item The number of people ever to have visited the David Hume memorial up until now is odd.\footnote{Examples from \textcite{Ahlstrom-Vij2013}}
\end{enumerate}

Intuitively, truly believing (1) seems a lot more epistemically valuable than believing (2). \textcite{sep-knowledge-value} put it like this:

\begin{quote}
    Moreover, some true beliefs are beliefs in trivial matters, and in these cases it isn't at all clear why we should value such beliefs at all. Imagine someone who, for no good reason, concerns herself with measuring each grain of sand on a beach, or someone who, even while being unable to operate a telephone, concerns herself with remembering every entry in a foreign phonebook. Such a person would thereby gain lots of true beliefs but, crucially, one would regard such truth-gaining activity as rather pointless. After all, these true beliefs do not seem to serve any valuable purpose, and so do not appear to have any instrumental value (or, at the very least, what instrumental value these beliefs have is vanishingly small).
\end{quote}

The argument against veritism then looks something like this:

\begin{enumerate}
    \item[P1] Intuitively, there are cases of two true beliefs with different epistemic value.
    \item[P2] If veritism is true, then two true beliefs have equal epistemic value.
    \item[C] Hence, intuitively, veritism is false.\footnote{This argument is somewhat similar to the one presented by \textcite{Hu2016-HUWDT}.}
\end{enumerate}

Why is it that some beliefs seem more epistemically valuable than others, and how could veritism account for these intuitions?

For most people, believing a proposition is valuable only if it is of interest, be it for day-to-day life or for other purposes. Hence, concerning oneself for no good reason remembering something trivial like each entry in a phone book seems like a utterly point- and valueless undertaking, even though this add lots of true beliefs to ones overall stock.

\subsubsection{Subjective Interests vs. Objective Significance}
A belief could be more or less relevant to one's own purposes. In that sense, my own interest determines what is significant to me. Whether the number of people ever to have visited the David Hume memorial up until now is even or odd may not be of interest to most people, but if I have a bet running on the outcome of (2) I am very interested in the result, as \textcite[333]{Ahlstrom-Vij2013} points out. But in this case, I wouldn't value this belief for its own sake but for something other besides its epistemic properties, valuing the truth of the belief only instrumentally. That is not an epistemic source of value, however. And we are looking for some kind of epistemic significance. So this type of significance cannot explain the intuition in purely epistemic terms.

A different approach is to grant genuine purely epistemic interests to people. Someone might be interested really just in the truth of a proposition. If we were to include this into the notion of epistemic value, the epistemic value of a belief would differ from person to person, the epistemic value is relative to a person's interests. This might not necessarily be a bad thing on the conceptual level. Consider: The ascription of value to a belief itself is subject-dependent since there are no beliefs with no one there holding those beliefs. But wouldn't we want to say that a person believing (2) does so in an epistemically worthless manner, even if there are genuine epistemic interests involved? Or granting some epistemic value, if small and obscure however, wouldn't we want to say that to believe (1) would still be more epistemically  valuable than (2)? If this is true, then this kind of subject-relative notion of epistemically relevant significance can't explain our intuitions about it. But still, sometimes we do evaluate a single true belief relative to the epistemic situation of the believer. For example, there seems to be a difference in value between a mere random true belief and a belief that has been acquired through meticulous research. However, this does not pertain to the \emph{significance} of the belief in question.

Perhaps some kind of majority-rule could deliver the desired verdict. It could be the case that what is epistemically significant is what most people deem interesting. I don't think this idea worthwhile to pursue, since it would be very strange building a notion epistemic normativity on a seemingly contingent empirical fact about majorities of people, if that is a fact, anyway.

But perhaps what is really meant here is what \emph{should} be of interest to the subject. So it may be the case that one is very interested in the outcome of the bet on (2), but really, what one epistemically should be interested in, is (1). This does not help the veritist one bit, though, as it is her aim to explain epistemic normativity. Presupposing that contingent normative requirements actually determine epistemic value, from which epistemic normativity is then argued for, would just be circular reasoning. It seems, then, that subjective interests regarding epistemic value either do not explain our intuitions or do no work for a normative theory.

What about some form of objective interest or significance? It is not quite clear what that would actually mean. We might assume some kind of \emph{natural curiosity} as a contingent fact about human beings that entails some form or other of a desire for relevant truths. This could be cashed out in purely epistemic terms \autocite[333]{Ahlstrom-Vij2013}. So somehow the relevance of the truth in question is something ingrained in human nature, and we as believers can't help to value some truths more than others. But there is no normative statement of the form “only relevant beliefs are correct” or the like, at least not in our veritist theory so far. To assume something like that has two consequences: on the one hand, it would mean to give up explaining where the significance comes from and instead just stipulate significance or interest on the grounds of accordance with intuition. On the other hand, this seems like a departure from value monism: Apart from truth, there is an additional fundamental value, significance. Both options don't seem perfectly viable for the veritist.

\textcite[61]{Goldman2002-GOLTUO-2} opts for just stipulating significance in the form of interest: „Let us just say that the core epistemic value is a high degree of truth-possession on topics of interest“. He agrees that this make the core value compound or complex in some manner, but then states that this does not challenge the thematic unity of his virtue theory. The latter part is sound, given that Goldman is in search of some „weak thematic unity“ of values that underpins his virtue-based account of epistemic normativity. But for the committed veritist this is not a serious option in my opinion. The only ontological commitment that the veritist is prepared to make in the realm of epistemic values is true, or accurate, belief, and not some additional qualification like not properly motivated interest.

\textcite[332]{Ahlstrom-Vij2013} first argue that intuitions about significance are trivially compatible with veritism, we just need to emphasize the point that not all true beliefs but only true beliefs are epistemically valuable, incorporating the possibility that some true beliefs are of no value at all. This commits to holding that believing (2) is not epistemically valuable, whereas believing (1) is, and thus accounts for the difference in intuition. This leaves open the question of what it is that distinguishes worthless types of belief from valuable ones. To address this, \textcite[334]{Ahlstrom-Vij2013} suggest that “significance measures the degree of epistemic value as a function of the extent to which the relevant true beliefs speak to inquiries that we deem worthwhile, either on practical grounds or on account of intellectual curiosity.” They don't stipulate an additional epistemic value, but instead give a belief's significance the status of a property of belief, that somehow connects true beliefs with their assigned epistemic value. They make it explicit that this is not an existential commitment as this is not a property in the ontological sense but instead the measure by which we, as human beings, value true beliefs. So it is in virtue of significance that values come in degrees. This would be quite convincing, in my eyes, were it not the practical grounds that they grant to significance. As outlined above, this does not seem to add to epistemic value at all. Additionally, it still does not \emph{explain} significance other than through being a primitive property of human nature or of ideal inquiry or give any kind of criterion as to how this might be determined.

\subsubsection{Proposal for a Veritistic Response}
Both accounts are somewhat useful. However, below I argue that considerations of practical value do not play a role in epistemic evaluation. Furthermore, maybe there is a account of truth-directed value that takes into account the significance of belief as an objective notion. I address how this might avoid diluting veritism to a pluralistic position.

What is the exact distinction between values in general and merely epistemic values? Epistemic values pertain to epistemic concepts like knowledge, justification and truth. But is that a division of values that is consistent with “reality” or just some conventional means to divide and conquer the problem? Speaking from an evolutionary standpoint, it seems likely that we as humans are fitted with a natural goal of believing what maximizes our chances of survival. In most (if not all) cases, this goal coincides with believing what is true. But conceptually, these two are not the same. There may be cases where it would be best from a survival perspective, to believe that there is a predator in the surroundings, even when there is not, and so to believe falsely. So to purport the actual, final value of beliefs as being true seems, if not fat-fetched, at least somewhat \emph{artificial}. It is a very useful distinction, however, as the epistemologist can free her reasoning from any practical considerations that would otherwise be still on the table. However, if epistemic value \emph{is} artificial it might also not be what guides our intuitions about thought experiments. These intuitions may not share this artificial division of goals and instead subsume epistemic evaluation under some, probably more practical, goals. I propose that these or some other not merely epistemic intuitions are what guides our evaluation in some cases. Sentence (2) maybe considered of no or low value mainly because it is of no general practical value.

This is to argue against P1: it is intuitively not clear whether the intuitions are about epistemic value. Some cases can be explained by appeal to the mix-up of practical and epistemic value in intuitions.

In the following, I also argue against P2, and present a putative explanation suitable for the veritist, so that the veritist \emph{can} account for two beliefs not having the same epistemic value.

From a different perspective, there might actually be an objective difference in significance in the purely epistemic sense. So it is not that I would give up on \emph{all} intuitions about different epistemic values of beliefs. And this may actually be the case for lots of cases, not just some. This notion of significance starts with observing what we actually mean when we compare values of beliefs. A straightforward approach would count each true belief. The higher the number, the higher the value. This is pretty naive and of course fails to do justice to the intuitions on (1) and (2). But it fails even earlier. Considering the meaning of (2), what is expressed is actually a number of propositions. That there exists a memorial, that memorial is dedicated to David Hume, that people visited it, that the number of people that visited it is odd. Now believing all that amounts to four beliefs, so it would be four times as valuable. Now, I stated earlier, that believing $p$ is not necessarily equivalent to believing some proposition equivalent to $p$, but the \emph{values} are equal to one another. So just in grasping what the actual contents of the belief are, the belief's value is raised. This might point to the idea that the value of a belief is reduced to the accumulated value of the atomic propositions that it contains. Hence, is it necessary to provide a complete logical analysis to ascertain the value of a belief. That does not seem like a desirable position to take. However, I merely want to offer a view of the direction where this could lead, and what its advantages would be\footnote{The only papers so far that I found proposing a similar approach are \textcite{Treanor2014-TRETTA} and \textcite{Pritchard2014} picking up Treanor's idea. Treanor proposes that believing something like (1) is just believing \emph{more truths} and hence is more valuable. He claims that for the objection to rebut teleological accounts of epistemic normativity it would have to present two sentences that contain a very similar “amount” of truth, and still generate the needed difference in intuition about their epistemic value. See footnote \ref{foot:noblackholes} for a possible counterexample. However, he also states that this method of quantifying the significance of a belief is somewhat elusive. It is easy to see that the vague nature of language makes this approach difficult.}.

Compare the sentences: “All ravens are black“ and „this raven is black“. Which is more valuable to belief truly? Apparently, the general statement is. Since it says a lot more, believing it is in value equal to a lot more true beliefs than the singular sentence is\footnote{\label{foot:noblackholes}This of curse leaves much to be desired. First, it is not uncontroversial what the actual contents of a belief are. Second, consider a negative statement like “there are no unicorns“. What is the actual truthmaker in the world? Then consider: “There are no black holes”. Doesn't the second seem a lot more epistemically valuable, if true? How could that be explained? A proper account would have to deal with these kinds of issues as well, I think.}.
This would account for why we value (1) more than (2): To grasp what it means that the universe is expanding at an accelerating rate is to have a working knowledge–or at least idea–of what the universe consists of, that things and concepts like stars, planets, velocity, gravity and general relativity have something to do with it and so forth. Hence, more true beliefs.

Consider this obvious objection: If true belief is the only valuable thing, then can't I just raise the value of my epistemic situation by taking all my true beliefs, and then believe all permutations of conjunctions of them, and thus increase the number of my true beliefs manifold? Well, on this account, since each of these conjunctions derives their value in full only from the value of their true conjuncts, they don't provide any additional value.

What about someone who doesn't understand\footnote{Unterstanding a belief in this sense is just grasping what the contents of a belief are, not in any more rich or complex sense.} the meaning of a sentence? Say I just heard (1), and now truly believe it, but do not understand what that actually means, entails, or coheres or doesn't cohere with. Would we still be inclined to say that this person is in an epistemically valuable position? I don't think so. 

One could object to this that the significance of a belief is now again relative to a subject's own peculiarities, this time the epistemic value is dependent not on the subject's personal interests but on the subject's level of understanding of the target belief. This looks like it would lead to a subjective conception of epistemic value, and as argued above I don't think that desirable. All is not lost, however. Consider the description of accuracy veritism in section \ref{sec:varieties}. Here, the amount of value a subject accumulates per belief depends on the accuracy of that belief. But is the concept of epistemic value thus subject-relative? It is not, I propose, if we take what it means to completely grasp a belief to be constituted in full by what the content of the belief \emph{actually} is fully realized by. So a claim about the significance of a belief would be a claim about the the richness of its content, about how much it has to say. The actual degree to which this significance is appreciated by the subject then depends on its level of understanding of the content of the belief. The understanding of the belief is a means to more true beliefs. This explains why understanding is not valuable for its own sake in this sense, instead, understanding is only instrumentally valuable to arrive at more true beliefs. Given a full understanding of the contents of a true belief, the subject fully realizes the belief's value. In other words, the significance of a true belief is dependent on the beliefs content and influences its \emph{potential} epistemic value. The degree to which a subject understands the belief influences the \emph{actual} epistemic value of a belief. So this does not pose a threat to value monism by incorporating understanding as a fundamental epistemic value. It is, in fact, still compatible, since understanding in this sense only has instrumental epistemic value.

I argue that in problematic cases concerning the intuition about values of beliefs with different significances, the person stating her intuition has enough grasp of the given propositions to notice a profound difference in significance to come the conclusion that one belief is more epistemically valuable than the other. So we don't need to assume perfect understanding to ascribe the ability to make judgments about it but just enough understanding to generate a clear sense for the comparative difference.

Of course, this approach in the current form is naive at best. How that should exactly work is beyond unclear at this point. But I think this approach lacks some flaws of the other accounts presented and is sound enough to explain the problematic intuitions about (1) and (2).

To sum up, I argue against this objection on the grounds that either (i) the intuition about differing epistemic value can be explained in terms of \emph{confusing practical with epistemic value} and (ii) in cases where this does not work, or in cases where there is a mixture of both, veritism \emph{can} explain different epistemic value of true beliefs with regard to their \emph{significance}.

\clearpage
\section{Veritistic Accounts of Epistemic Norms}

In this section, I want to cut to the chase and discuss whether and how deontic epistemic norms can plausibly be derived in the veritist picture. To do so, I first address the question whether there can be deontic epistemic norms at all, given some natural presuppositions. I present a putative way out of the problems posed, and then discuss how deontic norms may be derived. I critically put them to the test of common objections of my own and from the literature.

\subsection{Can There Be Norms of Belief at All?}

Before I investigate whether there are actual plausible norms that can be explained by appeal to veritism, I have to discuss whether it is actually possible for a deontic belief norm to apply. The reasoning behind why this might be threatened is straightforward: According to some popular opinion, it is not the case that we can voluntarily chose to believe a given proposition. Believing is not analogous to actions in this respect. Suppose further the also widely appreciated principle that ought implies can, meaning that if I ought to $\phi$, it has to be the case that I can $\phi$. Then, if applied to belief in the sense that I can believe something if and only if I have voluntary control over believing it, it's not the case that I can believe something, and therefore it is not the case that I ought to believe anything. So deontic norms of belief simply don't apply. That would be a blow to the deontic normative aspiration of veritism, and moreover, lead to a deontics-free epistemology. There would simply be no deontic norms governing reasoning, at least not in this strong sense. So claims like ‘You are misreading the evidence. You should  believe otherwise.’ are at most evaluative claims–I don't express a requirement for you to change your belief, I merely state that I think your believe epistemically worse than some other. This does not rule out the \emph{normative} nature of, say, justification, as it might be that it still is an evaluative category.

\subsubsection{The Alston Challenge}

The argument looks something like this\footnote{Loosely following and simplifying \textcite{Alston1988-ALSTDC}. \textcite{Feldman2000-FELTEO-2} proposes a similar argument.}:

\begin{enumerate}
    \item[P1] I ought to believe $p$ only if I have voluntary control over whether to believe $p$.
    \item[P2] I don't have voluntary control over whether to believe $p$.
    \item[C] Therefore, it is not the case that I ought to believe $p$.
\end{enumerate}

P1 is just ought implies can on a popular reading of ‘can’ applied to the case of belief, and P2 is just the claim that doxastic voluntarism is false. The argument is certainly valid, but how sound is it?

One could easily write a whole dissertation on just this question and still not do justice to the full extent of it, so for the present purposes I will just briefly sketch possible strategies to object to this argument and then develop what could be a response from the veritist. I conclude that there is a sense in which one can reasonably assume deontic norms to apply to belief formation.

P2 is very widely agreed upon. Doesn't it seem obvious that beliefs are not actions, and while I can chose to or refrain from raising my arm, it is much less so with what I believe? \textcite[91]{Alston1989} famously argued\footnote{I adopted this quote from \textcite{Steup2000-STEDVA}.}:

\begin{quote}
    When I see a car coming down the street, I am not capable of believing or disbelieving this at will; ... when I look out my window and see rain falling, water dripping off the leaves of trees ... I form the belief that rain is falling willy-nilly. There is no way I can inhibit this belief.
\end{quote}

While denying P2 is an interesting route to take, it would involve arguing against \textcite{Alston1988-ALSTDC} and stating that there are in fact forms of relevant voluntarily control that we can exercise over our belief formation, or, as \textcite{Steup2000-STEDVA} does, it would lead us deep into a discussion about actions, intentions, determinism, and free will, and I don't really want to open that box in this paper. 

What other options are there? P1 is an instance of the well-established principle ought implies can. It is based on the interpretation of 'can $\phi$' as having voluntary control over $\phi$. This, however, is an additional assumption that should be argued for. It might be the case that this is not as straightforward as it appears. I present this line of response in section \ref{sec:oughtdoesnot}.

Another option is to buy into the argument and try to arrange oneself with the claim that there are no deontic norms for belief formation in this very strong sense. However, there might be other forms of influencing how I form my beliefs, and these may be under voluntary control and hence not defeated by this argument. So deontic norms about beliefs wouldn't actually state which beliefs to hold and which not, but instead concern indirect forms of influencing belief formation over some period of time. This is what Alston calls “indirect voluntary influence“. For example, the “epistemic duty to seek more evidence“ \textcite{Hall1998-HALTED-2} when in lack of decisive evidence is such a proposal, or to engage in extended reasoning or inquiry.

\subsubsection{Response: Feldman's Role Oughts}

\textcite{Feldman2001-FELVBA} denies P2 by denying ought implies can. He states that there are several kinds of \emph{ought statements} that do not entail voluntary control. An example of what he calls \emph{contractual obligations} is “You can have an obligation to pay your mortgage even if you don't have the money to do so.“ \autocite[674]{Feldman2000-FELTEO-2}. It seems obvious, though, that there are no contracts involved when I ought to believe something. \emph{Paradigm Obligations} describe what is prescriptive under normal, regular circumstances, typically noted when not in those circumstances. For example, the physician says that “you ought to be walking again now, but you still can't“ \autocite[675]{Feldman2000-FELTEO-2}, This expresses what should normally be the case. However, Feldman notes there are no \emph{real} obligations here at all. Epistemic obligations can't be of this sort, then.

Finally, he describes what he calls \emph{role oughts}. This is the type of \emph{ought} that, according to \textcite[676]{Feldman2001-FELVBA}, might be at play in epistemic obligations. “Teachers ought to explain things clearly. Parents ought to take care of their kids. Cyclists ought to move in various ways. Incompetent teachers, incapable parents, and untrained cyclists may be unable to do what they ought to do.” These ought are based on good performance relative to a standard imposed by the role involved. This standard for belief is to believe truly. And since anyone is in the role of a believer, that ought applies to any human being. So even when there is absolutely no voluntary control involved, Feldman states that this \emph{ought} still applies. If a teacher is not able to communicate the teaching material clearly, it is still the case that she ought to do so. Why? Because that's what a good teacher would do. And since she is in the business of teaching, that's the gold standard.

How plausible is this account to allow for deontic norms of belief? A lot hangs on the intuitions about Feldman's example cases.
\textcite[9]{Cote-BouchardForthcoming-CTBCTA} presents some counterexamples as a reply to a different question, but the critique applies here as well. First, being in the role of something does not imply normativity. Consider someone in the role of a torturer. \emph{Qua torturer} he ought to make his victims suffer. But it would be rather cynical to ascribe good reasons to do so to the torturer. Is it really the case that a bad teacher ought to do better, even tough she is incompetent and is in no position to better herself? The phrasing I used here is somewhat contentious, for I described the teacher as bad. This is already a normative statement. But it is different in a vital way from presupposing obligations, of course, as it is merely evaluative. A teacher failing to live up to any of the job's standards is a bad teacher, but \emph{ought} she do better? According to Feldman, that it exactly what comes with the role of believer. It is a normative requirement to do it right.

But consider: A very engaged teacher suddenly loses her voice in a constructed and implausible manner, interrupting her ability to communicate clearly. Wouldn't it be rather cynical to say she now fails to fulfill her obligations as a teacher? I don't think it would, since what she ought to do now is find other ways to communicate. One could reject this on the grounds that is still something she \emph{can} do, so it is in compliance with ought implies can. Consider that she suddenly becomes utterly incompetent through some mysterious circumstances and is not able to communicate clearly anymore at all. Is it still that she ought to do so? I think that foremost, she perhaps should not be a teacher anymore, but given that she is, yes, that obligation still applies. This intuition might be difficult to tackle, but I think that Feldman has a good point here. This does not, after all, constitute an all-things-considered \emph{ought}, and there might be other reasons that outweigh the normativity of being a teacher. So all things considered, the situation may be different, but still, as a teacher, the obligation applies.

\textcite[686]{Peels2014-PEEADC} argues that there is another crucial difference between the cases. The teacher entered \emph{voluntarily} into her role. So the obligations associated imply some kind of voluntary control. This is not the case with a believer, I do not have voluntary control over my beliefs. So this analogy fails \autocite[687]{Peels2014-PEEADC}.

Consider another case: The unfortunate older sister suddenly has to care for her little brother. Say, anyone else who could be said to have this \emph{responsibility} has vanished, which leaves her as the only one to care for her brother. But sadly, she is in no position to fulfill this task, be it for financial or personal reasons. She just cannot do it, and can't voluntarily chose to do it, either. She did not enter voluntarily into this situation and cannot evade the role. So the big-sister-role is in these regards similar to the role of the believer. But still, we would say that she has a moral \emph{duty} or \emph{responsibility} to care for her little brother. And with that, she \emph{ought} to care for him, regardless of the circumstances. This maybe shows that there are such role oughts, and these role oughts aren't as distinguished from the role of a believer as argued.

But, granted, the intuition in this case maybe clouded from the emotional force of it. It wouldn't say it is decisive that we still would hold her \emph{responsible} in this extreme case. Maybe she still has the \emph{obligation}, but has a moral \emph{excuse} for it. But to discuss this more complex and differentiating terminology will have to stay out of this paper, I'm afraid.

So it is unclear whether Feldman really succeeds in denying \emph{ought implies can} for the role of believer. I propose, however, that this line of response is not convincing enough to go on and presuppose it for the normative purposes of veritism. Moreover, outright denying ought implies can to apply to any belief is a very bold claim, as it would mean to abandon a very plausible principle and have other consequences for reasoning about normativity in the case of belief, as I discuss in section \ref{sec:normsofbelief}.

\subsubsection{Response: Ought Does Not Imply Voluntary Control}\label{sec:oughtdoesnot}

There is a related but distinct line of response proposed by \textcite{Chuard2009-CHUENW}. They challenge P1 in that ‘I ought to believe $p$ only if I have voluntary control over whether to believe $p$’ may not actually be an accurate interpretation of \emph{ought implies can} (OC). They deny that this interpretation is correct, and claim that although OC is a valid principle, ‘can’ is not to be interpreted as ‘has voluntary control’
\footnote{\textcite[601]{Chuard2009-CHUENW} emphasize that they do not claim the falsity of the premise, but instead that it is unmotivated and without good reason.}.
Instead, on all interpretations they present, ‘ought implies can’ still holds while allowing epistemic deontic norms for belief. Hence, this is a form of \emph{doxastic compatibilism} \autocite[682]{Peels2014-PEEADC}.

They do so by first analyzing the normal usage of ‘can’ and then show that, for each of these usages, it simply does not follow that ‘can’ means ‘has voluntary control’. They start with the interpretation of possibility, which includes logical, metaphysical, biological, and so on. So ‘S can $\phi$’ would be interpreted as ‘It is [logically, metaphysically, ...] possible for S to $\phi$’. The logical and metaphysical readings don't really capture the full extent of ‘can’ expressed in the principle. If the only requirement were that there is some faint logical possibility to do what you ought to, OC would be completely toothless. OC is used with much success in the moral realm, but restricting it in this way would hinder any usefulness.

 \textcite[616]{Chuard2009-CHUENW} claim there are two proposals of an interpretation of ‘can’ that come closest to its meaning in OC. First, there is the principle of alternate possibilities which states that ‘S can $\phi$’ means ‘it is possible for S to $\phi$ and possible not to $\phi$’ \autocite[615]{Chuard2009-CHUENW}, with ‘possible’ read as stronger than logical or metaphysical. The resulting OC-variant would then look like this:

 \begin{description}
    \item[OC$_{AP}$] S ought to $\phi$ only if it is [biologically, psychologically, epistemically ...] possible for S to $\phi$ and possible not to $\phi$.
 \end{description}

But from this, it does not follow that
\begin{description}
    \item[OC$_{VC}$] S ought to $\phi$ only if S has voluntary control over whether to $\phi$.
 \end{description}

 To show this, it is enough to present a counterexample. They purport “Mathilda ought to be afraid of the crocodile in front of her”, since, you know, they are dangerous \autocite[616]{Chuard2009-CHUENW}. It is in any reading of ‘possible’ possible for Mathilda to be afraid of the crocodile in front of her, yet having the emotion of fear is not something she has voluntary control over. So $OC_{AP}$ holds, while $OC_{VC}$ does not.

 The second promising interpretation is ‘can’ as ‘ability’:

 \begin{description}
    \item[OC$_{A}$] S ought to $\phi$ only if S is able to $\phi$.
 \end{description}

 The kind of ability is determined by the kind of $\phi$ at question \autocite[617]{Chuard2009-CHUENW}. Now, they propose this example: “Judy ought to understand what Nicole is going through.”, where Nicole is clinically depressed. Suppose Judy has been suffering depression as well, so she is able to understand what Nicole is going through. The deontic claim seems plausible enough. However, it doesn't follow that Judy can voluntarily chose whether to understand Nicole. Understanding is a “success term: trying to understand isn’t the same thing as actually understanding.“ What do \textcite[617]{Chuard2009-CHUENW} mean by that? According to a dispositional definition of ability due to \textcite[848]{Fara2008-FARMAA}, “an agent has the ability to A in circumstances C if and only if she has the disposition to A when, in circumstances C, she tries to A“. So in order for July to understand Nicole, she needs to be disposed to understand her in the given circumstances and when she tries to. But it is of course still conceptually possible for her to fail to understand Nicole, in exactly those cases in which her disposition doesn't manifest. Thus there are cases in which Judy has the ability, but does not have any voluntary control over whether she understands Nicole (since she may fail to do so). Hence, OC$_A$ does not entail OC$_{VC}$.

This line of reasoning suffices for \textcite[618]{Chuard2009-CHUENW} to show that no plausible interpretation of ‘can’ implies voluntary control. What they achieved, in my opinion, is to argue in a sufficiently convincing manner as to shift the burden of proof to anyone claiming that OC$_{VC}$ \emph{does} follow from ‘S can $\phi$’. 

There is still a possibility to defend the original argument by stating that it does not have to hold in the general case, but merely in the special case of belief formation. So it is only for beliefs that ‘S can believe that $p$’ means ‘S has voluntary control over whether to believe that $p$’.

So one might propose that it is only in this case that ‘S can believe that $p$’ means ‘S is able to believe that $p$’. On Fara's definition of ability, it is trivially true that S is able to believe, since, plausibly, forming a belief is not analogous to actions in that it is formed through an \emph{intention}, and an intention plausibly is a necessary condition for trying. It is then trivially true that S has the ability to believe since there simply are no circumstances in which she tries to believe something, and hence cannot fail to do so\footnote{Fara's definition could also be read as that trying is also a necessary condition for the ability, which would entail that there are no belief-abilities, given one does not form believes intentionally.}. Perhaps the definition could be altered to more plausibly incorporate cognitive abilities and their conditions. \textcite[618]{Chuard2009-CHUENW}, anyway, suggest that S has the ability to believe: “S must possess the various concepts necessary for understanding the proposition that p, and this might demand enough cognitive sophistication on her part. But the further ability to form an intention to believe p, and to carry out that intention, is not one S must have in order to be able to believe that p.” They say that all other interpretations of ‘can’ don't yield anything close to voluntary control, either, so this response fails to undermine their argument. I think again that here, this challenge is strong enough to shift the burden of proof. Their argument provides a \emph{prima facie} reason to accept that deontic norms can apply in spite of lack of voluntary control over belief formation.

To sum up, while there certainly is a profound challenge from doxastic voluntarism to the applicability of norms of belief, there also are ways out of it. Hence it is time to take a look at what those norms might look like.

\subsection{ Norms of Belief }\label{sec:normsofbelief}

When I think about what I should believe, maybe the most important epistemological question, and consider some proposition $p$, what settles the question whether I should believe $p$ is just whether $p$ is true. That's it. This is so obvious and straightforward that it seems like a truism\footnote{For a comically long list of literature on the truth norm, see \textcite[25]{Mchugh2012-MCHTTN}.}. Is this immediately normatively relevant, however? It is not that I now spontaneously decide, that, yes, $p$ is true, so I'm going to believe it. Of course, I don't have direct connection to the truth, some supernatural sense that just spurts out the right answer. If I would have, we wouldn't need any epistemology, instead, we'd all be omniscient. The truth is neither \emph{accessible} to me nor \emph{guiding} in any sense. All I can do is ask myself what speaks in favor of $p$. This is what \textcite{Gibbons2013-GIBTNO} calls \emph{the natural reaction}: I've got evidence that points to non-$p$, but $p$ is in fact true. What \emph{should} I believe? If what I should believe is what the evidence tells me to, I would believe falsely. If I believe what is true, I would believe against my evidence. And that does not seem to be rational or reasonable in any way. What's more, I do not have any idea about whether $p$ is true except for some putative evidence that would speak in favor of it. But the evidence suggests otherwise.

I think this is one of the most natural and basic truisms about epistemology: We can't just believe the truth, it has to be in some way accessible. Any norm requiring every human being to just believe what's true seems therefore misguided. In this section, I want to elaborate on what this means for a veritist account of epistemic normativity. In general, I propose the following strategy:

The veritist goal is actually the state of an ideal being, without the pesky limitations that humans bring with them. This epistemically perfect being is not concerned with cognitive limitations or problems with truth-accessibility, instead she can just believe, like that, at will, what is true. Each and every of her doxastic attitudes is most valuable. Once the state of this being is adequately described and free of contradictions (since even this being does not like contradictory beliefs), a regulative norm is developed that aims towards this goal while taking into account the \emph{buts} and \emph{can'ts} of normal human beings. This regulative norm has to be plausible, of course, and somehow be derived from the veritist goal.

An important qualification is the following: I mostly speak of epistemic value and propose that it is only epistemic value that has any impact here. It may be that there are actual reasons to belief something that promote prudential or moral value. And there may be corresponding norms for it generating reasons to believe something. These reasons, however, are not epistemic reasons; they do not promote epistemic value.

Let’s be also clear on what I want to propose here. I want to say that there are actual epistemic deontic norms to the effect that given some circumstances, I should believe some proposition, and these norms are explained by appeal to true or accurate belief as the only epistemic value. I do not want to establish a kind of deontological conception of justification. In fact, I scarcely make reference to justification at all. How and if this might be developed into such an account, or if both are in some strong plausible sense compatible is left open.

\subsubsection{The Goal of Belief}\label{sec:goalofbelief}

What could this goal for an ideal epistemic being look like? And what would it help us with? \textcite[157]{David2001-DAVTAT-7} proposes:

\begin{description}
    \item[TG] For all propositions $p$, if $p$ is true then S believes $p$, and S believes $p$ only if $p$ is true.
\end{description}

or, in short, ‘S believes $p$ if and only if $p$.’, with S here only as the ideal epistemic being. This seems to be a reasonable approach. However, as argued extensively in section \ref{sec:varieties}, we want to take into account the value of suspension of judgment as well as disbelief. This phrasing of the goal, however, is silent on both. 

Is it, though? Let's take a look at disbelief. Say there are only two propositions, the true proposition $p$ and false non-$p$. Then this goal states that S believes $p$, and not believes non-$p$. If we require S to have a doxastic attitude towards non-$p$, that would be disbelief. But does that add any additional epistemic value? I stated earlier that believing some true proposition is equivalent in epistemic value to believing some equivalent proposition. So if S would also disbelieve non-$p$, this would not add any epistemic value. This seems plausible on first glance, but I disagree with this line of reasoning. Disbelieving non-$p$ would make a difference in epistemic value. This is, as I argued in section \ref{sec:varieties}, to take into account the difference in epistemic value between someone who believes $p$ and disbelieves non-$p$ and someone who just believes both. Additionally, as comes up repeatedly in this section, this formulation does not offer any guidance towards concerning, if we were to derive a deontic norm from it. Say I'm considering whether to believe $q$, with all and everything pointing towards $q$ being false. Then my goal should guide me to disbelieve $q$, and not be uninformative except excluding believing $q$. Should I be ignorant and forget about this proposition altogether? Should I suspend judgment? Should I disbelieve $q$? The answer is, of course, intuitively clear, but the goal does not give this answer. 

Consider the following rephrasing:

\begin{description}
    \item[TG*] For all $p$, S has a most accurate doxastic attitude towards $p$.
\end{description}

This would require the ideal epistemic being to have a doxastic attitude towards \emph{every} proposition, which are, potentially, indefinite in number. \textcite[159]{David2001-DAVTAT-7} proposes to restrict the propositions to those which one can grasp, or which one considers \autocite[17]{Chisholm1966-CHITOK}. However, these considerations don't concern us here, since the ideal epistemic being can grasp any proposition and considers all of them as well.

But, of course, there are problems of propositional self-reference. Consider, for example, this proposition:
\begin{description}
    \item[CP] I do not have a doxastic attitude towards this proposition that is most accurate.
\end{description}

This seems like it could pose a problem for the ideal being. Is it still possible for her to have a most accurate attitude towards $CP$?
If she does not, then TG* is false. That is no option. But if she does have a most accurate doxastic attitude towards CP, then CP is just false. So the accurate attitude would be disbelief. This proposition thus does not pose a problem for TN*.

In the literature on truth norms, one recurring issue are so-called \emph{Moore-paradoxical propositions} \autocite[6]{Wedgwood2013-WEDTRT}. They have the form of 
\begin{description}
    \item[MP] $p$, but I don't believe that $p$.
\end{description}
In combination with a simple truth norm, they spell trouble. Consider a truth norm of the form
\begin{description}
    \item[TN] For all subjects S and propositions P: S ought to believe that P if and only if P is true.
\end{description}

They present a particular problems to deontic truth norms in the following way: Suppose MP is true. Then, any TN would require the subject to believe MP, and if the subject does so, the resulting belief would be false, since S does now believe that $p$\footnote{That is, if belief is closed under elimination of conjunction, so if I believe that '$p$ and $q$', I believe $p$ and I believe $q$. I argued elsewhere that belief is not closed under logical entailment, but this is a different and weaker principle. Let's grant it for now to see where it goes.}. So TN requires S to believe MP, and also require S to not believe MP. This seems fairly paradoxical. Does it pose a problem for the ideal epistemic being, too? The most accurate attitude towards MP is disbelief. Here is why: Suppose $p$ is true. Then, the accurate attitude is belief in $p$. So MP is false. Suppose $p$ is false. Then, MP is false. Hence, the most accurate attitude is disbelief. But what if the ideal being hasn't considered $p$ at all, before considering MP? That is, what if MP is true? That's a stretch for an ideal epistemic being, but let's go through it: if MP is true, the ideal being believes it. But then it's false! So the ideal being disbelieves it. There simply isn't this kind of normative dissonance as in the case of the truth norm. So, the ideal being comes out unscathed.

What about flat-out self-contradictory sentences like (LP) “This sentence is false“? So far I assumed bivalence, that is, every proposition is either true or false. Then, the ideal being would just adopt the most accurate doxastic attitude towards that proposition. However, if we grant that this sentence is maybe both true and false, and also stipulate that per proposition, the ideal being has exactly one doxastic attitude, it does not matter which attitude she takes. Every doxastic attitude would not be perfectly valuable: If the ideal epistemic being fully beliefs LP, she believes truly, but since LP is also false, she also believes falsely. If she suspends judgment, she suspends judgment on a true and false proposition. If she disbelieves, she does so accurately, but since LP is also true, she also disbelieves a true proposition. Intuitively, these cases all seem epistemically equally valuable, if not maximally so. It is however most valuable given the circumstances that with one doxastic attitude, one cannot be perfectly accurate.

\subsubsection{The Derivation}

The question is now whether this can be spelled out in terms relevant to human beings to do justice to \emph{the natural reaction}. This is not a trivial task. \textcite[Ch. 4]{Gibbons2013-GIBTNO} is especially vocal about its challenge.
\noindent \textcite[Ch. 4]{Boghossian2008-BOGCAJ} acknowledges that it has proven extremely difficult to say how exactly such a norm could be derived. \textcite[16]{Greenberg2016-GREITN} puts it like this:
\begin{quote}
It [the truth norm] tells us (roughly) to believe truths and not believe falsehoods. But this isn’t very helpful as a guide. We can’t just believe the truth at will: we aren’t omnipotent. Cognitively limited creatures like us need to take means to this end.
\end{quote}
Both of these claims are made with respect to truth norms, but their key contents translate well to the veritist challenge. It is–obviously–not possible for us to just believe the truth. \textcite[62]{Goldman2002-GOLTUO-2} adds to the constraints the notions of reachability, realizability and accessibility.
So what the veritist is looking for is a norm that \emph{guides} us in a relevant and meaningful way, and is somehow accessible and realizable. Maybe these constraints can in some form or other be derived from an \emph{ought implies can} principle, but I wouldn't want to bet on it. However, they seem plausible and reasonable. For this is just what the natural reaction demands. Of course, the natural reaction could just be wrong. An argument for or against this, however, has to wait for another time. That there might be such a way do derive a norm, is indicated by \textcite[154]{David2001-DAVTAT-7}: “It is not hard to see how the truth‐goal fits into this picture. It promises to provide a connection between the concept of justification and the concept of true belief, tying together the different ingredients of knowledge.” Although I am not after an explication of justification in normative terms, it is reassuring that this approach seems worthwhile, and I find this translates just as well to explaining deontic norms.

Explaining epistemic normativity through veritism is a consequentialist approach. There are multitudes of consequentialist approaches out there, and if moral theory is any guide, it will be way to much to give an overview here \autocite{sep-consequentialism}. For example, one could focus on specific acts in their evaluation or on the value of the rules that guided acts. An epistemic analogue of this rule-consequentialism would be to evaluate the belief-forming \emph{process} instead of just the belief itself. One could take into account all consequences of that belief, so all those beliefs that are directly and indirectly caused by forming that belief, or completely ignore it. One could evaluate whether some threshold of value is met, and call the moral demands satisfied, or one could propose to maximize the value. So on and so forth. What all of these approaches have in common, however, is that they usually deny that the goal itself is guiding or directing, and instead propose it as a criterion or standard for evaluative and deontic judgments \autocite[16]{sep-consequentialism}. This is very similar to the current approach of stipulating a hypothetical ideal epistemic being.

A consequentialist theory needs some bridge principle or other to connect its evaluative standards to deontic normativity\footnote{Put slightly more technically, the set of choices available to the subject is to be epistemically ranked by the standard of the epistemically ranked outcome set \autocite{adler_ranking_2011}.}. The number of actual principles can be manifold; here, I just want to propose one particular example to show how it could be possible to arrive at epistemic deontic norms.

Let's go with something general like the following:

\begin{description}
    \item[BP] If S's goal is X, S ought to use the best means accessible to her to achieve X.
\end{description}

In the epistemic case these ‘best means’ could include responding to reasons, proportioning my beliefs to the evidence and so on. There is one contentious word in here, though, which is \emph{best}. This seems to be evaluative as well, and this bridge principle should not need an additional unspecified source of value. However, I don't think much hangs on this, and charitably read this could be interpreted as ‘the means that make it most likely’ or something along these lines.

The goal should be in some sense resembling TG*. In fact, any change would need motivation to not seem \emph{adhoc}. So let us just state TG* as the goal. The derivation then looks something like this:

\begin{enumerate}
    \item[P1] The goal of any believer S is that for all propositions $p$, S has a most accurate doxastic attitude towards $p$. (From TG*)
    \item[P2] If S's goal is X, S ought to use the best means available to her to achieve X. (BP)
    \item[\textbf{MN}] S ought to use the best means accessible to her to achieve that for all $p$, S has a most accurate doxastic attitude towards $p$. (From P1 and P2)
\end{enumerate}

I presuppose here that the goal stated just is the goal of any believer. This may be questioned, as for example \textcite{Kelly2003-KELERA} does. However, this is what the veritist account does for us. Compare: “it is no objection to hedonistic act-utilitarianism that there are some pleasurable experiences that some of us don’t desire to bring about” \autocite[362]{Berker2013-BERETA-2}. It is central to epistemic evaluation in a theory to be compared to the epistemic goal stated by the theory, regardless of whether the subject actually possesses that goal. Maybe P1 and P2 have to be rephrased a bit to incorporate this finer point.

One could object that MN is not really a norm of belief. After all, it does not state, exactly, when to adopt a doxastic attitude. Maybe proponents of this view would prefer a norm of the form
\begin{description}
    \item[MN*] For all $p$ (S ought to adopt D towards $p$ if and only if S's best means accessible to her suggest adopting a most accurate doxastic attitude D towards $p$).
\end{description}
I do not think that much hangs on this objection, and would, of course, be willing to concede alternative representations. However, one would have to tell a story as to how this can be derived in the veritist picture and would have to review whether the desiderata and objections below are met.

\subsubsection{The Norm of Belief}

To see whether MN is any plausible, I've compiled a list of desiderata and common objections to deontic epistemic norms, which I motivate and discuss in turn:
\begin{enumerate}
    \itemsep-0.2em
    \item \label{item:dox} The norm should account for the doxastic attitudes of \emph{believing}, \emph{disbelieving} and \emph{suspension of judgment}.
    \item \label{item:guiding} It should be \emph{guiding} in a relevant sense.
    \item \label{item:accessible}It should only require something \emph{accessible} from a subject.
    \item \label{item:arbitrarilylong}It should not require a subject to believe \emph{arbitrarily long propositions}.
    \item \label{item:arbitrarilymany}It should not require a subject to believe \emph{arbitrarily many propositions}.
    \item \label{item:blindspot}It should not require a subject belief so-called \emph{blindspot propositions}.
    \item \label{item:compatible}It should be \emph{compatible} with what on usually takes to be a deontic epistemic norm, such as following one's evidence.
    \item \label{item:nontrivial}It should be \emph{non-trivial} in that it is possible, not only theoretically, to violate the norm.
    \item \label{item:insignificant} It should not require a subject to believe insignificant propositions.
\end{enumerate}

(\ref{item:dox}.) \emph{Doxastic attitudes.} 
As argued before, a deontic norm should require the subject to believe, disbelieve and suspend judgment, given the circumstances. Does MN deliver that? I argue that it does. In cases where the subject's best means point to the truth of $p$, the most accurate attitude is belief. Analogously in the case of disbelief. For suspending judgment, the case is a bit more complicated, since suspension of judgment is never objectively most accurate. However, the norm requires the subject to use the best means accessible to her to achieve or bring it about to have the most accurate doxastic attitude towards $p$. In cases where the evidence is indecisive, from the subject's perspective, the most accurate attitude is to suspend judgment. Since for neither full belief nor disbelief would be most accurate in this case. This would extend to degrees of belief, if we were to incorporate this model into the theory. I concede that this point is arguably not fleshed out, and there may be some additional modification necessary to make the goal in the right sensitive to the subject's position. For the present purposes, this has to suffice.

(\ref{item:guiding}.) \& (\ref{item:accessible}.) \emph{Guidance and accessibility.}
 Is MN compatible with the natural reaction? It only requires the subject to use means accessible to her, so that condition is satisfied. Roughly speaking, it provides guidance in a relevant sense, too: S ought to use the best means available to her. This is not so innocent as it might sound, however, since how is S to know which means actually are the best means? Interpreted as those means that make it most likely, S might not be in a position to know which means available to her make it most likely, and then the problem of the natural reaction just reappears on a different level. To stipulate that she should use what she \emph{believes} to be the best means is not the right move, however, because then the norm would be completely arbitrary, depending on her beliefs. Someone trusting her, say, gut feeling to be the best means to achieve the epistemic goal would then be normatively required to believe so. Alternatively, to say those means which she \emph{ought} to believe to be the best means is simply begging the question. To say something about what she \emph{justifiedly} believes to be the best means would unnecessarily introduce this epistemic notion into the definition of deontic epistemic normativity, and that is not what the veritist wants. However, by spelling out what the actual means \emph{are}, this could be avoided, if these entail that S is in a position to know them. This question dives deep into the discussion about justification and might touch on an ongoing debate about internalism and externalism. This is not the place for that discussion.

(\ref{item:arbitrarilylong}.) \emph{Arbitrarily long propositions.} 
Some proposition are such that no human being can believe them. \textcite[12]{Mchugh2012-MCHTTN} poses this challenge thusly: “But there are many true propositions that are too complex for any person to grasp, and hence too complex to believe”. \textcite[279]{Bykvist2007-BYKDTI} stress that it is metaphysically or biologically impossible for human beings to understand some extremely complex propositions. With the appeal to \emph{ought implies can}, with ‘can’ read as metaphysical or biological possibility, a norm that would require the subject to do so would be false. I argue that MN does not require the subject to believe arbitrarily long propositions, on the grounds that S should use the best means accessible to her. These means most plausible employ some kind of cognitive process responsible for considering and evaluating propositions. If this process cannot deal with indefinitely complex propositions, then the norm does not require the subject to do so. If it can, however, I don't see why the subject can't form any doxastic attitude towards it. So MN does satisfy this condition.

One might alternatively respond to this and similar objections that what is to be expressed by the norm is not what a subject \emph{should} do or believe, but instead what she \emph{may} or what is \emph{permissible} for her to believe\footnote{Like, for example, \textcite{Whiting2013-WHITTA-3} does.}. After all, this would solve the present problem, as a norm of this form does not require one to believe any true proposition, they just allow it. Similarly, blindspot propositions (see item \ref{item:blindspot}) do not pose a problem. What they fail to provide however, is to state that one should believe in the case of compelling evidence. This is a very basic intuitive claim about epistemic normativity, but the norm fails to explain it \textcite[18]{Mchugh2012-MCHTTN}.

(\ref{item:arbitrarilymany}.) \emph{Arbitrarily many propositions.}
To believe all true proposition is something no human being can achieve. The number of true propositions is positively indefinite. A norm that requires this of the subject entails than one is “unavoidably violating an indefinite number of requirements” \autocite[12]{Mchugh2012-MCHTTN}. This is not only implausible, but given \emph{ought implies can}, it falsifies the norm. The response I give to this objection is the same as above, it simply fails to threaten MN. The subject's cognitive processes most probably won't allow her to believe an indefinite number of propositions. But since those are, arguably, the means with which the subject is to achieve the goal, she is not required by the norm to believe each and all propositions.

(\ref{item:blindspot}.) \emph{Blindspot propositions.}
In section \ref{sec:normsofbelief} I presented the challenge from \emph{Moore-paradoxial propositions} or what \textcite[281]{Bykvist2007-BYKDTI} call \emph{blindspot propositions}. Truth norms like TN above in particular suffer from this challenge and are usually altered to meet the requirements posed\footnote{See for example: \textcite[282]{Bykvist2007-BYKDTI}, \textcite[12]{Mchugh2012-MCHTTN}, \textcite[6]{Greenberg2016-GREITN}.}. The problem arises because the truth norm requires the subject to believe the true proposition, given it is true, and thereby making it false, and hence also requiring the subject to not believe it. So this proposition generates a normative contradiction in conjunction with a truth norm like TN. Does this pose a problem for MN, too? Consider a subject cognitively capable of recognizing that if she were to believe such a proposition, she would thereby believe falsely. Her best means then surely suggest not to believe this proposition. No contradiction here. But then suppose she is not capable of seeing the problem at all. She just goes for $p$, thinks it true, and then believes MP on the basis of that. TN tells her thusly to belief a false proposition. But this is not problematic, since we explicitly want to allow for the norm to encourage going after what the evidence suggests, even if the proposition is false.

(\ref{item:compatible}.) \emph{Compatibility with intuitions about epistemic normativity.}
This is a potentially extensive point, so I will restrict myself to explain a few central intuitions. \textcite[39]{Boghossian2003-BOGTNO} lists some of them, i address them in turn.

\emph{We ought to believe that which is supported by the evidence and not believe that which has no support.} This point is fairly straightforward. Most plausibly, the best means accessible to a subject are sensitive to evidential considerations. That is why believing what is supported by the evidence will most certainly be a prime candidate for any subject following MN. And what is not supported will result in suspension or disbelief.

\emph{ We ought not to believe $p$ if some alternative proposition incompatible with $p$ has a higher degree of support.} To recognize incompatibilities in one's belief set is one of the key virtues of cognitive agents and will most probably be covered by the best means available to the subject. Some might argue that there is an \emph{a priori} argument here: On this view, contradictions are \emph{never} what one ought to believe, even when recognizing them exceed the cognitive capabilities of the subject. Depending on the interpretation of the best means, however, MN might deliver the verdict that a subject ought to believe a contradiction. Now, this is a fine and not so clear point. Maybe there are contradictions that can be rationally believed, but certainly not without good reasons. I grant that if one is firm on this point, and hold that contradictions should never be believed, than MN can't live up to this standard. Since, after all, what one ought to believe is now dependent on one's best means accessible.

\emph{We ought to believe p only if its degree of support is high enough.} The correct doxastic attitude is determined by the best means. What exactly this means for full believe and for the degree of belief is left open. So this is certainly compatible with MN.

(\ref{item:nontrivial}.) \emph{Non-triviality.}
If a norm is such that one cannot conform to it, ever, then it is not much of a norm. The same applies if the norm is always automatically satisfied. This would make the norm trivial and of no epistemic interest whatsoever. Could one violate MN? Most positively, yes. If the evidence points one way, but the subject bases her beliefs on something else, like her gut feeling, she does not follow MN. And those are cases that MN wants to rule out, too. 

Now, there might be a different problem here. If the norm is read on a case-by-case-basis, that is, for every proposition, there are different ‘best means’ accessible to the subject, one might argue that in cases of misleading evidence, one should instead use the means that actually lead to an accurate doxastic attitude. That may be the gut feeling which is right, by happenstance. I don't think this grave though. First, what the best means actually are should be defined in some sense or other. If it is just, depending on the random circumstances, what actually conduces towards the truth in this particular instance, there is not much of an epistemological relevance about this norm. So this reading can't be right. Second, the way the norm is phrased, it is a wide-scope norm. It is not that ‘for every $p$, S ought to use the best means’, but that ‘S ought to use the best means so that for all $p$ ...’. So this implies some kind of general-purpose heuristic and is not entirely case-dependent.

(\ref{item:insignificant}.) \emph{Insignificant propositions.}
This point is in my eyes \emph{not} a desiderata for an epistemic norm. I explain why. A prominent objection to epistemic norms spelling out normativity in regulative terms like \emph{oughts} and \emph{shoulds} is that they seem to require me to believe seemingly insignificant propositions. That can't be right, it seems counterintuitive. I propose that this requirement can be right: as argued in section \ref{sec:varieties}, if a proposition is seemingly insignificant, in some cases the intuition confuses practical with epistemic value. It may be practically superfluous to believe some irrelevant truth, but not epistemically: Even the most insignificant proposition is of \emph{some} epistemic value. Of course, I may have practical reasons that outweigh my epistemic reasons by far, such as limited cognitive abilities, in order to still have the resources to focus on what's important to me. But I would suggest that this does not concern the epistemic standpoint.

To sum up, I think this first proposal of a deontic norm of belief meets the desiderata and objections posed to some degree. It meets some conditions better than others, though, and most probably, there are more objections right around the corner. My aim was just to show that it can be epistemically fruitful to go into these to some detail, and of course I do not believe this matter settled.

\subsubsection{Alternative Approaches}

The approach I presented is, as one may guess, not the only available strategy to the veritist. Although not necessarily identical to a veritist approach, the claim that the aim of truth is constitutive for belief promises to explain its normativity on the merits of true belief alone. Of course, beliefs do not literally aim at beliefs, since beliefs “are not little archers armed with little bows and arrows” \autocite[267]{Wedgwood2002-WEDTAO}. These approaches usually spell out a truth norm like TN discussed above, and try to account for other, secondary norms of belief by appealing to the fundamental normative importance of the truth norm. In other words, they state that the truth norm achieves its normative power through the constitutive aim of belief, and derive secondary norms by reference to the truth norm. As these approaches mostly want the norm to apply to human beings instead of an ideal epistemic being, these views are challenged by the same objections as listed in the previous section. Proponents of this view mostly deal with them by alteration. It is possible to formulate a truth norm that is not subject to the objections from arbitrarily long and many propositions \autocite[159]{David2001-DAVTAT-7} and not to the objections from blindspot propositions \autocite{Greenberg2016-GREITN} either. Some strategies involve denying regulative normative force of the truth norm and instead just stipulate that it's an evaluative norm \autocite{Mchugh2012-MCHTTN}. Others appeal to the difference between doxastic and propositional attitudes and introduce normative concepts relative to this distinction to accommodate blindspot propositions \autocite{Wedgwood2013-WEDTRT}. 

What I find most troubling in these accounts is that they do not convincingly provide \emph{guidance} and \emph{accessibility} to the subject. Since, after all, it's all fine that I should believe the truth, but I just can't. How to do justice to the natural reaction? A way out of this is to distinguish between \emph{objective} and \emph{subjective} norms. Objective oughts apply to someone of human cognitive capacity but with perfect information, so a theoretically achievable possible state, whereas subjective oughts apply to regular human reasoners. The truth norm would then spell out an objective requirement, whereas some derived, more accessible norm spells out the subjective requirements. \textcite[Ch. 2]{Gibbons2013-GIBTNO} makes a point out of the conflict of the two: given two different oughts, which one takes precedence over the other? If both are equally applicable, they generate an incompatibility, what Gibbons calls \emph{the puzzle}, in the case of reasonable false belief. If the truth norm takes precedence, this would mean to take an objectivist position on normativity. If the derived norm takes precedence, one would take a subjectivist position.

Alas, this discussion can't be spelled out here and now. I will leave it at that with this short overview. I hope to have shown though that, of course, the proposal above is not the only relevant position to take.

\subsection{A Remark About Justification}

So far, as noted earlier, I was fairly silent on the concept of justification. I wanted to take a look at the issue of deontic norms independently of the issues of epistemic justification, which brings a whole range of problems and discussion on its own, without taking normativity into account at all. But obviously, there is some sense in which the concept of epistemic justification is intimately tied to that of epistemic normativity.
I would not make too much of this, though. I believe that there is a perfectly fine way to derive deontic norms from veritism and also derive an account of justification from it without necessarily stating that one (completely) constitutes the other. I would suggest that there can be cases in which I follow all deontic norms, but I am still not justified, and cases with justification while violating deontic norms. It is still very reasonable to assume that these concepts mostly overlap.

\clearpage
\section{Conclusion and \emph{Quo Vadis}}

I tried to explain in this paper why it is that “What should I believe” seems, intuitively, so easy to answer. That, however, has proven not nearly as easy. The account presented is a consequentialist account of epistemic normativity grounded in veritist axiology.

I argued that, given some modifications, the veritist picture can plausibly account for the relevant intuitions about epistemic value. I still leave open an inordinate amount of questions regarding the axiology. In particular, it is not clear what it is that can be compared with respect to its value, and how exactly the resulting amounts of value are to be measured. In a sense, there is no specific answer to the latter question, at least not in this very tradition coarse-grained model of doxastic attitude. The former question, though, needs to be addressed in a bit more detail. I established that beliefs in the same proposition can easily be compared. I frequently made use of the comparison of epistemic situations. I mostly appealed to intuition or vague notions about what it means for an epistemic situation to be valuable, or to gain value, or to be more valuable than another situation. There might be some intuitive sense in which that is unproblematic to further my argument. However, I would very much like to see this spelled out a bit, and to be clear about where the limitations are. \textcite{Berker2013-BERETA-2} offers some helpful terminology. He calls the different bearers of epistemic value the \emph{evaluative focal points} of a theory and the method or way in which the somehow accumulated value is measured a \emph{theory of overall value}. It would be nice to have a precise notion on whats those bearers are and what such a theory entails. Then, much more specific work in deriving epistemic norms is possible, I gather.

I proposed some account to deal with the challenge from significance. This seems like an intuitive answer at first, but my presentation of the idea is sloppy at best. What I had in mind when talking about the objective significance of a proposition (and hence, a belief) was something like the amount of information it carries. The account of information in communication theory states very roughly that a message carries more information if it is more surprising. In other words, the more alternatives that could have been sent it rules out, the higher its informational value is. I could imagine some account of information applied to propositions that measures information in a similar way. The idea is simple: The informational value of a statement like ‘The universe is expanding at an accelerating rate’ is high because it rules out lots of alternatives, namely those in which its constituent propositions (and maybe entailments) are false. A very basic and highly contingent statement does not rule out many alternatives, it is compatible with a lot of them. It may not be impossible to \emph{actually} measure the amount of information, but for a comparative account, it might be enough. And that is all what an account of significance to support veritism against the challenge strives to do, anyway.

I presented the classic argument from Alston against the contingent possibility of synchronic deontic epistemic norms. My responses to it are somewhat opinionated, and I brought forth no conclusive evidence against the challenge. If one does not buy into the arguments against the challenges from doxastic voluntarism at all, though, the point about deontic epistemic normativity is moot. But all hope is not lost for deontic epistemic norms. There is a second, derivative form of these norms, that I previously mentioned briefly. I was mostly concerned, without explicitly stating it, with some form of direct or synchronic norms of belief. These state (roughly), given my current situation, what I should believe right now. Diachronic norms can additionally stipulate the need to search for more evidence, to ask a fellow reasoner and other time-consuming tasks. These demands may generate norms on their own. And they may perhaps also be explained via a veritist consequentialist picture. I don't see how these wouldn't be compatible. As before, however, I cannot support these ideas with arguments in the present paper.

Lastly, the deontic norm I presented seems plausible enough at first glance. It reeks of arbitrariness, though. The specific formulation is exactly so as to be able to respond to the challenges that I put forward. It is very much open that any other deontic norm might do as well, provided one takes into account the constraints or explains them away. 

In the end, I merely have indicated that the veritist account might be an acceptable candidate to explain epistemic normativity. But, perhaps, that is still something.

% \nocite{*}
% \begingroup
% \setstretch{1}
% \printbibliography
% \endgroup
\clearpage
\printbibliography

\end{document}