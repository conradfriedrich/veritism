%!TEX program = xelatex
\documentclass[12pt,numbers=noenddot]{scrartcl}
% \usepackage[ngerman]{babel}
\usepackage[a4paper,lmargin={3cm},rmargin={3.5cm}, tmargin={2.5cm},bmargin = {2.5cm}]{geometry}
\usepackage{amsmath}
\usepackage{mathabx}
\usepackage{mathtools}
\usepackage{stmaryrd}
\usepackage{enumitem}
\usepackage{graphicx}
\usepackage{courier}

\usepackage{relsize}
% \linespread{1.2}
\usepackage{setspace}
\onehalfspacing

\usepackage[authordate]{biblatex-chicago}
\addbibresource{veritism.bib}

% \usepackage[square]{natbib}
% \usepackage{jurabib}
% \usepackage {algorithm2e}

% Package, das die Benutzung von Old Standard erlaubt
\usepackage{fontspec}

% \setmainfont{OldStandard-Regular.otf}[
% Path = /usr/local/texlive/texmf-local/opentype/,
% BoldFont = OldStandard-Bold.otf,
% ItalicFont = OldStandard-Italic.otf]
\setlist[itemize]{leftmargin=*}


\setmainfont{Old Standard TT}

% \bibliographystyle{jurabib}
% \bibliographystyle{jureco} 
% \renewcommand*{\bibbtsep}{In: }
% \renewcommand*{\bibjtsep}{In: }

% \renewcommand*{\biblnfont}{\normalfont}
% \renewcommand*{\bibfnfont}{\normalfont}
% \renewcommand*{\bibelnfont}{\normalfont}
% \renewcommand*{\bibefnfont}{\normalfont}
% \renewcommand*{\bibtfont}{\textit}
% \renewcommand*{\bibbtfont}{\textit}
% \jurabibsetup{
%   bibformat=ibidem,
%   bibformat=compress,
%   dotafter=bibentry
% }

% \makeatletter
% % \renewcommand{\l@section}{\@dottedtocline{1}{1.5em}{2.6em}}
% \renewcommand{\l@subsection}{\@dottedtocline{2}{4.0em}{3.6em}}
% \renewcommand{\l@subsubsection}{\@dottedtocline{3}{7.4em}{4.5em}}
% \makeatother

% \addtokomafont{disposition}{\normalfont}
\addtokomafont{sectionentry}{\normalfont\bfseries}


\setkomafont{subject}{\normalfont\small}
\addtokomafont{title}{\normalfont\bfseries}
\addtokomafont{part}{\centering\normalfont\bfseries}
\addtokomafont{partnumber}{\centering\normalfont\bfseries}

\addtokomafont{partentry}{\normalfont\bfseries}
\addtokomafont{partentrypagenumber}{\normalfont\bfseries}

\addtokomafont{section}{\normalfont\centering\bfseries}
\addtokomafont{subsection}{\normalfont\centering\bfseries}
\addtokomafont{subsubsection}{\normalfont\centering\bfseries}

\addtokomafont{publishers}{\normalfont\small}
\addtokomafont{date}{\normalfont\small}
\addtokomafont{author}{\normalfont\small}
\addtokomafont{descriptionlabel}{\normalfont\bfseries}

% \renewcommand{\thesection}{\Roman{section}} 
% \renewcommand{\thesubsection}{\thesection.\Roman{subsection}}


\subject{  University of Cologne \\
  Department of Philosophy \\
  Bachelor Thesis
  }
\title{Veritism and Epistemic Normativity}
\author{C. Friedrich}
\date{\today}


\usepackage{amsthm}
\newtheorem*{defi}{Definition}


\begin{document}
\begin{titlepage}
\maketitle

\abstract{\textbf{Abstract.} Can epistemic normativity be plausibly explained by appealing to truth as the single final epistemic value? [All structural elements and section titles are placeholders.]}

\thispagestyle{empty}
\end{titlepage}

\tableofcontents
\newpage

\section{Introduction}

In a certain sense, and one that, for example, \textcite{Berker2013-BERETA-2} holds, epistemology is an inherently normative endeavor. To Berker the most fundamental question of epistemology is, despite the etymological origins of its name, not concerned with what makes a belief count as knowledge but is instead purely normative: “What should I believe“?

In pondering this question the similarities of epistemology to the theory of normativity and especially ethics are strikingly obvious, if only at a first superficial glance. It makes a lot of sense then, if one is to undergo the project of viewing epistemology through the lens of normativity, to at first take a look at ethical theories of value anf normativity. While the properties of epistemic and ethical objects of interest may stop sooner than expected, the groundwork in ethical theory shouldn't be completely disregarded, either.

Go on and on
\footnote{As may be obvious, I presuppose a realist notion of truth.}
\section{Veritism}

\subsection{Values}
\subsection{ What is Veritism? Shortly explain.}

Why is it ever that anyone \emph{should} believe something, or is \emph{required} to believe, or is \emph{justified} in believing, or is \emph{rational} to believe? What is that yields some kind of normative requirement to the believer? A most natural answer, one that immediately springs to mind, is it's connection to the truth. You should believe that proposition because it is most likely to be true! (You are justified in believing that proposition because it is the only one that your evidence supports. It would be completely irrational to believe that other proposition, you don't have any reason to believe it.) What else than true belief is the epistemic goal of anyone in the business of believing anything, that is, a cognitive agent? It is this very intuitive thought that motivates postulating truth as an -- or the -- epistemic value.
\begin{itemize}
    \item Taxonomy of the normative
    \item Intrinsic Value vs Extrinsic value 
    \item Instrumental Value vs. Final value 
    \item Good vs. right 
    \item value vs goal (can be footnote)
\end{itemize}

\subsection{What is Veritism then? Clearly explain the position, and what it tries to do. Use quotes from Bonjour, Lehrer.}

\textcite[54]{Goldman2002-GOLTUO-2} defines veritism as  “... the unity if epistemic virtues in which the cardinal value, or underlying motif, is something like true, or accurate, belief”. Borrowing the terminology from Goldman, \textcite[360]{Berker2013-BERETA-2} proposes to label as veritism the position that „ ... our only epistemic goals are (i) the accumulation of true beliefs and (ii) the avoidance of false beliefs”. \textcite{Zagzebski2004-ZAGEVM-2} has a different name for a very similar position, what she calls epistemic value monism: “Any epistemic value other than the truth of a belief derives from the good of truth“.

The unifying theme of these quotes is apparent: What’s central to all three accounts clearly is the notion of \emph{true belief}. True, or accurate, belief has to be a value, goal, or motif with a distinct role in the theory. This role is determined as cardinal, single, and non-derivative. In other words, veritism says that true belief has to be regarded as the only \emph{final} value. No other epistemic concept like knowledge or justification has final epistemic value. It is important to note, however, that this does \emph{not} entail that other epistemic concepts have \emph{no} epistemic value, they just don’t have \emph{final} epistemic value.

That, basically, is the gist of the position: veritism consists in the claim that true belief is the only epistemic value or goal. Yet just this simple claim is not enough. if the goal is to have as many true beliefs as possible, Gullible Gilbert could just believe every proposition he encounters. Gilbert would certainly stumble upon a lot of true ones, but would also hold lots and lots of plainly false beliefs. Still, this would promote the only epistemic goal, as Gilbert accumulates more and more true beliefs and generates more epistemic value. I will flesh out in more detail in what sense Gilbert’s mental state and the whole of his beliefs can have cumulative value derived from a just true belief as valuable. It seems then that by just believing every proposition there is Gilbert is able to get into an epistemic state of maximal value. This is what \textcite[360]{Berker2013-BERETA-2} calls \emph{epistemic recklessness}. Intuitively though, this state is a lot less valuable than the state Precise Priscilla is in: She beliefs all of the true propositions To respond to this, it seems quite natural to expand the epistemic goal to consist of to goals, really: ”...the twin goals of acquiring true beliefs and avoiding false ones“, as \textcite[339]{Berker2013-BERETA-2} puts it. This idea has already been proposed by \textcite[17]{James1896-JAMTWT-19}, who coined the phrase: ”Believe truth! Shun error!“.
\begin{itemize}
    
    \item how does it relate to the deontic? What does that mean, deontic epistemology? (Steup?)
    \item Truth-Value-Monism / Veritism. The same? List different definitions. Goldman / Berker / Zagzebski 
    \item Truth-Value-Monism vs. Pluralism about values 
    \item Other-Value-Monisms (Knowledge, ...) 
    \item TVM vs. Non-Value-based accounts of normativity 
    Deontological account of justification has the problem of explaining intuitions of the form you should believe it because its most likely to be true!

    Deontological accounts also need to provide an explanation for the problems posed by a false doxastic voluntarism and ought-implies-can. Veritism gets evaluative notions for free! Docastic voluntarism still is a problem for deontic requirements, though. Strategies: Flat-out deny deontic (regulative) requirements, or adopt one of the deontological accounts position and distinguish between different forms of doxastic voluntarism or deny ought-implies-can in one of its different forms. (Perhaps) we'll see later what these strategies exactly amount to .
    \item Why \emph{truth} as a value? What are the advantages? Maybe already bring on Joyce's accuracy argument for probabilism 
    \item Why only relevant truths? What is with the reduction to just interesting truths? Is it necessary, helpful? 
\end{itemize}

\subsection{Problems for the Veritistic account}
\begin{itemize}
    \item Objection from significance - Some true beliefs seem more valuable than other true beliefs. How can the truth monist account for that? - Novel Idea: Perhaps not truth, but INFORMATION is the fundamental epistemic value. Look up Definition of information by shannon. The amount of information a message carries is determined by the number of alternatives that can be ruled out by the receiver. The more alternatives ruled out, the higher the information! Compare: The value of belief has is determined by the amount of possibilities (or possible worlds) it rules out.
    \item Value Problem aka Meno Problem aka Swamping Problem (Show differences). Explain what it is and why it hurts veritism 
    \item Discuss how it could be solved in combination with accounts of justification (for example reliabilism) see Goldman/Olsson
    \item Own Discussion of a solution (perhaps from modal stability)
    \item The solution? Is there a solution without reliabilism, i.e. generic?
\end{itemize}
\begin{itemize}
    \item Problems that \textcite{Berker2013-BERETA-2} raises
    \item Answers that Goldman gives
    \item This does not solve the problem for the veritist in general. Even without deontic definition, this problem arises(?)
    \item Berker divides consequentialist theories in three parts: theory of final value, theory of overall value, and a deontic theory, as opposed to just divide into axiological and deontological considerations. 
    \item Explain IN DETAIL what this means. 
    \item For all (most) consequentialist theories a problem arises with firth-style cases. 
    \item Explain IN DETAIL what these cases are, what and how the problem arises for the consequentialist theories
    \item Explain shortly what the reliabilist (i.e.) Goldman replies.  
\end{itemize}




Veritism is the position that true belief is the only epistemic value. True belief and the avoidance of false belief are the only states of affairs that are epistemically valuable.

\section{Epistemic Normativity}

What is the exact distinction between Values, or goals, in general, and merely epistemic values, or goals? Epistemic Values pertain to something like Knowledge, Justification and Truth. But is that something that is really there or just some conventional means to divide and conquer the problem? Speaking from an evolutionary standpoint, it seems likely that we as humans are fitted with a natural goal of believing what maximizes our chances of survival. In most (if not all) cases, this goal coincides with believing what is true. But conceptually, these two are not the same. There may be cases where it would be best from a survival perspective, to believe that there is a predator in the surroundings, even when there is not, and so to believe falsely. So to purport the actual, final value of beliefs as being true seems, if nor fat-fetched, at least somewhat artificial. It is a very useful distinction, however, as the epistemologist can free her reasoning from any practical considerations that would otherwise be still on the table. However, if artificial, it might also not be what guides our intuitions about thought experiments. These intuitions may not share this artificial division of goals and subsume even epistemic evaluation under some, probably more practical, goals. I hold that these or some other not merely epistemic intuitions are what guides our evaluation in some cases. Consider a standard reply to a truth norm of the form: One should believe $p$ if and only if $p$ is true. This has the obvious consequence that one is required to believe every true proposition, no matter how menial or irrelevant to my current PRACTICAL purposes. I maintain that this relevance or interest is mostly practically motivated and hence should not be a part of any purely epistemic evaluation of the belief. From this perspective, every comparable true belief is comparably valuable in the epistemic sense. With comparable belief I mean beliefs that somehow have a similar epistemic status. They are of comparable generality and content. There may be differences in epistemic value, however, between a belief that is a very general (and true) law of nature and

Objection: 

\nocite{sep-value-theory,Goldman2002-GOLTUO-2,James1896-JAMTWT-19,James1907-JAMP,Zagzebski2004-ZAGEVM-2,Berker2013-BERETA-2,Berker2013-BERTRO-24,Goldman2015-GOLRVA,Berker2015-BERRTG-2}

\section{Conclusion}

\printbibliography

\end{document}