%!TEX program = xelatex
\documentclass[12pt,numbers=noenddot]{scrartcl}
% \usepackage[ngerman]{babel}
\usepackage[a4paper,lmargin={3cm},rmargin={3.5cm}, tmargin={2.5cm},bmargin = {2.5cm}]{geometry}
\usepackage{amsmath}
\usepackage{setspace}
\onehalfspacing

\usepackage[authordate]{biblatex-chicago}
\addbibresource{veritism.bib}

% \usepackage[square]{natbib}
% \usepackage{jurabib}
% \usepackage {algorithm2e}

% Package, das die Benutzung von Old Standard erlaubt
\usepackage{fontspec}

% \setmainfont{OldStandard-Regular.otf}[
% Path = /usr/local/texlive/texmf-local/opentype/,
% BoldFont = OldStandard-Bold.otf,
% ItalicFont = OldStandard-Italic.otf]


% \setmainfont{Old Standard TT}
\setmainfont{Baskerville}
\newfontfamily\osfamily{Old Standard TT}

% \settocfeature[toc][1]{entryhook}{\osfamily\bfseries\larger} 
% \settocfeature[toc][2]{entryhook}{\osfamily\normalsize} 
% \settocfeature[toc][3]{entryhook}{\osfamily\normalsize\itshape}

% \bibliographystyle{jurabib}
% \bibliographystyle{jureco} 
% \renewcommand*{\bibbtsep}{In: }
% \renewcommand*{\bibjtsep}{In: }

% \renewcommand*{\biblnfont}{\normalfont}
% \renewcommand*{\bibfnfont}{\normalfont}
% \renewcommand*{\bibelnfont}{\normalfont}
% \renewcommand*{\bibefnfont}{\normalfont}
% \renewcommand*{\bibtfont}{\textit}
% \renewcommand*{\bibbtfont}{\textit}
% \jurabibsetup{
%   bibformat=ibidem,
%   bibformat=compress,
%   dotafter=bibentry
% }

% \makeatletter
% % \renewcommand{\l@section}{\@dottedtocline{1}{1.5em}{2.6em}}
% \renewcommand{\l@subsection}{\@dottedtocline{2}{4.0em}{3.6em}}
% \renewcommand{\l@subsubsection}{\@dottedtocline{3}{7.4em}{4.5em}}
% \makeatother

% \addtokomafont{disposition}{\normalfont}

\setkomafont{subject}{\osfamily\small}
\addtokomafont{title}{\osfamily\bfseries}

\addtokomafont{publishers}{\normalfont\small}
\addtokomafont{date}{\normalfont\small}
\addtokomafont{author}{\normalfont\small}
\addtokomafont{descriptionlabel}{\normalfont\bfseries}

\addtokomafont{disposition}{\osfamily}

% \renewcommand{\thesection}{\Roman{section}} 
% \renewcommand{\thesubsection}{\thesection.\Roman{subsection}}


\subject{  University of Cologne \\
  Department of Philosophy \\
  Bachelor Thesis
  }
\title{Veritism and Epistemic Normativity}
\author{C. Friedrich}
\date{\today}


\usepackage{amsthm}
\newtheorem*{defi}{Definition}


\begin{document}
\begin{titlepage}
\maketitle

\abstract{\textbf{Abstract.} Can epistemic normativity be plausibly explained by appealing to truth as the single final epistemic value? [All structural elements and section titles are placeholders.]}

\thispagestyle{empty}
\end{titlepage}

\tableofcontents
\newpage

\section{Introduction}

In a certain sense, and one that, for example, \textcite{Berker2013-BERETA-2} holds, epistemology is an inherently normative endeavor. To Berker the most fundamental question of epistemology is, despite the etymological origins of its name, not concerned with what makes a belief count as knowledge but is instead purely normative: “What should I believe“?

In pondering this question the similarities of epistemology to the theory of normativity and especially ethics are strikingly obvious, if only at a first superficial glance. It makes a lot of sense then, if one is to undergo the project of viewing epistemology through the lens of normativity, to at first take a look at ethical theories of value anf normativity. While the properties of epistemic and ethical objects of interest may stop sooner than expected, the groundwork in ethical theory shouldn't be completely disregarded, either.

Go on and on
\begin{itemize}
    \item Where are we in epistemology? Why is a normative approach important / interesting?
    \item explain very clearly and in formidable prose what it is this paper is trying to do, how it will be done, and what it accomplished (obiovusly, write it at the end).
\end{itemize}

\begin{quote}
 I suggest that the primary function of cognition in human life is to acquire true rather than false beliefs about matters that are of interest or importance to us. \parencite[29]{Alston2005-ALSBJD}
\end{quote}

\begin{quote}
What makes us cognitive beings at all is our capacity for belief, and the goal of our distinctively cognitive endeavors is truth: we want our beliefs to correctly and accurately depict the world. \parencite[7]{Bonjour1985}
\end{quote}

[More quotes can be found in \textcite{Goldman2002-GOLTUO-2}.]
\footnote{As may be obvious, I presuppose a realist notion of truth. As perhaps less obvious, throughout this paper I assume that every proposition is either true or false and not neither or both. I won't discuss problems that arise for the veritist account if one supposes, say, dialetheism.}
\section{Veritism}

\begin{quote}
I don't know how to prove that the acquisition, retention, and use of true beliefs about matters that are of interest and/or importance is the most basic and most central goal of cognition. I don't know anything that is more obvious from which it could be derived. But I suggest that anyone can see its obviousness by reflecting on what would happen to human life if we were either without beliefs at all or if our beliefs were all or mostly false. Without beliefs we would be thrown back on instinct as our only guide to behavior. And as far as thought, understanding, linguistic communication, theorizing, science, art, religion-all the aspects of life that require higher-level cognitive processes-are concerned, we would be bereft of them altogether. And if we had beliefs but ones that were mostly false, we would constantly be led astray in our practical endeavors and would be unlikely to survive for long. \parencite[30]{Alston2005-ALSBJD}
\end{quote}

Why is it ever that anyone \emph{should} believe something, or is \emph{required} to believe, or is \emph{justified} in believing, or is \emph{rational} to believe? What is that yields some kind of normative requirement to the believer? A most natural answer, one that immediately springs to mind, is it's connection to the truth. You should believe that proposition because it is most likely to be true! (You are justified in believing that proposition because it is the only one that your evidence supports. It would be completely irrational to believe that other proposition, you don't have any reason to believe it.) What else than true belief is the epistemic goal of anyone in the business of believing anything, that is, a cognitive agent? It is this very intuitive thought that motivates postulating truth as an–or the–epistemic value.

When talking about values, the deontic or about normativity in general, it is helpful to look for established theories as a guideline. Naturally, these theories have been developed in the realm of practical philosophy, so that's where I'll turn for some clarificatory groundwork.

A good starting point to look for some guidance when talking about values are the fields of value theory or, more specifically, axiology.

\begin{itemize}
    \item Taxonomy of the normative
    \item Intrinsic Value vs Extrinsic value 
    \item Instrumental Value vs. Final value 
    \item Good vs. right 
    \item value vs goal (can be footnote)
    \item relation to the deontic: what does it mean for values to produce or guide or motivate or explain deontic norms? is justification inherently normative? reasons / rationality?
    \item Teleology / Consequentialism: A theory of Normativity with a central value / central values as its fundamental principle.
    \item axiological / deontic part. Perhaps list desiderata from ahlstrom, or taxonomy by Berker
    \item What is the bearer of epistemic value? -> states of affairs
\end{itemize}

\subsection{What is Veritism?}

\textcite[54]{Goldman2002-GOLTUO-2} defines veritism as  “... the unity if epistemic virtues in which the cardinal value, or underlying motif, is something like true, or accurate, belief”. Borrowing the terminology from Goldman, \textcite[360]{Berker2013-BERETA-2} proposes to label as veritism the position that „ ... our only epistemic goals are (i) the accumulation of true beliefs and (ii) the avoidance of false beliefs”. \textcite{Zagzebski2004-ZAGEVM-2} has a different name for a very similar position, what she calls epistemic value monism: “Any epistemic value other than the truth of a belief derives from the good of truth“. 

The unifying theme of these quotes is apparent: What’s central to all three accounts clearly is the notion of \emph{true belief}. True, or accurate, belief has to be a value, goal, function or motif with a distinct role in the theory. This role is determined as cardinal, single, primary and non-derivative. In other words, veritism says that true belief has to be regarded as the only \emph{final} value. No other epistemic concept like knowledge or justification has final epistemic value. It is important to note, however, that this does \emph{not} entail that other epistemic concepts have \emph{no} epistemic value, they just don’t have \emph{final} epistemic value.

\subsection{Simple Veritism}
That, basically, is the gist of the position: veritism claims that true belief is the only epistemic value or goal. 

\begin{description}
    \item[Simple Veritism:] True belief is the only state of affairs that has final epistemic value.
\end{description}

\begin{itemize}
    \item of course, this is still underspecified. What exactly is of value? Having as many true beliefs as possible? How are true and false aggregated to accommodate something like a total value?
    \item add a comparative notion to it.
\end{itemize}

\subsection{How Plausible Is Simple Veritism?}

Can this account explain intuition about scenarios concerning epistemic value?  A clear-cut case: two propositions $P$ and $Q$ where $P$ is true and $Q$ is false–it couldn't be more obvious: believing $P$ is more valuable than believing $Q$. This very natural thought directly leads to a comparative notion which holds that true belief is more valuable or better than false belief. Simple enough.

Now what if we want to evaluate the doxastic system of a person, or her intellectual attainment, as \textcite[12]{Goldman2002-GOLTUO-2} puts it? This seems just as straightforward. The more true beliefs a person has, the more valuable the position that she occupies. So no matter the epistemic situation she am in, by acquiring more true beliefs she can better her epistemic standing. So far, so good.

All is not so peachy, though. Consider these two:

\emph{Gullible Gilbert} believes everything anyone tells tells him, anything he thinks about, in general any proposition he encounters.

\emph{Precise Priscilla} is more careful in forming her beliefs. She carefully weighs her evidence and, as a result, has a lot fewer beliefs than Gilbert, but her beliefs are largely true.

Intuitively, it is obvious that Priscilla occupies the more valuable epistemic state. Yet on the account just sketched, Gilbert amasses lots and lots of beliefs, among these many true beliefs. Hence his epistemic situation is immensely valuable. Priscilla on the other hand has fewer true beliefs to show for, her epistemic situation is therefore less valuable. This is a terrible result for the simple account.

Gilbert’s situation is an instance of what \textcite[360]{Berker2013-BERETA-2} calls \emph{epistemic recklessness}. 

\subsection{Dualistic Veritism}

To avoid this result, it seems only natural to expand the epistemic goal to consist of to goals, really: ”...the twin goals of acquiring true beliefs and avoiding false ones“ \parencite[339]{Berker2013-BERETA-2}. This dualistic rendition of epistemic value has already been proposed by \textcite[17]{James1896-JAMTWT-19}, who coined the phrase: ”Believe truth! Shun error!“.

\begin{description}
    \item[Dualistic Veritism] True belief and the avoidance of error are the only two states of affairs that have final epistemic value.
\end{description}

This account fares better with respect to Gilbert and Priscilla. As Gilbert is in error quite often and thereby diminishes the value of his epistemic situation, Priscilla’s carefulness now brims with epistemic value. So this intuition, at least, can be explained. Splendid!

\subsection{How Plausible Is Dualistic Veritism?}

But let's revisit one of the key merits of veritism at the core of any deontic framework. One central motivation is to unify all epistemic evaluation. This approach reaches as far back as Socrates, who proclaimed: ”Virtue is one!” and meant it quite literally \parencite{penner1973}, entailing that what Socrates regarded as virtues where really all the same thing. So bravery, wisdom, temperance, justice, piety and even knowledge amount to something equivalent, if not identical.
\textcite{Goldman2002-GOLTUO-2} picks up the ball in his self-appointed task to unify all epistemic virtues\footnote{Now, virtues are not the same thing as values or goals, and I don't want to get into a discussion about virtue epistemology at all here. What Goldman concedes, though, is that his conception of epistemic virtues builds upon or at the very least is compatible with that form of consequentialism value-directed accounts present us with.}. As one might think, he is not so quick to drop value monism to make it two, that is to switch to a dualistic account, which would pretty much mean to give up the idea of a single unifying epistemic virtue.

\subsection{Goldman's Accuracy Veritism}
\textcite[58]{Goldman2002-GOLTUO-2} proposes an a little more complex form of veritism which takes its motivation from the model of partial belief, or degree of belief, or levels of confidence, or credences. Here, not mere true belief has final epistemic value but the value of a certain credence in a true proposition derives its value as a function of its degree: maximal value if the degree is maximal, and minimal value if the degree is minimal. This yields a slightly different comparative notion than before: A degree of belief in a true proposition is more valuable than another degree of belief simply if it is higher. Given a true proposition $P$, your credence of $0.9$ is more valuable than mine of $0.4$, simple as that.
Goldman then supposes a workable way to translate degrees of beliefs into full beliefs, suspension of belief, and disbelieve, by some threshold measure that is left unspecified. He can then compare: ”... believing a truth carries more veritistic value than suspension of judgment; and suspension of judgment carries more veritistic value than disbelief“. This leads to a comparative notion:

\begin{quote}
    If a person regularly has a high level of belief in the true propositions she considers or takes an interest in, then she qualifies as “well‐informed.” Someone with intermediate levels of belief on many such questions, amounting to “no opinion,” qualifies as uninformed, or ignorant. And someone who has very low levels of belief for true propositions—or, equivalently, high levels of belief for false propositions—is seriously misinformed. \parencite[12]{Goldman2002-GOLTUO-2}
\end{quote}

In this way, Goldman concludes, the veritistic account can accommodate for challenges that stipulate two epistemic goals instead of one. In particular, he claims that that having very low levels of belief for true propositions is \emph{equivalent} to having high level of belief in false propositions, at least insofar as the epistemic value is concerned. On this account, a credence of $0.2$ in a true proposition has the same epistemic value as a credence of $0.8$ in a false proposition, and both counts equally towards a person being  seriously misinformed. For Goldman, this notion of comparative value does justice to our intuitions regarding relevant cases.

\subsubsection{How Plausible Is Goldman's Accuracy Veritism?}

However, as I see it, there are problems with this translation of conclusions drawn with the notion of degree of belief to the notion of full belief that Goldman does not seem to take into account.

Goldman claims that disbelieving a proposition $P$ is equivalent to believing a proposition non-$P$, so that I disbelieve a proposition $P$ if and only if I believe a proposition non-$P$. That is, whenever it is true that i disbelieve $P$, it is also true that I believe non-$P$, and whenever it is false that i disbelieve $P$, it is also false that i believe non-$P$, and \emph{vice versa}. I presume this is a consequence from the translation of the notion of degree of beliefs to the notion of full beliefs. Given scaling of degree of beliefs on the unit interval, whenever I believe $P$ to the degree $c$, I also believe non-$P$ to the degree $1-c$.  For example: I am pretty unconvinced of $P$, my credence is $0.1$. Hence, my credence in non-$P$ is $0.9$. Translated back to full beliefs, this would then amount to belief and disbelieve, if $c$ is sufficiently high for full belief. So Goldman's claim about equivalence would follow. What shouldn't be disregarded, though, is that this only applies to agents that obey the axioms of probability theory, in many theories a necessary condition on rationality of degree of beliefs. It is still very much conceptually possible to believe $P$ to the degree $0.8$ \emph{and also} believe non-$P$ to the degree, say $0.7$. I wouldn't be quite rational to do so, of course, but it is certainly possible. Now, Goldman does not state conceptual identity, granted. Equivalence is still a very strong claim, but: my believing $P$ to the degree $c$ \emph{does not entail} my believing non-$P$ to the degree $1-c$ (or vice versa). Similarly, with full beliefs, I surely can disbelieve $P$ and also not believe non-$P$, that is, disbelieve non-$P$ or suspend judgment towards non-$P$. There is no entailment relation between either of these doxastic attitudes. Would it be so, I could never believe a direct contradiction, not only not rationally, but not with good reasons, either\footnote{As in the case of certain paradoxes, for example the liar paradox, as one might argue \textcite{Priest1998-PRIWIS}}.

Goldman could still retreat to the position that the \emph{epistemic value} of disbelieving $P$ is equivalent to the \emph{epistemic value} of believing non-$P$. Let's suppose $P$ is false, and hence non-$P$ true, then on this account, disbelieving $P$ has epistemic value–the corresponding credence is very close to the actual truth-value– and believing non-$P$ has value, since one believes a true proposition, namely non-$P$. So what can be accounted for is the dual value of disbelieving a falsehood and believing a truth. Consider the case:

\emph{Skeptical Stephanie} is in a very unfortunate situation. Her questionable colleagues all present her with unconvincing propositions. As a result, she comes to disbelieve most of them. And rightly so, all of them are false.

We intuitively want to say that Stephanie's disbeliefs contribute something epistemically valuable to her situation. Yet, on the simple account of veritism, I could only ascribe value to her disbeliefs if I presume the equivalence stated by Goldman, that

How much hangs on this in Goldman's argument?

- Claims that disbelieving a proposition p is equivalent to believing a proposition p. That is not correct. This comes from proabilistic accounts. given the plausible principle that for every proposition, one can have only one doxastic attitude at most towards it.
Goldman claims that, in the case of full belief, disbelieving a proposition p is equivalent to believing a proposition p. Lets assume that

- i suggest that the translation to a model of full belief does not work in all respects: A degree of belief of 0.5 is not the same a suspension of judgment, which does not entail a doxastic attitude towards a proposition \footnote{I owe this point to \textcite{Balg2018} }

- This Works for cases where we compare doxastic attidues over a given, fixed set of opinionated propositions.

- This does NOT solve epistemic recklessness: he would need something like disvalue for false beliefs that take away from the overall value. given a simple additional model of value



In the application of this account to full belief Goldman makes two assumptions that in my eyes do not do the work he expects them to do. First, Goldman makes the claim that disbelieving a proposition is equivalent to believing the proposition non-$p$. He doesn't claim identity, which would not make much sense, since these two are obviously different things. Yet he claims equivalence to the effect that believing $p$ can't occur without simultaneously also disbelieving non-$p$. This seems to be a consequence from the application of the credence-based approach, in the following way: Given my credence in $p$ of, say, $0.9$, 
% subsection subsection_name (end)

\begin{itemize}
    \item epistemic value vs practical value
    \item how does it relate to the deontic? What does that mean, deontic epistemology? (Steup?)
    \item Truth-Value-Monism / Veritism. The same? List different definitions. Goldman / Berker / Zagzebski 
    \item Truth-Value-Monism vs. Pluralism about values 
    \item Other-Value-Monisms (Knowledge, ...) 
    \item TVM vs. Non-Value-based accounts of normativity 
    Deontological account of justification has the problem of explaining intuitions of the form you should believe it because its most likely to be true!

    Deontological accounts also need to provide an explanation for the problems posed by a false doxastic voluntarism and ought-implies-can. Veritism gets evaluative notions for free! Docastic voluntarism still is a problem for deontic requirements, though. Strategies: Flat-out deny deontic (regulative) requirements, or adopt one of the deontological accounts position and distinguish between different forms of doxastic voluntarism or deny ought-implies-can in one of its different forms. (Perhaps) we'll see later what these strategies exactly amount to .
    \item Why \emph{truth} as a value? What are the advantages? Maybe already bring on Joyce's accuracy argument for probabilism 
    \item Why only relevant truths? What is with the reduction to just interesting truths? Is it necessary, helpful? 
\end{itemize}

\subsection{Problems for the Veritistic account}
\begin{itemize}
    \item Objection from significance - Some true beliefs seem more valuable than other true beliefs.

    Compare these two beliefs:
    \begin{enumerate}
        \item The universe is expanding at an accelerating rate.
        \item The number of people ever to have visited the David Hume memorial up until now is even. \footnote{Examples from \textcite{Ahlstrom-Vij2013}}
    \end{enumerate}


    How can the truth monist account for that? - Novel Idea: Perhaps not truth, but INFORMATION is the fundamental epistemic value. Look up Definition of information by shannon. The amount of information a message carries is determined by the number of alternatives that can be ruled out by the receiver. The more alternatives ruled out, the higher the information! Compare: The value of belief has is determined by the amount of possibilities (or possible worlds or other propositions) it rules out. This would explain why some true beliefs are more valuable than others, they carry more information. this carries the air of popper's verisimilitude for scientific theories, that should somehow measure its explanatory power (see keuth, my own paper).
    Moreover, it might even explain why justified belief is more valuable than mere true belief. Does it? How? Somehow the modal stability adds to the information contained? That seems a little far fetched but perhaps its worth a try. 
    What it DOES do is take the reek of practical value that comes with interest and significance! Argh! We veritists want delicious, pure epistemic value. If it is possible to define information in a way that it does not pertain to any practical evaluation and can be evaluated in epistemic terms only, its an account of significance or interest of true beliefs that is purely epistemic.

    \item Value Problem aka Meno Problem aka Swamping Problem (Show differences). Explain what it is and why it hurts veritism 
    \item Discuss how it could be solved in combination with accounts of justification (for example reliabilism) see Goldman/Olsson
    \item Own Discussion of a solution (perhaps from modal stability)
    \item The solution? Is there a solution without reliabilism, i.e. generic?
\end{itemize}
\begin{itemize}
    \item Problems that \textcite{Berker2013-BERETA-2} raises
    \item Answers that Goldman gives
    \item This does not solve the problem for the veritist in general. Even without deontic definition, this problem arises(?)
    \item Berker divides consequentialist theories in three parts: theory of final value, theory of overall value, and a deontic theory, as opposed to just divide into axiological and deontological considerations. 
    \item Explain IN DETAIL what this means. 
    \item For all (most) consequentialist theories a problem arises with firth-style cases. 
    \item Explain IN DETAIL what these cases are, what and how the problem arises for the consequentialist theories
    \item Explain shortly what the reliabilist (i.e.) Goldman replies.  
\end{itemize}

\section{Epistemic Normativity}

\begin{itemize}
    \item What is the connection to the normative, exactly? Is it just justification, or can we „deduce“ actual NORMS from veritism? 
    \item If so, how plausible are they? Perhaps see Chisholm (quoted in \textcite{Goldman2002-GOLTUO-2})
\end{itemize}

What is the exact distinction between Values, or goals, in general, and merely epistemic values, or goals? Epistemic Values pertain to something like Knowledge, Justification and Truth. But is that something that is really there or just some conventional means to divide and conquer the problem? Speaking from an evolutionary standpoint, it seems likely that we as humans are fitted with a natural goal of believing what maximizes our chances of survival. In most (if not all) cases, this goal coincides with believing what is true. But conceptually, these two are not the same. There may be cases where it would be best from a survival perspective, to believe that there is a predator in the surroundings, even when there is not, and so to believe falsely. So to purport the actual, final value of beliefs as being true seems, if nor fat-fetched, at least somewhat artificial. It is a very useful distinction, however, as the epistemologist can free her reasoning from any practical considerations that would otherwise be still on the table. However, if artificial, it might also not be what guides our intuitions about thought experiments. These intuitions may not share this artificial division of goals and subsume even epistemic evaluation under some, probably more practical, goals. I hold that these or some other not merely epistemic intuitions are what guides our evaluation in some cases. Consider a standard reply to a truth norm of the form: One should believe $p$ if and only if $p$ is true. This has the obvious consequence that one is required to believe every true proposition, no matter how menial or irrelevant to my current PRACTICAL purposes. I maintain that this relevance or interest is mostly practically motivated and hence should not be a part of any purely epistemic evaluation of the belief. From this perspective, every comparable true belief is comparably valuable in the epistemic sense. With comparable belief I mean beliefs that somehow have a similar epistemic status. They are of comparable generality and content. There may be differences in epistemic value, however, between a belief that is a very general (and true) law of nature and

Objection: 

\nocite{sep-value-theory,Goldman2002-GOLTUO-2,James1896-JAMTWT-19,James1907-JAMP,Zagzebski2004-ZAGEVM-2,Berker2013-BERETA-2,Berker2013-BERTRO-24,Goldman2015-GOLRVA,Berker2015-BERRTG-2}

\section{Conclusion}

\printbibliography

\end{document}